% Options for packages loaded elsewhere
% Options for packages loaded elsewhere
\PassOptionsToPackage{unicode}{hyperref}
\PassOptionsToPackage{hyphens}{url}
\PassOptionsToPackage{dvipsnames,svgnames,x11names}{xcolor}
%
\documentclass[
  spanish,
  a4paper,
  oneside,
  shorthands=off]{scrbook}
\usepackage{xcolor}
\usepackage[top=25mm,bottom=25mm,left=30mm,right=20mm]{geometry}
\usepackage{amsmath,amssymb}
\setcounter{secnumdepth}{5}
\usepackage{iftex}
\ifPDFTeX
  \usepackage[T1]{fontenc}
  \usepackage[utf8]{inputenc}
  \usepackage{textcomp} % provide euro and other symbols
\else % if luatex or xetex
  \usepackage{unicode-math} % this also loads fontspec
  \defaultfontfeatures{Scale=MatchLowercase}
  \defaultfontfeatures[\rmfamily]{Ligatures=TeX,Scale=1}
\fi
\usepackage{lmodern}
\ifPDFTeX\else
  % xetex/luatex font selection
  \setmainfont[]{Times New Roman}
\fi
% Use upquote if available, for straight quotes in verbatim environments
\IfFileExists{upquote.sty}{\usepackage{upquote}}{}
\IfFileExists{microtype.sty}{% use microtype if available
  \usepackage[]{microtype}
  \UseMicrotypeSet[protrusion]{basicmath} % disable protrusion for tt fonts
}{}
\makeatletter
\@ifundefined{KOMAClassName}{% if non-KOMA class
  \IfFileExists{parskip.sty}{%
    \usepackage{parskip}
  }{% else
    \setlength{\parindent}{0pt}
    \setlength{\parskip}{6pt plus 2pt minus 1pt}}
}{% if KOMA class
  \KOMAoptions{parskip=half}}
\makeatother
% Make \paragraph and \subparagraph free-standing
\makeatletter
\ifx\paragraph\undefined\else
  \let\oldparagraph\paragraph
  \renewcommand{\paragraph}{
    \@ifstar
      \xxxParagraphStar
      \xxxParagraphNoStar
  }
  \newcommand{\xxxParagraphStar}[1]{\oldparagraph*{#1}\mbox{}}
  \newcommand{\xxxParagraphNoStar}[1]{\oldparagraph{#1}\mbox{}}
\fi
\ifx\subparagraph\undefined\else
  \let\oldsubparagraph\subparagraph
  \renewcommand{\subparagraph}{
    \@ifstar
      \xxxSubParagraphStar
      \xxxSubParagraphNoStar
  }
  \newcommand{\xxxSubParagraphStar}[1]{\oldsubparagraph*{#1}\mbox{}}
  \newcommand{\xxxSubParagraphNoStar}[1]{\oldsubparagraph{#1}\mbox{}}
\fi
\makeatother

\usepackage{color}
\usepackage{fancyvrb}
\newcommand{\VerbBar}{|}
\newcommand{\VERB}{\Verb[commandchars=\\\{\}]}
\DefineVerbatimEnvironment{Highlighting}{Verbatim}{commandchars=\\\{\}}
% Add ',fontsize=\small' for more characters per line
\usepackage{framed}
\definecolor{shadecolor}{RGB}{241,243,245}
\newenvironment{Shaded}{\begin{snugshade}}{\end{snugshade}}
\newcommand{\AlertTok}[1]{\textcolor[rgb]{0.68,0.00,0.00}{#1}}
\newcommand{\AnnotationTok}[1]{\textcolor[rgb]{0.37,0.37,0.37}{#1}}
\newcommand{\AttributeTok}[1]{\textcolor[rgb]{0.40,0.45,0.13}{#1}}
\newcommand{\BaseNTok}[1]{\textcolor[rgb]{0.68,0.00,0.00}{#1}}
\newcommand{\BuiltInTok}[1]{\textcolor[rgb]{0.00,0.23,0.31}{#1}}
\newcommand{\CharTok}[1]{\textcolor[rgb]{0.13,0.47,0.30}{#1}}
\newcommand{\CommentTok}[1]{\textcolor[rgb]{0.37,0.37,0.37}{#1}}
\newcommand{\CommentVarTok}[1]{\textcolor[rgb]{0.37,0.37,0.37}{\textit{#1}}}
\newcommand{\ConstantTok}[1]{\textcolor[rgb]{0.56,0.35,0.01}{#1}}
\newcommand{\ControlFlowTok}[1]{\textcolor[rgb]{0.00,0.23,0.31}{\textbf{#1}}}
\newcommand{\DataTypeTok}[1]{\textcolor[rgb]{0.68,0.00,0.00}{#1}}
\newcommand{\DecValTok}[1]{\textcolor[rgb]{0.68,0.00,0.00}{#1}}
\newcommand{\DocumentationTok}[1]{\textcolor[rgb]{0.37,0.37,0.37}{\textit{#1}}}
\newcommand{\ErrorTok}[1]{\textcolor[rgb]{0.68,0.00,0.00}{#1}}
\newcommand{\ExtensionTok}[1]{\textcolor[rgb]{0.00,0.23,0.31}{#1}}
\newcommand{\FloatTok}[1]{\textcolor[rgb]{0.68,0.00,0.00}{#1}}
\newcommand{\FunctionTok}[1]{\textcolor[rgb]{0.28,0.35,0.67}{#1}}
\newcommand{\ImportTok}[1]{\textcolor[rgb]{0.00,0.46,0.62}{#1}}
\newcommand{\InformationTok}[1]{\textcolor[rgb]{0.37,0.37,0.37}{#1}}
\newcommand{\KeywordTok}[1]{\textcolor[rgb]{0.00,0.23,0.31}{\textbf{#1}}}
\newcommand{\NormalTok}[1]{\textcolor[rgb]{0.00,0.23,0.31}{#1}}
\newcommand{\OperatorTok}[1]{\textcolor[rgb]{0.37,0.37,0.37}{#1}}
\newcommand{\OtherTok}[1]{\textcolor[rgb]{0.00,0.23,0.31}{#1}}
\newcommand{\PreprocessorTok}[1]{\textcolor[rgb]{0.68,0.00,0.00}{#1}}
\newcommand{\RegionMarkerTok}[1]{\textcolor[rgb]{0.00,0.23,0.31}{#1}}
\newcommand{\SpecialCharTok}[1]{\textcolor[rgb]{0.37,0.37,0.37}{#1}}
\newcommand{\SpecialStringTok}[1]{\textcolor[rgb]{0.13,0.47,0.30}{#1}}
\newcommand{\StringTok}[1]{\textcolor[rgb]{0.13,0.47,0.30}{#1}}
\newcommand{\VariableTok}[1]{\textcolor[rgb]{0.07,0.07,0.07}{#1}}
\newcommand{\VerbatimStringTok}[1]{\textcolor[rgb]{0.13,0.47,0.30}{#1}}
\newcommand{\WarningTok}[1]{\textcolor[rgb]{0.37,0.37,0.37}{\textit{#1}}}

\usepackage{longtable,booktabs,array}
\usepackage{calc} % for calculating minipage widths
% Correct order of tables after \paragraph or \subparagraph
\usepackage{etoolbox}
\makeatletter
\patchcmd\longtable{\par}{\if@noskipsec\mbox{}\fi\par}{}{}
\makeatother
% Allow footnotes in longtable head/foot
\IfFileExists{footnotehyper.sty}{\usepackage{footnotehyper}}{\usepackage{footnote}}
\makesavenoteenv{longtable}
\usepackage{graphicx}
\makeatletter
\newsavebox\pandoc@box
\newcommand*\pandocbounded[1]{% scales image to fit in text height/width
  \sbox\pandoc@box{#1}%
  \Gscale@div\@tempa{\textheight}{\dimexpr\ht\pandoc@box+\dp\pandoc@box\relax}%
  \Gscale@div\@tempb{\linewidth}{\wd\pandoc@box}%
  \ifdim\@tempb\p@<\@tempa\p@\let\@tempa\@tempb\fi% select the smaller of both
  \ifdim\@tempa\p@<\p@\scalebox{\@tempa}{\usebox\pandoc@box}%
  \else\usebox{\pandoc@box}%
  \fi%
}
% Set default figure placement to htbp
\def\fps@figure{htbp}
\makeatother


% definitions for citeproc citations
\NewDocumentCommand\citeproctext{}{}
\NewDocumentCommand\citeproc{mm}{%
  \begingroup\def\citeproctext{#2}\cite{#1}\endgroup}
\makeatletter
 % allow citations to break across lines
 \let\@cite@ofmt\@firstofone
 % avoid brackets around text for \cite:
 \def\@biblabel#1{}
 \def\@cite#1#2{{#1\if@tempswa , #2\fi}}
\makeatother
\newlength{\cslhangindent}
\setlength{\cslhangindent}{1.5em}
\newlength{\csllabelwidth}
\setlength{\csllabelwidth}{3em}
\newenvironment{CSLReferences}[2] % #1 hanging-indent, #2 entry-spacing
 {\begin{list}{}{%
  \setlength{\itemindent}{0pt}
  \setlength{\leftmargin}{0pt}
  \setlength{\parsep}{0pt}
  % turn on hanging indent if param 1 is 1
  \ifodd #1
   \setlength{\leftmargin}{\cslhangindent}
   \setlength{\itemindent}{-1\cslhangindent}
  \fi
  % set entry spacing
  \setlength{\itemsep}{#2\baselineskip}}}
 {\end{list}}
\usepackage{calc}
\newcommand{\CSLBlock}[1]{\hfill\break\parbox[t]{\linewidth}{\strut\ignorespaces#1\strut}}
\newcommand{\CSLLeftMargin}[1]{\parbox[t]{\csllabelwidth}{\strut#1\strut}}
\newcommand{\CSLRightInline}[1]{\parbox[t]{\linewidth - \csllabelwidth}{\strut#1\strut}}
\newcommand{\CSLIndent}[1]{\hspace{\cslhangindent}#1}

\ifLuaTeX
\usepackage[bidi=basic]{babel}
\else
\usepackage[bidi=default]{babel}
\fi
\ifPDFTeX
\else
\babelfont{rm}[]{Times New Roman}
\fi
% get rid of language-specific shorthands (see #6817):
\let\LanguageShortHands\languageshorthands
\def\languageshorthands#1{}


\setlength{\emergencystretch}{3em} % prevent overfull lines

\providecommand{\tightlist}{%
  \setlength{\itemsep}{0pt}\setlength{\parskip}{0pt}}



 


% ---------- Paquetes tipograficos y microajustes ----------
\usepackage{microtype}        % mejor interletrado y justificado
\usepackage{csquotes}         % comillas tipograficas
\usepackage{iftex}
\ifPDFTeX\else
  % Seleccion de fuentes solo cuando XeTeX/LuaTeX esta disponible.
  \IfFontExistsTF{Times New Roman}{
    \setmainfont{Times New Roman}
  }{
    \IfFontExistsTF{TeX Gyre Termes}{
      \setmainfont{TeX Gyre Termes}
    }{
      \IfFontExistsTF{Latin Modern Roman}{\setmainfont{Latin Modern Roman}}{}
    }
  }
  \IfFontExistsTF{Times New Roman}{
    \setsansfont{Times New Roman}
  }{
    \IfFontExistsTF{TeX Gyre Heros}{
      \setsansfont{TeX Gyre Heros}
    }{
      \IfFontExistsTF{Latin Modern Sans}{\setsansfont{Latin Modern Sans}}{}
    }
  }
  \IfFontExistsTF{Times New Roman}{
    \setmonofont{Times New Roman}
  }{
    \IfFontExistsTF{TeX Gyre Cursor}{
      \setmonofont{TeX Gyre Cursor}
    }{
      \IfFontExistsTF{Latin Modern Mono}{\setmonofont{Latin Modern Mono}}{}
    }
  }
\fi
\usepackage{enumitem}         % listas compactas
\setlist{itemsep=.2em, topsep=.2em}

% ---------- KOMA-Script: estilo de titulos y espaciado ----------
\KOMAoptions{
  headings=big,
  parskip=half,
  fontsize=12pt,
  appendixprefix=true
}
\setlength{\parindent}{1.5em}
\setlength{\parskip}{0.6em}
\usepackage{indentfirst}
\usepackage{setspace}
\setstretch{1.5}


% ---------- Encabezados y pies (scrlayer-scrpage) ----------
\usepackage[automark,headsepline]{scrlayer-scrpage}
\clearpairofpagestyles
\automark[chapter]{chapter}
\ihead{\pagemark}
\ohead{\itshape\headmark}
\setheadsepline{0.4pt}
\renewcommand*{\chaptermarkformat}{}
\renewcommand*{\chapterpagestyle}{scrheadings}
\pagestyle{scrheadings}
\makeatletter
\let\ps@plain\ps@scrheadings
\makeatother
\cfoot{}

% ---------- Hipervinculos mas sobrios ----------
\usepackage{hyperref}
\hypersetup{
  colorlinks=true,
  linkcolor=blue,
  citecolor=blue,
  urlcolor=blue,
  pdfauthor={\@author},
  pdftitle={\@title}
}

% ---------- Leyendas de figuras/tablas ----------
\usepackage[labelfont=bf,textfont=it]{caption}
\captionsetup{
  skip=8pt
}

% ---------- Tabla de contenidos ----------
\KOMAoptions{toc=graduated}
\RedeclareSectionCommand[tocnumwidth=3em]{chapter}
\RedeclareSectionCommand[tocindent=3.25em,tocnumwidth=2.8em]{section}
\RedeclareSectionCommand[tocindent=6.5em,tocnumwidth=2.5em]{subsection}
\makeatletter
\newcommand*{\tocdotfill}{\leavevmode\leaders\hbox to .6em{\hss.\hss}\hfill}
\makeatother
\RedeclareSectionCommand[toclinefill=\tocdotfill]{chapter}
\RedeclareSectionCommand[toclinefill=\tocdotfill]{section}
\RedeclareSectionCommand[toclinefill=\tocdotfill]{subsection}
\renewcommand*{\contentsname}{Tabla de contenido}
\setkomafont{chapterentry}{\normalfont}
\setkomafont{chapterentrypagenumber}{\normalfont}

% ---------- Viudas/Huerfanas y cortes de pagina ----------
\clubpenalty=10000
\widowpenalty=10000
\displaywidowpenalty=10000

% ---------- Entorno abstract para scrbook ----------
\providecommand{\abstractname}{Resumen}
\makeatletter
\@ifundefined{abstract}{
  \newenvironment{abstract}{
    \cleardoublepage
    \thispagestyle{plain}
    \null\vfill
    \begin{center}
      {\bfseries\Large \abstractname\par}
    \end{center}\vspace{1em}
    \begingroup
  }{
    \par\endgroup
    \vfill\null
    \cleardoublepage
  }
}{}
\makeatother

% ---------- Soporte de subtitulo desde YAML ----------
\makeatletter
\providecommand{\subtitle}[1]{\gdef\@subtitle{#1}}
\providecommand{\@subtitle}{}
\makeatother

% --- Desactivar portada y abstract automaticos de Pandoc (PDF) ---
\AtBeginDocument{\let\maketitle\relax}
\renewenvironment{abstract}{}{}
\providecommand{\appendixname}{}
\providecommand{\appendixtocname}{}
\providecommand{\appendixpagename}{}
\renewcommand*{\appendixname}{Anexo}
\renewcommand*{\appendixtocname}{Anexos}
\renewcommand*{\appendixpagename}{Anexos}


% ---------- Indicadores para LOF/LOT condicionales ----------
\newif\iffacsofigexists
\newif\iffacsotableexists
\InputIfFileExists{\jobname.facsoflags}{}{}
\newwrite\FacsoFlagStream
\AtEndDocument{%
  \immediate\openout\FacsoFlagStream=\jobname.facsoflags
  \ifnum\value{figure}>0
    \immediate\write\FacsoFlagStream{\string\facsofigexiststrue}
  \else
    \immediate\write\FacsoFlagStream{\string\facsofigexistsfalse}
  \fi
  \ifnum\value{table}>0
    \immediate\write\FacsoFlagStream{\string\facsotableexiststrue}
  \else
    \immediate\write\FacsoFlagStream{\string\facsotableexistsfalse}
  \fi
  \immediate\closeout\FacsoFlagStream
}
\hypersetup{
  colorlinks=true,
  linkcolor=blue,
  urlcolor=blue,
  citecolor=blue
}
\makeatletter
\@ifpackageloaded{tcolorbox}{}{\usepackage[skins,breakable]{tcolorbox}}
\@ifpackageloaded{fontawesome5}{}{\usepackage{fontawesome5}}
\definecolor{quarto-callout-color}{HTML}{909090}
\definecolor{quarto-callout-note-color}{HTML}{0758E5}
\definecolor{quarto-callout-important-color}{HTML}{CC1914}
\definecolor{quarto-callout-warning-color}{HTML}{EB9113}
\definecolor{quarto-callout-tip-color}{HTML}{00A047}
\definecolor{quarto-callout-caution-color}{HTML}{FC5300}
\definecolor{quarto-callout-color-frame}{HTML}{acacac}
\definecolor{quarto-callout-note-color-frame}{HTML}{4582ec}
\definecolor{quarto-callout-important-color-frame}{HTML}{d9534f}
\definecolor{quarto-callout-warning-color-frame}{HTML}{f0ad4e}
\definecolor{quarto-callout-tip-color-frame}{HTML}{02b875}
\definecolor{quarto-callout-caution-color-frame}{HTML}{fd7e14}
\makeatother
\makeatletter
\@ifpackageloaded{bookmark}{}{\usepackage{bookmark}}
\makeatother
\makeatletter
\@ifpackageloaded{caption}{}{\usepackage{caption}}
\AtBeginDocument{%
\ifdefined\contentsname
  \renewcommand*\contentsname{Tabla de contenidos}
\else
  \newcommand\contentsname{Tabla de contenidos}
\fi
\ifdefined\listfigurename
  \renewcommand*\listfigurename{Listado de Figuras}
\else
  \newcommand\listfigurename{Listado de Figuras}
\fi
\ifdefined\listtablename
  \renewcommand*\listtablename{Listado de Tablas}
\else
  \newcommand\listtablename{Listado de Tablas}
\fi
\ifdefined\figurename
  \renewcommand*\figurename{Lista de figuras}
\else
  \newcommand\figurename{Lista de figuras}
\fi
\ifdefined\tablename
  \renewcommand*\tablename{Lista de tablas}
\else
  \newcommand\tablename{Lista de tablas}
\fi
}
\@ifpackageloaded{float}{}{\usepackage{float}}
\floatstyle{ruled}
\@ifundefined{c@chapter}{\newfloat{codelisting}{h}{lop}}{\newfloat{codelisting}{h}{lop}[chapter]}
\floatname{codelisting}{Listado}
\newcommand*\listoflistings{\listof{codelisting}{Listado de Listados}}
\makeatother
\makeatletter
\makeatother
\makeatletter
\@ifpackageloaded{caption}{}{\usepackage{caption}}
\@ifpackageloaded{subcaption}{}{\usepackage{subcaption}}
\makeatother
\usepackage{bookmark}
\IfFileExists{xurl.sty}{\usepackage{xurl}}{} % add URL line breaks if available
\urlstyle{same}
\hypersetup{
  pdftitle={IA al servicio pedagógico},
  pdfauthor={Katherine Aravena Herrera; Juan Castaño Giraldo; Paula Cerda Torres; Manuel Sierras},
  pdflang={es},
  colorlinks=true,
  linkcolor={black},
  filecolor={Maroon},
  citecolor={blue},
  urlcolor={blue},
  pdfcreator={LaTeX via pandoc}}


\title{IA al servicio pedagógico}
\usepackage{etoolbox}
\makeatletter
\providecommand{\subtitle}[1]{% add subtitle to \maketitle
  \apptocmd{\@title}{\par {\large #1 \par}}{}{}
}
\makeatother
\subtitle{Herramientas prácticas para el trabajo docente}
\author{Katherine Aravena Herrera \and Juan Castaño Giraldo \and Paula
Cerda Torres \and Manuel Sierras}
\date{12 de enero de 2026}
\begin{document}
\frontmatter
\maketitle

% --- title-pdf.tex personalizado para portada formal ---
\newcommand{\CoverFont}{
  \ifdefined\fontspec
    \fontspec{Times New Roman}
  \else
    \fontfamily{ptm}\selectfont
  \fi
}
\makeatletter
\providecommand{\subtitle}[1]{\gdef\@subtitle{#1}}
\providecommand{\@subtitle}{}
\providecommand{\frontmattercontext}{}
\providecommand{\advisorname}{}
\providecommand{\advisorlabel}{Profesor guia:}
\providecommand{\frontmatterlocation}{}
\InputIfFileExists{includes/cover-config.tex}{}{}

\newcommand{\PrintTitle}{%
  {\CoverFont\fontsize{24pt}{28pt}\selectfont \@title\par}%
}
\newcommand{\PrintSubtitle}{%
  \begingroup
  \edef\temp{\detokenize{\@subtitle}}%
  \ifx\temp\empty\relax
    % sin subtitulo
  \else
    {\CoverFont\large \@subtitle\par}%
  \fi
  \endgroup
}
\newcommand{\PrintAuthor}{%
  \begingroup
  \renewcommand{\and}{\\[0.35em]}% separa autores en lineas, sin tabular
  {\CoverFont\Large\bfseries \@author\par}%
  \endgroup
}
\newcommand{\PrintDate}{%
  {\CoverFont\small \@date\par}%
}
\makeatother

\begin{titlepage}
\thispagestyle{empty}
\begin{center}
\CoverFont
\vspace*{10mm}

% Logo o imagen institucional
\includegraphics[width=0.35\textwidth]{assets/cover.png}\par
\vspace{12mm}

% Titulo y subtitulo
\PrintTitle
\vspace{6mm}
\PrintSubtitle

\vspace{22mm}
\begingroup
\edef\temp{\detokenize{\frontmattercontext}}%
\ifx\temp\empty\relax
  % sin contexto adicional
\else
  {\normalsize \CoverFont \frontmattercontext\par}%
\fi
\endgroup

\vspace{18mm}
\PrintAuthor

\vfill
\begingroup
\edef\temp{\detokenize{\frontmatterlocation}}%
\ifx\temp\empty\relax
  % sin ubicacion
\else
  {\small \CoverFont \frontmatterlocation\par}%
\fi
\endgroup
\PrintDate

\end{center}
\end{titlepage}

% Preliminares (numeros romanos) y estilo simple
\frontmatter
\pagestyle{scrheadings}

\setcounter{tocdepth}{2} % (o 1/3 segun prefieras)
\tableofcontents
\iffacsofigexists
  \cleardoublepage
  \listoffigures
\fi
\cleardoublepage


\mainmatter
\bookmarksetup{startatroot}

\chapter*{Presentación}\label{presentaciuxf3n}
\addcontentsline{toc}{chapter}{Presentación}

\markboth{Presentación}{Presentación}

\small\textbf{Palabras clave: } Inteligencia artificial; Trabajo
docente; Planificación escolar; Evaluación formativa; Inclusión
educativa. \normalsize

El presente documento presenta el taller \emph{``IA al servicio
pedagógico: herramientas prácticas para el trabajo docente''}, dirigido
a profesoras y profesores que participan en la Escuela de Verano 2026
del Museo de la Educación Gabriela Mistral. El taller se inscribe en una
perspectiva crítica y situada sobre el uso de la inteligencia artificial
en contextos escolares, entendida como un apoyo al trabajo profesional
docente y no como su reemplazo.

El propósito central de esta instancia es ofrecer un espacio práctico y
reflexivo para explorar cómo herramientas de IA pueden contribuir a
aliviar la sobrecarga laboral, mejorar la planificación, la evaluación y
la retroalimentación, así como adaptar materiales a distintos niveles y
necesidades de estudiantes. A partir de ejemplos concretos, actividades
guiadas y momentos de conversación pedagógica, se busca que las y los
participantes desarrollen criterios informados para decidir cuándo, cómo
y para qué utilizar estas tecnologías, resguardando siempre la autonomía
profesional y el sentido pedagógico de las decisiones en el aula. Este
diseño organiza los objetivos, contenidos, secuencias de trabajo y
orientaciones metodológicas del taller, de modo que pueda ser
implementado y adaptado en distintos contextos escolares.

\bookmarksetup{startatroot}

\chapter{¿Qué es la IA y cómo
funciona?}\label{quuxe9-es-la-ia-y-cuxf3mo-funciona}

\section{¿Qué entendemos por inteligencia artificial
hoy?}\label{quuxe9-entendemos-por-inteligencia-artificial-hoy}

En este taller entenderemos por \emph{inteligencia artificial (IA)} un
conjunto de técnicas informáticas que permiten a los computadores
realizar tareas que, si las hiciera una persona, consideraríamos
``inteligentes'': reconocer patrones, generar texto o imágenes, resumir
información, traducir, clasificar, entre otras. No se trata de una
``mente'' ni de un sujeto, sino de programas que aprenden a partir de
grandes volúmenes de datos.

En palabras sencillas: la IA es un conjunto de herramientas matemáticas
que \textbf{aprenden de muchos ejemplos} y luego usan lo aprendido para
hacer predicciones. Por ejemplo, a partir de miles de textos escolares
puede aprender a redactar explicaciones similares; a partir de muchas
preguntas de pruebas puede aprender a proponer nuevas preguntas
parecidas.

En la práctica cotidiana, cuando hablamos de IA hoy casi siempre nos
referimos a sistemas basados en \textbf{aprendizaje automático}
(\emph{machine learning}) y, en particular, a \textbf{modelos de
lenguaje} y otros modelos generativos que se entrenan con enormes
cantidades de textos, imágenes, audio o video. Estos modelos aprenden a
detectar regularidades estadísticas y a ``predecir'' qué palabra, imagen
o respuesta es más probable según el contexto.

En el contexto escolar chileno esto se traduce, por ejemplo, en
herramientas que pueden:

\begin{itemize}
\tightlist
\item
  sugerir actividades para una clase de 6º básico en Lenguaje sobre
  comprensión lectora;
\item
  proponer ideas de preguntas para una prueba de Historia en 2º medio;
\item
  redactar un borrador de correo para apoderadas y apoderados;
\item
  simplificar un texto de Ciencias Naturales para un curso con niveles
  de lectura muy distintos.
\end{itemize}

Es importante distinguir entre:

\begin{itemize}
\tightlist
\item
  \textbf{IA general}: la idea de una inteligencia similar o superior a
  la humana, capaz de hacer cualquier tarea intelectual. Hoy \textbf{no
  existe} este tipo de IA, aunque aparezca en películas o noticias.
\item
  \textbf{IA específica o aplicada}: programas que resuelven tareas
  concretas (por ejemplo, sugerir actividades para una clase, redactar
  un correo, resumir un texto escolar).
\item
  \textbf{IA generativa}: sistemas que crean nuevos contenidos (textos,
  imágenes, código, etc.) combinando patrones aprendidos a partir de
  muchos ejemplos.
\end{itemize}

En este taller trabajaremos principalmente con \textbf{IA generativa}
para texto, entendiéndola siempre como una \textbf{herramienta al
servicio del juicio pedagógico} de las y los docentes, y no como un
sustituto de su trabajo profesional ni de las decisiones que se toman en
la escuela.

\textbf{Idea clave para el aula:} la IA no es una persona ni una mente,
sino una herramienta estadística que aprende de muchos ejemplos y que
puede apoyar tareas concretas del trabajo docente, siempre bajo el
criterio profesional de las y los profesores.

\begin{center}\rule{0.5\linewidth}{0.5pt}\end{center}

\section{Modelos de lenguaje: cómo funcionan ``a grandes
rasgos''}\label{modelos-de-lenguaje-cuxf3mo-funcionan-a-grandes-rasgos}

En este taller usaremos principalmente \textbf{modelos de lenguaje}:
programas que, a partir de mucho texto de entrenamiento, aprenden a
continuar frases y responder preguntas de manera coherente. Una forma
sencilla de entenderlo es pensar en un juego de completar frases: el
modelo intenta adivinar cuál es la siguiente palabra más probable según
lo que ya se escribió.

Para trabajar pedagógicamente con estas herramientas \textbf{no es
necesario saber programar} ni entender todas las fórmulas internas. Pero
sí ayuda tener una idea general de cómo aprenden y qué limitaciones
tienen, para poder usarlas de manera crítica.

\subsection{Redes neuronales
artificiales}\label{redes-neuronales-artificiales}

Los modelos de lenguaje actuales se basan en \textbf{redes neuronales
artificiales}. Una red neuronal es un modelo matemático formado por
capas de ``neuronas'' conectadas entre sí. Cada neurona recibe números
como entrada, los combina y produce una salida. Al entrenar la red con
muchos ejemplos, va ajustando sus conexiones internas para cometer cada
vez menos errores. Las redes neuronales artificiales se inspiran muy
superficialmente en el cerebro, pero son ante todo modelos matemáticos
que ajustan números. La IA no ``piensa'' ni ``siente'': calcula, predice
y repite patrones que ha aprendido de los datos.

\begin{figure}[H]

\caption{Ejemplo de relación entre una Neurona biologica y un
Perceptrón: la neurona artificial básica.}

{\centering \pandocbounded{\includegraphics[keepaspectratio]{assets/ann1.png}}

}

\end{figure}%

Podemos pensarlo con una metáfora escolar:\\
cuando una profesora corrige muchas pruebas de su curso, empieza a
\textbf{reconocer patrones} (respuestas típicas, errores frecuentes,
formas de escribir). Con esa experiencia, puede anticipar qué
dificultades aparecerán en la siguiente evaluación. La red neuronal hace
algo similar, pero con números y a una escala muchísimo mayor.

En términos muy simples, el proceso de entrenamiento funciona así:

\begin{enumerate}
\def\labelenumi{\arabic{enumi}.}
\tightlist
\item
  Se convierte el texto en números (llamados \emph{tokens}).
\item
  La red neuronal recibe esos números y genera una predicción (por
  ejemplo, la siguiente palabra).
\item
  Se compara la predicción con la respuesta correcta.
\item
  El modelo ajusta internamente sus parámetros para equivocarse menos la
  próxima vez.
\item
  Se repite el proceso millones de veces con enormes conjuntos de datos.
\end{enumerate}

Podemos visualizar una red neuronal básica así:

\begin{figure}[H]

\caption{Ejemplo de red neuronal artificial con una capa de entrada, una
oculta y una de salida. Fuente: Cburnett, \emph{Artificial neural
network}, Wikimedia Commons (CC BY-SA 3.0).}

{\centering \pandocbounded{\includegraphics[keepaspectratio]{assets/ann.png}}

}

\end{figure}%

En los modelos actuales, en vez de unas pocas capas hay \textbf{decenas
o cientos de capas}, con millones o miles de millones de parámetros. Por
eso se habla de \emph{modelos grandes}.

\subsection{Del perceptrón al
Transformer}\label{del-perceptruxf3n-al-transformer}

Los modelos de lenguaje modernos como ChatGPT (Generative Pre-trained
Transformer) utilizan una arquitectura llamada \textbf{Transformer},
propuesta en 2017 (\citeproc{ref-vaswani2017attention}{Vaswani et~al.,
2017}), que cambió radicalmente la forma de trabajar con texto. La idea
central es el mecanismo de \textbf{auto-atención}
(\emph{self-attention}), que permite que el modelo mire todas las
palabras de una frase a la vez y decida a cuáles prestar más atención
para entender el contexto.

Una forma cotidiana de imaginarlo es pensar en una profesora leyendo una
respuesta larga de un estudiante: no se fija solo en la última frase,
sino que revisa todo el párrafo, subraya mentalmente las ideas más
importantes y, a partir de eso, evalúa o redacta una retroalimentación.
El Transformer hace algo análogo, pero a nivel numérico y automatizado.

A grandes rasgos, un Transformer:

\begin{enumerate}
\def\labelenumi{\arabic{enumi}.}
\tightlist
\item
  \textbf{Recibe una secuencia de palabras} (convertidas en vectores
  numéricos).
\item
  \textbf{Calcula la atención}: para cada palabra, pondera cuánto se
  relaciona con las demás palabras de la secuencia.
\item
  \textbf{Transforma la representación interna del texto} pasando por
  varias capas que combinan atención y redes neuronales.
\item
  \textbf{Predice la siguiente palabra} (o el siguiente fragmento de
  texto) eligiendo la opción más probable según lo aprendido.
\end{enumerate}

La arquitectura típica de un Transformer puede representarse así:

\begin{figure}[H]

\caption{Arquitectura general de un modelo tipo Transformer con bloques
de codificador y decodificador.\\
Fuente: Yuening Jia, \emph{The Transformer - model architecture},
Wikimedia Commons (CC BY-SA 3.0).}

{\centering \pandocbounded{\includegraphics[keepaspectratio]{assets/transformer.png}}

}

\end{figure}%

Para el trabajo docente \textbf{no necesitamos dominar este diagrama}.
Lo importante es retener tres ideas clave:

\begin{itemize}
\tightlist
\item
  El modelo \textbf{no ``piensa'' como una persona}, sino que calcula
  probabilidades a partir de patrones aprendidos en los datos.
\item
  El modelo \textbf{no sabe qué es verdadero o falso} por sí mismo: ha
  visto textos, pero no la realidad; por eso puede equivocarse o
  ``inventar'' información.
\item
  El modelo funciona como un \textbf{completador de texto muy
  sofisticado}, que podemos guiar mediante instrucciones claras
  (\emph{prompts}) para que su producción sea más útil en el contexto
  escolar.
\end{itemize}

\textbf{Idea clave para el aula:} un modelo de lenguaje es, en esencia,
un completador de texto entrenado con muchísimos ejemplos. No entiende
el mundo como una persona, pero puede ser muy útil si sabemos darle
buenas instrucciones y revisar críticamente sus respuestas.

\subsection{Algunas precisiones importantes sobre cómo
aprende}\label{algunas-precisiones-importantes-sobre-cuxf3mo-aprende}

Para usar estas herramientas de manera crítica en la escuela, no
necesitamos saber programarlas, pero sí comprender tres ideas clave
sobre su funcionamiento:

\begin{itemize}
\item
  \textbf{Aprenden de grandes bases de datos}\\
  Los modelos se entrenan con enormes cantidades de textos e imágenes
  disponibles en internet, libros digitales, repositorios, etc. A partir
  de esos datos aprenden patrones: qué palabras suelen ir juntas, cómo
  se estructura un texto explicativo, cómo se formula una pregunta de
  prueba, etc.
\item
  \textbf{No aprenden ``en vivo'' de cada conversación}\\
  Cuando usamos un modelo de lenguaje en el aula, este no ``recuerda''
  de manera estable lo que le dijimos en sesiones anteriores ni
  ``actualiza'' su conocimiento curricular en tiempo real (a menos que
  el sistema esté diseñado específicamente para eso). Esto significa que
  puede quedarse con información desactualizada o incompleta.
\item
  \textbf{Pueden ``inventar'' información (alucinaciones)}\\
  Como el modelo funciona prediciendo la palabra o respuesta más
  probable, a veces completa con datos que suenan convincentes, pero que
  no son verdaderos o que no están bien fundamentados. Por eso no es
  recomendable usar la IA como única fuente de información factual o
  curricular, sino más bien como apoyo para redactar, proponer ideas o
  reformular materiales.
\end{itemize}

\textbf{Idea clave:} no necesitamos conocer las ecuaciones internas del
modelo, pero sí saber que aprende de datos pasados, que puede
equivocarse con seguridad y que, por lo mismo, requiere siempre revisión
crítica por parte de las y los docentes.

\section{Qué puede hacer y qué no puede hacer en
educación}\label{quuxe9-puede-hacer-y-quuxe9-no-puede-hacer-en-educaciuxf3n}

En el sistema escolar chileno, con cursos numerosos, diversidad de
trayectorias, presencia de PIE y alta carga laboral docente, la IA puede
convertirse en un apoyo concreto para ciertas tareas, siempre que se use
con criterio y resguardos.

\subsection{¿Qué sí puede hacer?}\label{quuxe9-suxed-puede-hacer}

En el contexto escolar, modelos de lenguaje e IA generativa pueden
apoyar, entre otras tareas:

\begin{itemize}
\item
  \textbf{Planificación y preparación de clases}\\
  Sugerir objetivos, actividades, secuencias y preguntas guía para una
  unidad de Historia, Matemática, Lenguaje, Ciencias, etc., alineadas
  con las Bases Curriculares que la/el docente ya definió.
\item
  \textbf{Elaboración de materiales}\\
  Generar borradores de guías, ejercicios, ejemplos contextualizados al
  territorio (por ejemplo, situaciones cercanas al barrio o comuna),
  casos para debatir en Formación Ciudadana, etc.
\item
  \textbf{Evaluación y retroalimentación}\\
  Proponer ítems y criterios de evaluación, sugerir comentarios de
  retroalimentación que la/el docente revisa, adapta y decide si usar
  (por ejemplo, para un ensayo argumentativo de 2º medio).
\item
  \textbf{Adaptación de materiales}\\
  Simplificar un texto de Ciencias Naturales para 6º básico, cambiar el
  nivel de complejidad de un problema de Matemática, generar versiones
  alternativas de una misma actividad para distintos cursos.
\item
  \textbf{Organización del trabajo cotidiano}\\
  Redactar borradores de comunicaciones a familias, sistematizar
  acuerdos de reuniones de departamento, sintetizar documentos extensos
  (orientaciones, decretos, circulares) en puntos clave.
\end{itemize}

En todos los casos, la IA actúa como \textbf{asistente de borradores}:
entrega propuestas iniciales que la/el docente revisa, corrige y
contextualiza según su criterio profesional y el proyecto educativo del
establecimiento.

\subsection{¿Qué no puede hacer (ni debería
hacer)?}\label{quuxe9-no-puede-hacer-ni-deberuxeda-hacer}

Hay tareas que, por razones técnicas, éticas o pedagógicas, la IA
\textbf{no debiera reemplazar}:

\begin{itemize}
\item
  \textbf{Conocer a las y los estudiantes}\\
  Su historia, contexto familiar, trayectoria escolar, emociones y
  vínculos se construyen en la relación cotidiana, no en un modelo
  estadístico.
\item
  \textbf{Tomar decisiones evaluativas de alto impacto}\\
  Aprobación o reprobación, repitencia, derivaciones a apoyos
  especializados o sanciones disciplinarias deben ser decisiones
  humanas, sustentadas en información múltiple y diálogo profesional.
\item
  \textbf{Definir objetivos de formación}\\
  Los sentidos educativos de largo plazo, el proyecto educativo
  institucional y las prioridades pedagógicas se construyen en la
  comunidad educativa, no se delegan a un algoritmo.
\item
  \textbf{Sostener vínculos afectivos y pedagógicos}\\
  El acompañamiento emocional, el clima de aula, la construcción de
  confianza y respeto requieren presencia humana, cuidado y
  responsabilidad, no pueden automatizarse.
\item
  \textbf{Garantizar veracidad y ausencia de sesgos}\\
  Los modelos pueden inventar datos, equivocarse, reproducir
  estereotipos de género, clase, origen o etnia presentes en los textos
  con que fueron entrenados.
\end{itemize}

Por estas razones, en el taller asumiremos que la IA es una
\textbf{herramienta de apoyo} al trabajo pedagógico que requiere
siempre:

\begin{itemize}
\tightlist
\item
  Revisión crítica por parte de las y los docentes.
\item
  Adaptación al contexto de cada escuela, liceo o liceo técnico
  profesional.
\item
  Alineamiento con el proyecto educativo, las Bases Curriculares y la
  normativa vigente.
\item
  Discusión colectiva en los equipos docentes y directivos sobre sus
  usos y límites.
\end{itemize}

\textbf{Idea clave para el aula:} la IA puede ayudar a ahorrar tiempo y
a ampliar ideas, pero no puede conocer a tus estudiantes, ni decidir por
ti qué, cómo y para qué enseñar. Las decisiones pedagógicas y éticas
siguen siendo responsabilidad de las y los profesionales de la
educación.

\bookmarksetup{startatroot}

\chapter{Claves de prompt engineering
docente}\label{claves-de-prompt-engineering-docente}

\section{Qué es un ``prompt'' y por qué importa en el trabajo
docente}\label{quuxe9-es-un-prompt-y-por-quuxe9-importa-en-el-trabajo-docente}

En términos simples, un \emph{prompt} es el mensaje o instrucción que le
damos a la inteligencia artificial para que haga algo por nosotras/os:
proponer ideas de clase, redactar una rúbrica, adaptar un texto,
escribir un comunicado a apoderadas/os, crear imagenes, videos, ZZZetc.
Es la consigna que guía la respuesta del modelo, muy parecido a cuando
redactamos una buena instrucción de trabajo para nuestro curso.

En el trabajo docente, la calidad del \emph{prompt} es clave porque:

\begin{itemize}
\tightlist
\item
  define qué tarea realizará la IA (por ejemplo, ``sugerir actividades
  para\ldots{}'', ``proponer preguntas de evaluación sobre\ldots{}'');
\item
  entrega el contexto pedagógico (curso, asignatura, objetivos de
  aprendizaje, características del establecimiento y del grupo);
\item
  indica cómo queremos recibir la respuesta (tabla, lista, texto breve,
  tono formal o cercano, extensión aproximada);
\item
  establece límites y resguardos (no inventar datos, no usar lenguaje
  técnico, no tomar decisiones pedagógicas por la/el docente).
\end{itemize}

Por ejemplo, no es lo mismo escribir:

\begin{quote}
``Ayúdame con una clase de fracciones'',
\end{quote}

que decir:

\begin{quote}
``Actúa como profesora de Matemática de 5º básico en una escuela pública
con cursos numerosos. Propón tres actividades de 20 minutos para
introducir fracciones equivalentes, usando solo pizarra y cuadernos, y
explícame en una tabla objetivo, actividad y cierre de cada una''.
\end{quote}

En el primer caso, la respuesta será genérica y poco útil. En el
segundo, la IA tiene una tarea clara, contexto, formato y límites, por
lo que puede apoyar de manera más ajustada el trabajo docente.

\textbf{Idea fuerza:} si el \emph{prompt} es vago, la respuesta también
lo será; un buen \emph{prompt} se parece a una buena consigna de
trabajo: clara, contextualizada y con criterios explícitos.

\section{Estructura de un buen prompt
pedagógico}\label{estructura-de-un-buen-prompt-pedaguxf3gico}

En esta sección se propone una ``columna vertebral'' para redactar
prompts útiles en educación. No se trata de reglas rígidas, sino de una
guía que cada docente puede adaptar a su estilo y a la realidad de su
escuela.

\subsection{Los 5 componentes
básicos}\label{los-5-componentes-buxe1sicos}

\textbf{1) Rol de la IA: ¿desde qué lugar quiero que responda?}

Aquí le decimos al modelo ``qué tipo de profesional'' queremos que
simule.\\
Ejemplos:

\begin{itemize}
\tightlist
\item
  ``Actúa como profesora de Historia de enseñanza media en un liceo
  público\ldots{}''
\item
  ``Actúa como asesor pedagógico con experiencia en evaluación
  formativa\ldots{}''
\item
  ``Actúa como especialista en educación inclusiva y diseño universal
  para el aprendizaje (DUA)\ldots{}''.
\end{itemize}

\textbf{Uso típico:} pedir sugerencias de actividades, rúbricas o
adaptaciones considerando miradas pedagógicas específicas (inclusión,
evaluación, orientación, etc.).

\textbf{2) Objetivo / tarea principal: ¿qué necesito que haga?}

Es la acción central que esperamos de la IA, en una frase concreta.\\
Ejemplos:

\begin{itemize}
\tightlist
\item
  ``Tu objetivo es proponer tres actividades breves para\ldots{}''
\item
  ``Tu tarea es elaborar una rúbrica de evaluación para\ldots{}''
\item
  ``Tu misión es sintetizar el siguiente texto para que lo pueda
  comprender un curso de 6º básico\ldots{}''.
\end{itemize}

\textbf{Uso típico:} planificación de clases o unidades, diseño de
instrumentos, redacción de comunicaciones, etc.

\textbf{3) Contexto pedagógico: ¿qué información de mi realidad escolar
debe considerar?}

Aquí entregamos los datos clave de la situación:

\begin{itemize}
\tightlist
\item
  nivel y curso (5º básico, 2º medio técnico-profesional, etc.);
\item
  asignatura y contenido específico (por ejemplo, ``geometría: círculo y
  circunferencia'');
\item
  características del establecimiento (público, subvencionado, rural,
  urbano, con PIE, etc.);
\item
  características del grupo (curso numeroso, heterogéneo, presencia de
  NEE, alta sobrecarga de tareas, etc.).
\end{itemize}

Ejemplo: ``Curso: 1º medio humanista-científico, 42 estudiantes, varios
con apoyo PIE en lenguaje escrito''.

\textbf{4) Formato de salida esperado: ¿cómo quiero recibir la
respuesta?}

Definimos la forma del producto final para que sea más fácil de usar:

\begin{itemize}
\tightlist
\item
  ``Entrega la respuesta en una tabla con columnas: clase, objetivo,
  actividad principal y evaluación''.
\item
  ``Escribe una lista numerada de máximo 5 puntos''.
\item
  ``Redacta un texto breve (máx. 200 palabras) pensado para estudiantes
  de 7º básico''.
\item
  ``Proporciona ejemplos de preguntas de alternativa y de desarrollo
  corto''.
\end{itemize}

\textbf{Uso típico:} generar planificaciones, guías, rúbricas,
comunicados o resúmenes listos para pegar y ajustar.

\textbf{5) Criterios y restricciones (guardarraíles): ¿qué límites
quiero poner?}

Sirven para cuidar el sentido pedagógico, la ética y la viabilidad real
en la escuela:

\begin{itemize}
\tightlist
\item
  ``Usa lenguaje sencillo, sin tecnicismos''.
\item
  ``No inventes datos ni fuentes; si no sabes, dilo explícitamente''.
\item
  ``No entregues actividades que requieran recursos costosos ni conexión
  a internet''.
\item
  ``Respeta lenguaje inclusivo y enfoque no sexista''.
\end{itemize}

Ejemplo: al pedir una propuesta de evaluación, aclarar que la IA no debe
decidir notas ni aprobar/reprobar estudiantes.

\textbf{Idea fuerza:} un buen prompt pedagógico siempre responde a cinco
preguntas: ¿qué rol?, ¿qué tarea?, ¿en qué contexto?, ¿con qué formato?
y ¿con qué límites y resguardos?

\section{Técnicas de prompting para
docentes}\label{tuxe9cnicas-de-prompting-para-docentes}

A continuación se presentan algunas técnicas sencillas, ordenadas como
pasos posibles, para mejorar la calidad de las respuestas de la IA en el
trabajo pedagógico cotidiano.

{1.Sé específica/o y precisa/o}.

Evita pedir ``ideas para una clase'' de forma genérica. Indica curso,
asignatura, contenido, objetivo y tiempo disponible. Mientras más claro
sea el encargo, más útil será la respuesta.

\textbf{Ejemplo de prompt:}

\begin{quote}
``Actúa como profesora de Lenguaje en 7º básico de una escuela pública
urbana. Tu objetivo es proponer dos actividades de 30 minutos para
trabajar comprensión lectora de crónicas periodísticas. Contexto: curso
numeroso (40 estudiantes), varios con apoyo PIE. Formato de salida:
tabla con columnas objetivo, actividad, recursos y sugerencias de
cierre. Criterios: lenguaje sencillo, actividades sin uso de
computadores.''
\end{quote}

{\textbf{2.} Usa restricciones claras (guardarraíles)}.

Señala lo que la IA no debe hacer o aquello que quieres acotar:
extensión, tono, tipo de recursos, nivel de lenguaje, etc.

\textbf{Ejemplo de prompt:}

\begin{quote}
``Proponte tres ideas de actividad para trabajar porcentajes en 8º
básico. No uses jerga técnica, máximo 200 palabras en total, y plantea
actividades que se puedan realizar solo con cuadernos, lápiz y
calculadora simple.''
\end{quote}

{\textbf{3.} Itera y refina: piensa el diálogo como un proceso}.

No esperes que la primera respuesta sea perfecta. Puedes pedir ajustes,
correcciones o versiones mejoradas.

\textbf{Ejemplos de seguimientos:}

\begin{itemize}
\tightlist
\item
  ``Haz la propuesta más breve y con lenguaje más simple para 5º
  básico''.
\item
  ``Adapta esta misma actividad para un curso con varios estudiantes con
  NEE en lectura''.
\item
  ``Agrega una columna con posibles errores frecuentes de los
  estudiantes''.
\end{itemize}

\textbf{4.} Encadena prompts (\emph{chain prompting})

Para tareas complejas, es mejor dividir el trabajo en pasos: primero
objetivos, luego actividades, después evaluación. También puedes pedir
que la propia IA sugiera el siguiente paso.

\textbf{Ejemplo encadenado:}

\begin{quote}
``Ayúdame a definir tres objetivos de aprendizaje para una unidad sobre
migración contemporánea en 2º medio, alineados con las Bases
Curriculares de Historia.''\\
``Con esos objetivos, propón una secuencia de 5 clases con actividades
principales y tiempos estimados.''\\
``Ahora sugiere posibles instrumentos de evaluación formativa para esta
secuencia, indicando qué recoge cada uno.''
\end{quote}

{\textbf{5.} Pide pasos, criterios o fuentes para revisar la calidad}.

Solicita que la IA muestre ``cómo llegó'' a lo que propone, qué
supuestos usa o qué referencias generales considera. Esto ayuda a
detectar errores y a tomar decisiones informadas.

\textbf{Ejemplo de prompt:}

\begin{quote}
``Construye una rúbrica simple para evaluar un informe escrito en 8º
básico sobre contaminación ambiental. Luego explica en 4 pasos cómo
definiste los criterios y niveles de logro, de manera que yo pueda
revisarlos críticamente.''
\end{quote}

{\textbf{6.} Modela el tipo de reflexión que quieres que haga}.

Puedes pedirle explícitamente que revise críticamente su propia
respuesta antes de entregarla.

\textbf{Ejemplo de prompt:}

\begin{quote}
``Propon tres actividades para trabajar el acoso escolar en Consejo de
Curso de 6º básico. Antes de entregarlas, indica dos posibles problemas
o riesgos de tu propuesta (por ejemplo, revictimización, exposición de
estudiantes) y luego ofrece una versión mejorada que intente
abordarlos.''
\end{quote}

\textbf{Idea fuerza:} las técnicas de prompting no reemplazan el
criterio profesional docente, pero sí ayudan a que la IA se parezca más
a una colega que apoya: tú sigues tomando las decisiones; la herramienta
solo propone y tú filtras, adaptas y mejoras.

\section{Plantillas de prompts reutilizables para
docencia}\label{plantillas-de-prompts-reutilizables-para-docencia}

A continuación se proponen algunas plantillas listas para copiar, pegar
y adaptar según las necesidades de cada docente. La idea es que
funcionen como ``moldes'' rápidos para planificación, evaluación,
adaptación de materiales o comunicaciones.

\subsection{Plantillas base de prompt
docente}\label{plantillas-base-de-prompt-docente}

\begin{Shaded}
\begin{Highlighting}[]
\NormalTok{actúa como }\CommentTok{[}\OtherTok{rol o perfil que necesito}\CommentTok{]}\NormalTok{ en una escuela }\CommentTok{[}\OtherTok{tipo de establecimiento: pública, subvencionada, tp, rural, etc.}\CommentTok{]}\NormalTok{ de chile.}

\NormalTok{tu objetivo es }\CommentTok{[}\OtherTok{tarea principal que quiero lograr}\CommentTok{]}\NormalTok{ relacionada con }\CommentTok{[}\OtherTok{breve descripción del tema o contenido}\CommentTok{]}\NormalTok{.}

\NormalTok{contexto:}

\NormalTok{curso y nivel: }\CommentTok{[}\OtherTok{curso, nivel}\CommentTok{]}\NormalTok{.}

\NormalTok{asignatura y contenido: }\CommentTok{[}\OtherTok{asignatura, contenido específico}\CommentTok{]}\NormalTok{.}

\NormalTok{características del grupo: }\CommentTok{[}\OtherTok{número de estudiantes, presencia de pie, diversidad de niveles, etc.}\CommentTok{]}\NormalTok{.}

\NormalTok{recursos disponibles: }\CommentTok{[}\OtherTok{pizarra, cuadernos, proyector, laboratorio, etc.}\CommentTok{]}\NormalTok{.}

\NormalTok{instrucciones:}

\CommentTok{[}\OtherTok{primera instrucción específica, por ejemplo “propón x actividades de y minutos”}\CommentTok{]}\NormalTok{.}

\CommentTok{[}\OtherTok{segunda instrucción específica, por ejemplo “incluye una idea de evaluación formativa”}\CommentTok{]}\NormalTok{.}

\CommentTok{[}\OtherTok{tercera instrucción específica, por ejemplo “sugiere ajustes para estudiantes que necesiten más apoyo”}\CommentTok{]}\NormalTok{.}

\NormalTok{formato de salida:}

\CommentTok{[}\OtherTok{tipo de formato: tabla, lista numerada, texto breve para estudiantes, etc.}\CommentTok{]}\NormalTok{.}

\NormalTok{criterios y restricciones:}

\CommentTok{[}\OtherTok{tono y nivel de lenguaje, por ejemplo “usa lenguaje sencillo y ejemplos cercanos a la vida cotidiana de las y los estudiantes”}\CommentTok{]}\NormalTok{.}

\CommentTok{[}\OtherTok{límites de recursos, por ejemplo “no propongas actividades que requieran celulares o internet”}\CommentTok{]}\NormalTok{.}

\CommentTok{[}\OtherTok{resguardos éticos y pedagógicos, por ejemplo “no tomes decisiones de notas ni apruebes/repruebes estudiantes”}\CommentTok{]}\NormalTok{.}

\NormalTok{Antes de responder, si falta información importante, hazme hasta 3 preguntas breves para aclarar el contexto.}
\end{Highlighting}
\end{Shaded}

\subsection{Diseñador de prompts pedagógicos para
docentes}\label{diseuxf1ador-de-prompts-pedaguxf3gicos-para-docentes}

\begin{Shaded}
\begin{Highlighting}[]
\NormalTok{Actúa como especialista en educación y en prompt engineering para docentes.}

\NormalTok{Tu tarea es ayudarme a DISEÑAR un prompt claro y efectivo que luego pueda usar con otra IA.}

\NormalTok{Contexto:}
\SpecialStringTok{{-} }\NormalTok{Objetivo general del prompt que necesito: }\CommentTok{[}\OtherTok{por ejemplo, “planificar una unidad de Historia para 8º básico sobre dictadura y memoria”}\CommentTok{]}\NormalTok{.}
\SpecialStringTok{{-} }\NormalTok{Tipo de producto que quiero obtener con ese futuro prompt: }\CommentTok{[}\OtherTok{por ejemplo, “una tabla con clases, objetivos, actividades y evaluaciones”}\CommentTok{]}\NormalTok{.}
\SpecialStringTok{{-} }\NormalTok{Público destinatario de los resultados: }\CommentTok{[}\OtherTok{por ejemplo, “docentes de escuela pública con alta carga laboral”}\CommentTok{]}\NormalTok{.}

\NormalTok{Instrucciones:}
\SpecialStringTok{1. }\NormalTok{Formula un único prompt completo y bien redactado que yo pueda copiar y usar tal cual.}
\SpecialStringTok{2. }\NormalTok{Asegúrate de que el prompt incluya: rol de la IA, objetivo, contexto pedagógico, formato de salida, criterios y restricciones.}
\SpecialStringTok{3. }\NormalTok{Al final, sugiere brevemente cómo podría ajustar ese prompt (por ejemplo, para otros niveles o asignaturas).}

\NormalTok{Formato de salida:}
\SpecialStringTok{{-} }\NormalTok{Primero, escribe el prompt final entre comillas.}
\SpecialStringTok{{-} }\NormalTok{Luego, en 3–4 viñetas, entrega sugerencias de ajustes posibles.}

\NormalTok{No expliques qué es un prompt ni teorices: concéntrate en darme un buen prompt listo para usar.}
\end{Highlighting}
\end{Shaded}

\textbf{Idea fuerza:} contar con plantillas reutilizables ahorra tiempo
y energía; cada docente puede ir guardando sus prompts favoritos,
mejorarlos con la experiencia y compartirlos con sus colegas como parte
de una cultura de colaboración en torno al uso pedagógico de la IA.

\bookmarksetup{startatroot}

\chapter{Herramientas prácticas para el trabajo
docente}\label{herramientas-pruxe1cticas-para-el-trabajo-docente}

\section{IA para planificación de clases y
unidades}\label{ia-para-planificaciuxf3n-de-clases-y-unidades}

La planificación es una de las tareas que más tiempo demanda en el
trabajo docente. La inteligencia artificial puede funcionar aquí como un
\textbf{asistente de borradores}: ayuda a proponer ideas, ordenar
secuencias y sugerir actividades, pero la decisión final sobre qué, cómo
y cuándo enseñar sigue siendo siempre pedagógica y profesional, basada
en las Bases Curriculares y en el conocimiento que cada docente tiene de
su curso y de sus estudiantes.

Usada de manera crítica y situada, la IA puede ahorrar tiempo en la fase
inicial de la planificación, de modo que las y los profesores puedan
dedicar más energía a la reflexión pedagógica, al trabajo colaborativo
con colegas y a la adaptación fina de las propuestas a su realidad
escolar concreta (cursos numerosos, presencia de PIE, uso de textos
oficiales, etc.).

\textbf{Usos posibles en planificación:}

\begin{itemize}
\tightlist
\item
  \textbf{Diseño de unidades y secuencias:}

  \begin{itemize}
  \tightlist
  \item
    Proponer borradores de secuencias de clases para una unidad
    específica.
  \item
    Ordenar actividades ya existentes en una secuencia más coherente
    (inicio--desarrollo--cierre).
  \item
    Sugerir tiempos aproximados para cada momento de la clase.
  \end{itemize}
\item
  \textbf{Objetivos y actividades:}

  \begin{itemize}
  \tightlist
  \item
    Sugerir objetivos de aprendizaje y actividades de inicio, desarrollo
    y cierre, alineadas con OA u objetivos priorizados.
  \item
    Ayudar a alinear actividades con objetivos de aprendizaje definidos
    previamente por la/el docente.
  \item
    Proponer preguntas guía, problemas o situaciones iniciales para
    activar conocimientos previos.
  \end{itemize}
\item
  \textbf{Ajustes según realidad del curso:}

  \begin{itemize}
  \tightlist
  \item
    Ofrecer variantes de una misma clase según el tiempo disponible
    (bloques dobles, horarios reducidos, etc.).
  \item
    Sugerir adaptaciones iniciales para cursos con alta diversidad de
    niveles de logro o presencia de PIE.
  \end{itemize}
\end{itemize}

\textbf{Ejemplo de prompt para planificación de una unidad:}

\begin{Shaded}
\begin{Highlighting}[]
\NormalTok{Actúa como profesor/a de }\CommentTok{[}\OtherTok{ASIGNATURA}\CommentTok{]}\NormalTok{ en }\CommentTok{[}\OtherTok{NIVEL}\CommentTok{]}\NormalTok{ en una escuela }\CommentTok{[}\OtherTok{TIPO DE ESTABLECIMIENTO: pública, subvencionada, particular}\CommentTok{]}\NormalTok{ de Chile.}

\NormalTok{Tu objetivo es ayudarme a elaborar un borrador de planificación para UNA unidad didáctica sobre }\CommentTok{[}\OtherTok{TEMA O CONTENIDO}\CommentTok{]}\NormalTok{.}

\NormalTok{Contexto:}
\SpecialStringTok{{-} }\NormalTok{Curso y tamaño aproximado: }\CommentTok{[}\OtherTok{por ejemplo, 8º básico, 40 estudiantes}\CommentTok{]}\NormalTok{.}
\SpecialStringTok{{-} }\NormalTok{Marco curricular: Bases Curriculares chilenas para }\CommentTok{[}\OtherTok{ASIGNATURA}\CommentTok{]}\NormalTok{; OA a trabajar: }\CommentTok{[}\OtherTok{pegar aquí si es posible}\CommentTok{]}\NormalTok{.}
\SpecialStringTok{{-} }\NormalTok{Tiempo disponible: }\CommentTok{[}\OtherTok{número aproximado de clases u horas pedagógicas}\CommentTok{]}\NormalTok{.}
\SpecialStringTok{{-} }\NormalTok{Características del grupo: }\CommentTok{[}\OtherTok{curso diverso, presencia de PIE, diferencias importantes en niveles de logro, etc.}\CommentTok{]}\NormalTok{.}
\SpecialStringTok{{-} }\NormalTok{Recursos habituales: }\CommentTok{[}\OtherTok{por ejemplo, pizarra, cuadernos, proyector, textos escolares}\CommentTok{]}\NormalTok{.}

\NormalTok{Instrucciones:}
\SpecialStringTok{{-} }\NormalTok{Propón una secuencia de }\CommentTok{[}\OtherTok{X}\CommentTok{]}\NormalTok{ clases para esta unidad.}
\SpecialStringTok{{-} }\NormalTok{Para cada clase, indica: objetivo específico, actividad principal de aprendizaje, recurso clave a utilizar y forma sencilla de evaluación (por ejemplo, pregunta de salida, ejercicio breve).}
\SpecialStringTok{{-} }\NormalTok{Incluye al menos una sugerencia de ajuste o variación para estudiantes que requieran más apoyo y otra para quienes puedan profundizar.}
\SpecialStringTok{{-} }\NormalTok{Procura que las actividades sean viables en un contexto escolar chileno con recursos limitados.}

\NormalTok{Formato de salida:}
\SpecialStringTok{{-} }\NormalTok{Tabla con columnas: Clase – Objetivo específico – Actividad principal – Recurso – Evaluación sugerida – Ajustes/variantes.}

\NormalTok{Criterios y restricciones:}
\SpecialStringTok{{-} }\NormalTok{Usa lenguaje claro y concreto, adecuado al contexto escolar chileno.}
\SpecialStringTok{{-} }\NormalTok{No inventes contenidos fuera de las Bases Curriculares; organiza y ejemplifica a partir de los OA indicados.}
\SpecialStringTok{{-} }\NormalTok{No propongas actividades que requieran recursos tecnológicos que no haya mencionado.}
\SpecialStringTok{{-} }\NormalTok{Si te falta información importante para planificar mejor, hazme hasta 3 preguntas breves antes de responder.}
\end{Highlighting}
\end{Shaded}

\begin{tcolorbox}[enhanced jigsaw, colframe=quarto-callout-note-color-frame, opacityback=0, coltitle=black, colbacktitle=quarto-callout-note-color!10!white, titlerule=0mm, left=2mm, breakable, leftrule=.75mm, bottomtitle=1mm, opacitybacktitle=0.6, toptitle=1mm, bottomrule=.15mm, toprule=.15mm, arc=.35mm, rightrule=.15mm, title=\textcolor{quarto-callout-note-color}{\faInfo}\hspace{0.5em}{Nota}, colback=white]

\textbf{Idea fuerza:} mientras más concreto y contextualizado sea el
prompt, más útil será el borrador de planificación. La IA puede ahorrar
tiempo en la fase de diseño, pero las decisiones sobre qué se enseña y
cómo se enseña siguen siendo siempre de la/el docente.

\end{tcolorbox}

\section{IA para generación y adaptación de
materiales}\label{ia-para-generaciuxf3n-y-adaptaciuxf3n-de-materiales}

Otra tarea muy demandante en el trabajo docente es la elaboración,
revisión y adaptación de materiales: guías, ejercicios, textos,
ejemplos, preguntas, lecturas, imágenes, entre otros. La IA puede
convertirse en un apoyo relevante para producir primeros borradores de
estos recursos, que luego la/el docente revisa, ajusta y contextualiza
según su curso, su proyecto educativo y las Bases Curriculares.

Esto puede ahorrar tiempo en la redacción inicial y abrir posibilidades
para ofrecer más variedad de ejemplos y actividades, manteniendo siempre
el criterio profesional sobre la pertinencia pedagógica, la carga de
trabajo de las y los estudiantes y el nivel de dificultad.

\textbf{Usos posibles en materiales:}

\begin{itemize}
\tightlist
\item
  \textbf{Generación de insumos:}

  \begin{itemize}
  \tightlist
  \item
    Generar ejercicios adicionales a partir de un contenido ya definido
    (por ejemplo, más problemas de proporcionalidad, más oraciones para
    analizar, más ítems de comprensión lectora).
  \item
    Proponer ejemplos contextualizados a la realidad chilena (barrio,
    transporte, ferias, medios de comunicación locales, etc.).
  \item
    Crear casos o situaciones para debates, estudios de caso o trabajo
    en grupo.
  \end{itemize}
\item
  \textbf{Adaptación y diferenciación:}

  \begin{itemize}
  \tightlist
  \item
    Adaptar un mismo texto a distintos niveles de complejidad (más
    sencillo o más desafiante).
  \item
    Crear versiones alternativas de una actividad para diferentes
    niveles de logro dentro del mismo curso.
  \item
    Reformular instrucciones para que sean más claras para estudiantes
    de distintos cursos o para quienes presentan NEE.
  \end{itemize}
\item
  \textbf{Transformación de formatos:}

  \begin{itemize}
  \tightlist
  \item
    Transformar un contenido en distintos formatos: preguntas de opción
    múltiple, desarrollo, verdadero/falso, organizadores gráficos, etc.
  \item
    Generar bancos iniciales de preguntas a partir de un texto escolar o
    una guía ya existente.
  \end{itemize}
\end{itemize}

\textbf{Ejemplo de prompt para generar y adaptar materiales:}

\begin{Shaded}
\begin{Highlighting}[]
\NormalTok{Actúa como profesor/a de }\CommentTok{[}\OtherTok{ASIGNATURA}\CommentTok{]}\NormalTok{ con experiencia en diseño de materiales didácticos para escuelas públicas chilenas.}

\NormalTok{Tu objetivo es ayudarme a generar y adaptar materiales para trabajar el tema }\CommentTok{[}\OtherTok{TEMA ESPECÍFICO}\CommentTok{]}\NormalTok{ con un curso de }\CommentTok{[}\OtherTok{NIVEL}\CommentTok{]}\NormalTok{ en el marco de las Bases Curriculares.}

\NormalTok{Contexto:}
\SpecialStringTok{{-} }\NormalTok{Tipo de establecimiento: }\CommentTok{[}\OtherTok{público, subvencionado, técnico{-}profesional, rural, urbano, etc.}\CommentTok{]}\NormalTok{.}
\SpecialStringTok{{-} }\NormalTok{Características del grupo: }\CommentTok{[}\OtherTok{tamaño del curso, diversidad de niveles, presencia de PIE, estudiantes con diferentes trayectorias escolares, etc.}\CommentTok{]}\NormalTok{.}
\SpecialStringTok{{-} }\NormalTok{Recursos disponibles: }\CommentTok{[}\OtherTok{pizarra, cuadernos, proyector, fotocopias, laboratorio, etc.}\CommentTok{]}\NormalTok{.}

\NormalTok{Instrucciones:}
\SpecialStringTok{1. }\NormalTok{Propón }\CommentTok{[}\OtherTok{NÚMERO}\CommentTok{]}\NormalTok{ ejercicios o actividades breves para trabajar este contenido.}
\SpecialStringTok{   {-} }\NormalTok{Para cada actividad, indica: propósito, descripción breve y tiempo estimado.}
\SpecialStringTok{2. }\NormalTok{Luego, elige UNA de las actividades y genera:}
\SpecialStringTok{   {-} }\NormalTok{Una versión más sencilla para estudiantes que necesitan mayor apoyo (por ejemplo, con ejemplos resueltos o pasos más guiados).}
\SpecialStringTok{   {-} }\NormalTok{Una versión más desafiante para estudiantes que avanzan más rápido (por ejemplo, con preguntas abiertas o aplicación a nuevas situaciones).}

\NormalTok{Formato de salida:}
\SpecialStringTok{{-} }\NormalTok{Primero, una lista numerada de actividades con su propósito, descripción y tiempo.}
\SpecialStringTok{{-} }\NormalTok{Después, subtítulos “Versión más sencilla” y “Versión más desafiante” para la actividad elegida, explicadas en no más de 6 líneas cada una.}

\NormalTok{Criterios y restricciones:}
\SpecialStringTok{{-} }\NormalTok{Usa lenguaje claro, adecuado al nivel del curso.}
\SpecialStringTok{{-} }\NormalTok{No incluyas recursos costosos ni que requieran conexión a internet permanente, a menos que los haya mencionado explícitamente.}
\SpecialStringTok{{-} }\NormalTok{No cambies el contenido central; solo ajusta la complejidad y el tipo de apoyo.}
\SpecialStringTok{{-} }\NormalTok{Si te falta información importante, hazme hasta 3 preguntas breves antes de responder.}
\end{Highlighting}
\end{Shaded}

\begin{tcolorbox}[enhanced jigsaw, colframe=quarto-callout-note-color-frame, opacityback=0, coltitle=black, colbacktitle=quarto-callout-note-color!10!white, titlerule=0mm, left=2mm, breakable, leftrule=.75mm, bottomtitle=1mm, opacitybacktitle=0.6, toptitle=1mm, bottomrule=.15mm, toprule=.15mm, arc=.35mm, rightrule=.15mm, title=\textcolor{quarto-callout-note-color}{\faInfo}\hspace{0.5em}{Nota}, colback=white]

\textbf{Idea fuerza:} la IA puede apoyar la creación y adaptación de
materiales, pero siempre como borrador inicial. El ajuste fino, la
pertinencia cultural y la coherencia con el proyecto educativo dependen
de la revisión y edición de la/el docente.

\end{tcolorbox}

\section{IA para diversificar actividades según
curso}\label{ia-para-diversificar-actividades-seguxfan-curso}

En un mismo curso suelen convivir estudiantes con ritmos de aprendizaje
distintos, intereses variados, trayectorias escolares diversas y, en
muchos casos, con participación de programas de integración escolar
(PIE). La IA puede ayudar a generar variantes de una misma actividad
para atender esta diversidad, siempre que la/el docente mantenga el
control sobre los objetivos, la evaluación y el sentido formativo de las
adaptaciones.

El papel de la IA no es decidir quién hace qué, sino ofrecer un abanico
de opciones que la/el profesor/a puede asignar, combinar o adaptar a su
grupo, evitando etiquetar estudiantes y resguardando una mirada
inclusiva y de derechos.

\textbf{Usos posibles en diversificación:}

\begin{itemize}
\tightlist
\item
  \textbf{Niveles de apoyo y desafío:}

  \begin{itemize}
  \tightlist
  \item
    Diseñar tres niveles de dificultad para una misma actividad (más
    guiada, intermedia, profundización).
  \item
    Proponer actividades de repaso para quienes necesitan consolidar y
    desafíos extra para quienes avanzan más rápido.
  \end{itemize}
\item
  \textbf{Canales y formas de expresión:}

  \begin{itemize}
  \tightlist
  \item
    Proponer actividades que privilegien diferentes canales: oral,
    escrito, visual, manipulativo.
  \item
    Sugerir formas alternativas de mostrar lo aprendido (afiches,
    pequeñas presentaciones, mapas conceptuales, cápsulas de audio,
    etc.).
  \end{itemize}
\item
  \textbf{Apoyos adicionales:}

  \begin{itemize}
  \tightlist
  \item
    Sugerir apoyos (preguntas guía, ejemplos resueltos, organizadores
    gráficos, listas de palabras clave) para estudiantes con NEE o
    trayectorias más interrumpidas.
  \item
    Adaptar consignas para que sean más claras y breves, manteniendo el
    mismo objetivo de aprendizaje.
  \end{itemize}
\end{itemize}

\textbf{Ejemplo de prompt para diversificar una actividad base:}

\begin{Shaded}
\begin{Highlighting}[]
\NormalTok{Actúa como especialista en diferenciación pedagógica e inclusión educativa en contexto escolar chileno.}

\NormalTok{Tu objetivo es ayudarme a diversificar una actividad sobre }\CommentTok{[}\OtherTok{TEMA}\CommentTok{]}\NormalTok{ para un curso de }\CommentTok{[}\OtherTok{NIVEL}\CommentTok{]}\NormalTok{, manteniendo el mismo contenido central y una mirada inclusiva.}

\NormalTok{Contexto:}
\SpecialStringTok{{-} }\NormalTok{Tipo de establecimiento: }\CommentTok{[}\OtherTok{público/subvencionado/particular, rural/urbano}\CommentTok{]}\NormalTok{.}
\SpecialStringTok{{-} }\NormalTok{Curso diverso en niveles de logro; hay estudiantes que requieren más apoyo, otros que avanzan más rápido y presencia de PIE.}
\SpecialStringTok{{-} }\NormalTok{Recursos disponibles: }\CommentTok{[}\OtherTok{pizarra, cuadernos, proyector, impresora, etc.}\CommentTok{]}\NormalTok{.}

\NormalTok{Actividad base:}
\CommentTok{[}\OtherTok{PEGAR AQUÍ LA ACTIVIDAD ORIGINAL QUE YA DISEÑÓ EL/LA DOCENTE}\CommentTok{]}\NormalTok{.}

\NormalTok{Instrucciones:}
\SpecialStringTok{1. }\NormalTok{A partir de la actividad base, propone tres versiones:}
\SpecialStringTok{   {-} }\NormalTok{Una “Versión más guiada” para estudiantes que necesitan mayor apoyo.}
\SpecialStringTok{   {-} }\NormalTok{Una “Versión intermedia” para la mayoría del curso.}
\SpecialStringTok{   {-} }\NormalTok{Una “Versión de profundización” para quienes pueden avanzar más rápido.}
\SpecialStringTok{2. }\NormalTok{Para cada versión, indica:}
\SpecialStringTok{   {-} }\NormalTok{Propósito.}
\SpecialStringTok{   {-} }\NormalTok{Pasos principales de la actividad.}
\SpecialStringTok{   {-} }\NormalTok{Tipo de apoyo o desafío que se incluye (ejemplos, preguntas guía, uso de material concreto, etc.).}
\SpecialStringTok{3. }\NormalTok{Evita etiquetar a los grupos como “buenos” o “malos”; describe las versiones sin juicios de valor.}

\NormalTok{Formato de salida:}
\SpecialStringTok{{-} }\NormalTok{Subtítulos: “Versión más guiada”, “Versión intermedia” y “Versión de profundización”.}
\SpecialStringTok{{-} }\NormalTok{Bajo cada subtítulo, un breve párrafo de 6 a 8 líneas describiendo la propuesta.}

\NormalTok{Criterios:}
\SpecialStringTok{{-} }\NormalTok{Mantén el mismo contenido central en las tres versiones.}
\SpecialStringTok{{-} }\NormalTok{Usa lenguaje claro y respetuoso, sin etiquetas que estigmaticen a estudiantes.}
\SpecialStringTok{{-} }\NormalTok{No propongas actividades que dependan de tecnologías que no se han mencionado.}
\SpecialStringTok{{-} }\NormalTok{Si te falta información importante, hazme hasta 3 preguntas breves antes de responder.}
\end{Highlighting}
\end{Shaded}

\begin{tcolorbox}[enhanced jigsaw, colframe=quarto-callout-note-color-frame, opacityback=0, coltitle=black, colbacktitle=quarto-callout-note-color!10!white, titlerule=0mm, left=2mm, breakable, leftrule=.75mm, bottomtitle=1mm, opacitybacktitle=0.6, toptitle=1mm, bottomrule=.15mm, toprule=.15mm, arc=.35mm, rightrule=.15mm, title=\textcolor{quarto-callout-note-color}{\faInfo}\hspace{0.5em}{Nota}, colback=white]

\textbf{Idea fuerza:} la IA puede ayudar a pensar distintas puertas de
entrada a un mismo contenido, pero nunca para clasificar ni segregar.
Las decisiones sobre qué versión ofrece a quién, y cómo se comunica,
siguen siendo parte del criterio pedagógico e inclusivo de la/el docente
y del equipo de apoyo.

\end{tcolorbox}

\section{IA para organizar el trabajo cotidiano (síntesis,
comunicaciones, actas,
etc.)}\label{ia-para-organizar-el-trabajo-cotidiano-suxedntesis-comunicaciones-actas-etc.}

Además de las tareas directamente pedagógicas, el trabajo docente
incluye una importante carga administrativa y de organización: lectura
de documentos extensos, elaboración de actas, sistematización de
acuerdos, redacción de comunicaciones a familias, entre otras. La IA
puede ser un apoyo para sintetizar, ordenar y redactar borradores,
siempre que se tengan resguardos claros para no exponer datos sensibles
de estudiantes, familias o colegas (RUT, nombres completos,
diagnósticos, antecedentes de salud, etc.).

Utilizada con cuidado, la IA puede ayudar a disminuir el tiempo dedicado
a la redacción inicial de estos documentos, de modo que las y los
profesores puedan concentrarse en la toma de decisiones, la coordinación
con sus equipos y la atención directa a estudiantes.

\textbf{Usos posibles en organización del trabajo:}

\begin{itemize}
\tightlist
\item
  \textbf{Síntesis y sistematización:}

  \begin{itemize}
  \tightlist
  \item
    Resumir documentos largos, orientaciones o actas en síntesis breves
    y claras.
  \item
    Transformar un conjunto de notas dispersas en listas de acuerdos o
    tareas ordenadas por prioridad o responsable.
  \end{itemize}
\item
  \textbf{Comunicaciones:}

  \begin{itemize}
  \tightlist
  \item
    Proponer borradores de comunicaciones a familias (reuniones,
    actividades especiales, cambios de horario), que luego la/el docente
    revisa y ajusta.
  \item
    Sugerir versiones alternativas del mismo mensaje para distintos
    medios (agenda, correo, WhatsApp).
  \end{itemize}
\item
  \textbf{Estructura de documentos:}

  \begin{itemize}
  \tightlist
  \item
    Sugerir estructuras de actas o pautas para registrar acuerdos de
    reuniones de ciclo, subdirecciones, consejos escolares, etc.
  \item
    Ordenar ideas para proyectos, unidades o talleres en esquemas más
    claros.
  \end{itemize}
\end{itemize}

\textbf{Ejemplo de prompt para síntesis y organización de información}

\begin{Shaded}
\begin{Highlighting}[]
\NormalTok{Actúa como asistente de organización de trabajo docente en una escuela pública de Chile.}

\NormalTok{Tu objetivo es ayudarme a sintetizar y ordenar información para facilitar mi trabajo.}

\NormalTok{Contexto:}
\SpecialStringTok{{-} }\NormalTok{Soy profesor/a de }\CommentTok{[}\OtherTok{NIVEL/ASIGNATURA}\CommentTok{]}\NormalTok{.}
\SpecialStringTok{{-} }\NormalTok{Necesito convertir un conjunto de notas e información en un resumen claro y una lista de acuerdos o tareas.}

\NormalTok{Instrucciones:}
\SpecialStringTok{1. }\NormalTok{A partir del texto que pegaré a continuación (sin datos sensibles como RUT, diagnósticos o nombres completos), genera:}
\SpecialStringTok{   {-} }\NormalTok{Un resumen breve de máximo 150 palabras.}
\SpecialStringTok{   {-} }\NormalTok{Una lista de 5 a 7 acuerdos o tareas concretas.}
\SpecialStringTok{2. }\NormalTok{Si detectas información poco clara o contradictoria, señálalo al final en un apartado de observaciones.}

\NormalTok{Texto a sintetizar:}
\CommentTok{[}\OtherTok{PEGAR AQUÍ NOTAS, ACTA O DOCUMENTO SIN DATOS SENSIBLES}\CommentTok{]}\NormalTok{.}

\NormalTok{Formato de salida:}
\SpecialStringTok{{-} }\NormalTok{Subtítulo “Resumen breve” y el resumen en un párrafo.}
\SpecialStringTok{{-} }\NormalTok{Subtítulo “Lista de acuerdos o tareas” y la lista numerada.}
\SpecialStringTok{{-} }\NormalTok{Si corresponde, subtítulo “Observaciones” con 2 a 3 líneas.}

\NormalTok{Criterios y restricciones:}
\SpecialStringTok{{-} }\NormalTok{No inventes acuerdos ni información que no esté en el texto.}
\SpecialStringTok{{-} }\NormalTok{Usa lenguaje claro y profesional.}
\SpecialStringTok{{-} }\NormalTok{No incluyas nombres propios ni datos personales; reemplázalos por descripciones generales (por ejemplo, “un/a estudiante”, “un/a apoderado/a”).}
\SpecialStringTok{{-} }\NormalTok{Si te falta información importante, hazme hasta 3 preguntas breves antes de responder.}
\end{Highlighting}
\end{Shaded}

\textbf{Ejemplo de prompt para redactar un comunicado breve a familias}

\begin{Shaded}
\begin{Highlighting}[]
\NormalTok{Actúa como profesor/a jefe con experiencia en comunicación clara y respetuosa con familias en el sistema escolar chileno.}

\NormalTok{Tu objetivo es ayudarme a redactar un comunicado breve para apoderadas y apoderados sobre }\CommentTok{[}\OtherTok{TEMA: por ejemplo, reunión, actividad especial, cambio de horario}\CommentTok{]}\NormalTok{.}

\NormalTok{Contexto:}
\SpecialStringTok{{-} }\NormalTok{Curso: }\CommentTok{[}\OtherTok{por ejemplo, 5º básico}\CommentTok{]}\NormalTok{.}
\SpecialStringTok{{-} }\NormalTok{Tipo de establecimiento: }\CommentTok{[}\OtherTok{público/subvencionado/particular}\CommentTok{]}\NormalTok{.}
\SpecialStringTok{{-} }\NormalTok{Medio de envío: }\CommentTok{[}\OtherTok{agenda, correo electrónico, WhatsApp}\CommentTok{]}\NormalTok{.}

\NormalTok{Instrucciones:}
\SpecialStringTok{{-} }\NormalTok{Redacta un texto de máximo 180 palabras.}
\SpecialStringTok{{-} }\NormalTok{Usa un tono cercano, respetuoso y profesional.}
\SpecialStringTok{{-} }\NormalTok{Explica de forma sencilla qué ocurrirá, cuándo, dónde y por qué es importante la actividad o la información.}
\SpecialStringTok{{-} }\NormalTok{Incluye, si corresponde, qué se espera de las familias (asistencia, autorización, envío de materiales, etc.).}

\NormalTok{Formato de salida:}
\SpecialStringTok{{-} }\NormalTok{Texto continuo, listo para copiar y pegar en el medio de comunicación indicado.}

\NormalTok{Criterios y restricciones:}
\SpecialStringTok{{-} }\NormalTok{Evita tecnicismos; si mencionas “inteligencia artificial” u otros conceptos, explícalos en palabras simples.}
\SpecialStringTok{{-} }\NormalTok{No incluyas datos personales ni información que no te haya proporcionado.}
\SpecialStringTok{{-} }\NormalTok{Mantén un enfoque colaborativo, reconociendo el rol de las familias en el proceso educativo.}
\SpecialStringTok{{-} }\NormalTok{Si te falta información importante, hazme hasta 3 preguntas breves antes de responder.}
\end{Highlighting}
\end{Shaded}

En todos estos casos, la IA se utiliza como apoyo para organizar,
sintetizar y redactar, pero la revisión final, la adecuación al contexto
y la decisión sobre qué se envía o se registra siguen siendo
responsabilidad profesional de la/el docente y de los equipos escolares.

\begin{tcolorbox}[enhanced jigsaw, colframe=quarto-callout-note-color-frame, opacityback=0, coltitle=black, colbacktitle=quarto-callout-note-color!10!white, titlerule=0mm, left=2mm, breakable, leftrule=.75mm, bottomtitle=1mm, opacitybacktitle=0.6, toptitle=1mm, bottomrule=.15mm, toprule=.15mm, arc=.35mm, rightrule=.15mm, title=\textcolor{quarto-callout-note-color}{\faInfo}\hspace{0.5em}{Nota}, colback=white]

\textbf{Idea fuerza:} la IA puede aliviar parte de la carga de escritura
y organización, pero nunca debe recibir ni procesar datos sensibles, y
su producción siempre debe pasar por una revisión crítica antes de
compartirse con estudiantes, familias o equipos directivos.

\end{tcolorbox}

\section{IA para proponer ítems, rúbricas y
criterios}\label{ia-para-proponer-uxedtems-ruxfabricas-y-criterios}

El diseño de evaluaciones es una de las tareas más exigentes del trabajo
docente: hay que formular buenas preguntas, definir criterios claros,
cuidar la coherencia con los objetivos de aprendizaje y, además, hacerlo
con poco tiempo. La inteligencia artificial puede apoyar este proceso
como asistente de borradores: propone ideas iniciales de ítems, rúbricas
y criterios que luego la/el docente revisa, ajusta y valida según las
Bases Curriculares chilenas y el proyecto educativo del establecimiento.

Usada de manera crítica y situada, la IA permite ganar tiempo en la fase
inicial de diseño, ampliar el repertorio de ejemplos y explicitar
criterios, sin reemplazar nunca las decisiones profesionales sobre qué,
cómo y para qué evaluar.

\textbf{{Usos posibles (ejemplos concretos)}.}

\begin{itemize}
\tightlist
\item
  \textbf{Ítems e instrumentos de evaluación}

  \begin{itemize}
  \tightlist
  \item
    Proponer ítems de evaluación (alternativa, desarrollo,
    verdadero/falso, preguntas abiertas) coherentes con OA u objetivos
    priorizados.
  \item
    Sugerir tareas de desempeño simples (pequeños proyectos,
    exposiciones, informes breves) alineadas con habilidades de las
    Bases Curriculares.
  \item
    Ofrecer variantes de una misma pregunta con distintos niveles de
    dificultad para usar en distintos cursos o en evaluaciones
    diferenciadas.
  \end{itemize}
\item
  \textbf{Rúbricas y criterios}

  \begin{itemize}
  \tightlist
  \item
    Sugerir rúbricas simples o pautas de corrección para trabajos
    escritos, proyectos, presentaciones orales, etc.
  \item
    Ayudar a redactar criterios de evaluación en lenguaje claro,
    comprensible para estudiantes y familias.
  \item
    Transformar criterios ``técnicos'' del equipo docente en
    descriptores más concretos (``se entiende la idea principal'', ``usa
    al menos dos ejemplos del texto'', etc.).
  \end{itemize}
\item
  \textbf{Organización del banco de ítems}

  \begin{itemize}
  \tightlist
  \item
    Generar bancos iniciales de preguntas sobre un contenido que luego
    el equipo docente revisa, selecciona y ajusta.
  \item
    Proponer agrupaciones de ítems según habilidad (comprensión,
    análisis, aplicación) o nivel de complejidad.
  \end{itemize}
\end{itemize}

\begin{tcolorbox}[enhanced jigsaw, colframe=quarto-callout-note-color-frame, opacityback=0, coltitle=black, colbacktitle=quarto-callout-note-color!10!white, titlerule=0mm, left=2mm, breakable, leftrule=.75mm, bottomtitle=1mm, opacitybacktitle=0.6, toptitle=1mm, bottomrule=.15mm, toprule=.15mm, arc=.35mm, rightrule=.15mm, title=\textcolor{quarto-callout-note-color}{\faInfo}\hspace{0.5em}{Nota}, colback=white]

\textbf{Idea fuerza:} la IA puede ayudar a ``llenar la hoja en blanco''
con ideas de preguntas, criterios y rúbricas, pero lo que se evalúa y
cómo se califica siempre lo decide la/el docente.

\end{tcolorbox}

\textbf{Ejemplo de prompt para proponer ítems, rúbricas y criterios}

\begin{Shaded}
\begin{Highlighting}[]
\NormalTok{Actúa como asesor/a en evaluación formativa en el sistema escolar chileno.}

\NormalTok{Tu objetivo es ayudarme a diseñar una evaluación breve y una rúbrica sencilla para }\CommentTok{[}\OtherTok{TIPO DE TAREA: por ejemplo, informe escrito, presentación oral, proyecto grupal}\CommentTok{]}\NormalTok{ en la asignatura de }\CommentTok{[}\OtherTok{ASIGNATURA}\CommentTok{]}\NormalTok{ para un curso de }\CommentTok{[}\OtherTok{NIVEL}\CommentTok{]}\NormalTok{.}

\NormalTok{Contexto:}
\SpecialStringTok{{-} }\NormalTok{Curso: }\CommentTok{[}\OtherTok{por ejemplo, 8º básico, 40 estudiantes}\CommentTok{]}\NormalTok{.}
\SpecialStringTok{{-} }\NormalTok{Tipo de establecimiento: }\CommentTok{[}\OtherTok{público/subvencionado/particular}\CommentTok{]}\NormalTok{.}
\SpecialStringTok{{-} }\NormalTok{Contenidos y habilidades a evaluar: }\CommentTok{[}\OtherTok{describir brevemente o pegar OA de las Bases Curriculares}\CommentTok{]}\NormalTok{.}
\SpecialStringTok{{-} }\NormalTok{Tiempo estimado para la tarea: }\CommentTok{[}\OtherTok{por ejemplo, 2 clases}\CommentTok{]}\NormalTok{.}
\SpecialStringTok{{-} }\NormalTok{Realidad del grupo: }\CommentTok{[}\OtherTok{curso diverso, presencia de PIE, diferencias importantes en niveles de logro, etc.}\CommentTok{]}\NormalTok{.}

\NormalTok{Instrucciones:}
\SpecialStringTok{1. }\NormalTok{Propón entre 5 y 8 ítems o tareas concretas que permitan evaluar estos contenidos y habilidades.}
\SpecialStringTok{   {-} }\NormalTok{Incluye, si es posible, al menos una pregunta de desarrollo y una de alternativa.}
\SpecialStringTok{2. }\NormalTok{Diseña una rúbrica sencilla con 3 o 4 criterios clave (por ejemplo, dominio de contenidos, claridad, organización, uso de ejemplos) y 4 niveles de logro.}
\SpecialStringTok{3. }\NormalTok{Redacta los criterios y descriptores en lenguaje claro, comprensible para estudiantes y familias.}
\SpecialStringTok{4. }\NormalTok{Incluye una breve nota final con recomendaciones para usar esta rúbrica en clave formativa (por ejemplo, para retroalimentar y no solo calificar).}

\NormalTok{Formato de salida:}
\SpecialStringTok{{-} }\NormalTok{Primero, una lista numerada de ítems o tareas.}
\SpecialStringTok{{-} }\NormalTok{Luego, una tabla con la rúbrica (filas = criterios, columnas = niveles de logro).}
\SpecialStringTok{{-} }\NormalTok{Finalmente, una sección breve llamada “Recomendaciones para el uso formativo”.}

\NormalTok{Criterios y restricciones:}
\SpecialStringTok{{-} }\NormalTok{Usa lenguaje sencillo y adecuado al nivel del curso.}
\SpecialStringTok{{-} }\NormalTok{No inventes contenidos fuera de las Bases Curriculares; mantente dentro de lo que se ha trabajado en clase.}
\SpecialStringTok{{-} }\NormalTok{No incluyas ejemplos que requieran recursos tecnológicos que no he mencionado.}
\SpecialStringTok{{-} }\NormalTok{Mantén un enfoque inclusivo y respetuoso, evitando estereotipos o juicios hacia estudiantes.}
\end{Highlighting}
\end{Shaded}

\section{Ejemplos de retroalimentación formativa apoyada por
IA}\label{ejemplos-de-retroalimentaciuxf3n-formativa-apoyada-por-ia}

La retroalimentación formativa es una de las prácticas más potentes para
el aprendizaje, pero también una de las más demandantes en tiempo,
especialmente en cursos numerosos. La IA puede colaborar generando
borradores de comentarios que la/el docente revisa, ajusta y personaliza
para sus estudiantes, manteniendo el foco en criterios claros y en
orientaciones concretas de mejora.

Más que sustituir la voz del profesor o profesora, la IA puede ayudar a
encontrar palabras claras, respetuosas y específicas, especialmente
cuando hay muchos trabajos que retroalimentar en poco tiempo y cuando se
busca comunicar de manera comprensible a estudiantes y familias.

\textbf{{Usos posibles (ejemplos concretos)}.}

\begin{itemize}
\tightlist
\item
  \textbf{Modelos de comentarios}

  \begin{itemize}
  \tightlist
  \item
    Redactar comentarios modelo para distintos niveles de logro, a
    partir de una rúbrica o pauta ya definida por el equipo docente.
  \item
    Generar versiones más breves o más extensas de un mismo comentario
    según el medio (agenda, plataforma, informe, conversación
    presencial).
  \end{itemize}
\item
  \textbf{Foco en el proceso y la mejora}

  \begin{itemize}
  \tightlist
  \item
    Proponer frases de retroalimentación centradas en el proceso, no
    solo en el resultado (``lo que hiciste bien fue\ldots{}'', ``un
    próximo paso posible sería\ldots{}'').
  \item
    Sugerir preguntas que inviten a la reflexión del/la estudiante sobre
    su propio trabajo (``¿qué parte te resultó más difícil y por
    qué?'').
  \end{itemize}
\item
  \textbf{Diversidad de estudiantes}

  \begin{itemize}
  \tightlist
  \item
    Apoyar la elaboración de retroalimentación escrita para estudiantes
    con distintas necesidades, por ejemplo, formulando comentarios en
    lenguaje más sencillo o más visual.
  \item
    Ayudar a redactar mensajes diferenciados para estudiantes que
    requieren más acompañamiento y para quienes pueden asumir desafíos
    adicionales.
  \end{itemize}
\end{itemize}

\begin{tcolorbox}[enhanced jigsaw, colframe=quarto-callout-note-color-frame, opacityback=0, coltitle=black, colbacktitle=quarto-callout-note-color!10!white, titlerule=0mm, left=2mm, breakable, leftrule=.75mm, bottomtitle=1mm, opacitybacktitle=0.6, toptitle=1mm, bottomrule=.15mm, toprule=.15mm, arc=.35mm, rightrule=.15mm, title=\textcolor{quarto-callout-note-color}{\faInfo}\hspace{0.5em}{Nota}, colback=white]

\textbf{Idea fuerza:} la IA puede ayudar a ``afinar'' el lenguaje de la
retroalimentación y a ganar tiempo, pero la mirada pedagógica sobre qué
decirle a cada estudiante sigue siendo insustituible.

\end{tcolorbox}

\textbf{Ejemplo de prompt para generar retroalimentación formativa}

\begin{Shaded}
\begin{Highlighting}[]
\NormalTok{Actúa como asesor/a en evaluación formativa y retroalimentación para el sistema escolar chileno.}

\NormalTok{Tu objetivo es ayudarme a redactar comentarios de retroalimentación formativa para estudiantes de }\CommentTok{[}\OtherTok{NIVEL}\CommentTok{]}\NormalTok{ que han realizado la siguiente tarea:}
\CommentTok{[}\OtherTok{DESCRIBIR BREVE LA TAREA: por ejemplo, “ensayo argumentativo sobre el uso de celulares en el aula”}\CommentTok{]}\NormalTok{.}

\NormalTok{Contexto:}
\SpecialStringTok{{-} }\NormalTok{Asignatura: }\CommentTok{[}\OtherTok{ASIGNATURA}\CommentTok{]}\NormalTok{.}
\SpecialStringTok{{-} }\NormalTok{Tipo de establecimiento: }\CommentTok{[}\OtherTok{público/subvencionado/particular}\CommentTok{]}\NormalTok{.}
\SpecialStringTok{{-} }\NormalTok{Criterios de evaluación: }\CommentTok{[}\OtherTok{pegar criterios o rúbrica resumida}\CommentTok{]}\NormalTok{.}
\SpecialStringTok{{-} }\NormalTok{Realidad del curso: }\CommentTok{[}\OtherTok{por ejemplo, curso numeroso, presencia de PIE, diferencias importantes en comprensión lectora}\CommentTok{]}\NormalTok{.}
\SpecialStringTok{{-} }\NormalTok{Necesito ejemplos para tres niveles de desempeño: inicial, intermedio y avanzado.}

\NormalTok{Instrucciones:}
\SpecialStringTok{1. }\NormalTok{Propón 3 comentarios de retroalimentación formativa:}
\SpecialStringTok{   {-} }\NormalTok{Uno para desempeño inicial.}
\SpecialStringTok{   {-} }\NormalTok{Uno para desempeño intermedio.}
\SpecialStringTok{   {-} }\NormalTok{Uno para desempeño avanzado.}
\SpecialStringTok{2. }\NormalTok{Cada comentario debe:}
\SpecialStringTok{   {-} }\NormalTok{Mencionar al menos un aspecto logrado.}
\SpecialStringTok{   {-} }\NormalTok{Señalar con claridad qué se puede mejorar.}
\SpecialStringTok{   {-} }\NormalTok{Sugerir un próximo paso concreto para el/la estudiante.}
\SpecialStringTok{3. }\NormalTok{Redacta los comentarios en segunda persona (“tú”) y en lenguaje cercano, respetuoso y motivador.}

\NormalTok{Formato de salida:}
\SpecialStringTok{{-} }\NormalTok{Subtítulos: “Desempeño inicial”, “Desempeño intermedio”, “Desempeño avanzado”.}
\SpecialStringTok{{-} }\NormalTok{Bajo cada subtítulo, un comentario de máximo 6 líneas.}

\NormalTok{Criterios y restricciones:}
\SpecialStringTok{{-} }\NormalTok{No uses un tono punitivo ni culposo.}
\SpecialStringTok{{-} }\NormalTok{Evita tecnicismos; prioriza explicaciones sencillas.}
\SpecialStringTok{{-} }\NormalTok{Mantén el foco en el aprendizaje y la mejora, no solo en la calificación.}
\SpecialStringTok{{-} }\NormalTok{No incluyas datos personales ni información sensible sobre estudiantes.}
\end{Highlighting}
\end{Shaded}

\section{IA como apoyo a la autorregulación del
aprendizaje}\label{ia-como-apoyo-a-la-autorregulaciuxf3n-del-aprendizaje}

La autorregulación del aprendizaje implica que las y los estudiantes
puedan planificar, monitorear y evaluar su propio trabajo. La IA puede
contribuir ofreciendo andamiajes para que revisen sus producciones,
comparen con criterios de calidad y tomen decisiones sobre cómo mejorar,
siempre bajo acompañamiento y orientación docente.

El énfasis no está en que la IA ``corrija'' por ellos, sino en que
entregue preguntas guía, listas de chequeo y sugerencias que ayuden a
los estudiantes a mirar críticamente lo que hicieron y a hacerse
responsables de sus procesos de aprendizaje, en coherencia con los
objetivos del curso y con las Bases Curriculares.

\textbf{{Usos posibles (ejemplos concretos)}.}

\begin{itemize}
\tightlist
\item
  \textbf{Revisión antes de entregar}

  \begin{itemize}
  \tightlist
  \item
    Generar listas de verificación (checklists) para que estudiantes
    revisen sus trabajos antes de entregarlos (``revisé que\ldots{}'',
    ``me aseguré de\ldots{}'').
  \item
    Transformar criterios de evaluación en lenguaje amigable para
    estudiantes, que puedan usar como guía.
  \end{itemize}
\item
  \textbf{Autoevaluación y metacognición}

  \begin{itemize}
  \tightlist
  \item
    Proponer preguntas de autoevaluación alineadas con criterios o
    rúbricas (``¿qué parte de tu trabajo muestra mejor tu argumento?'').
  \item
    Sugerir preguntas para reflexionar sobre el proceso (``¿en qué
    momento te sentiste más perdido/a?'').
  \end{itemize}
\item
  \textbf{Estrategias de mejora}

  \begin{itemize}
  \tightlist
  \item
    Sugerir estrategias de estudio o de mejora a partir de dificultades
    detectadas (por ejemplo, ``si te costó organizar tus ideas, prueba
    primero hacer un esquema'').
  \item
    Crear esquemas de planificación (pasos para desarrollar un proyecto,
    un informe, una investigación corta) que los estudiantes puedan
    completar.
  \end{itemize}
\end{itemize}

\begin{tcolorbox}[enhanced jigsaw, colframe=quarto-callout-note-color-frame, opacityback=0, coltitle=black, colbacktitle=quarto-callout-note-color!10!white, titlerule=0mm, left=2mm, breakable, leftrule=.75mm, bottomtitle=1mm, opacitybacktitle=0.6, toptitle=1mm, bottomrule=.15mm, toprule=.15mm, arc=.35mm, rightrule=.15mm, title=\textcolor{quarto-callout-note-color}{\faInfo}\hspace{0.5em}{Nota}, colback=white]

\textbf{Idea fuerza:} la IA puede ayudar a que los y las estudiantes se
hagan mejores preguntas sobre su propio trabajo, pero la construcción de
hábitos de estudio y responsabilidad sigue dependiendo del
acompañamiento humano.

\end{tcolorbox}

\textbf{Ejemplo de prompt para apoyar la autorregulación del
aprendizaje}

\begin{Shaded}
\begin{Highlighting}[]
\NormalTok{Actúa como orientador/a pedagógico/a especializado/a en autorregulación del aprendizaje en educación básica/media.}

\NormalTok{Tu objetivo es ayudarme a generar apoyos para que estudiantes de }\CommentTok{[}\OtherTok{NIVEL}\CommentTok{]}\NormalTok{ se autoevalúen y mejoren su trabajo en }\CommentTok{[}\OtherTok{TIPO DE TAREA: por ejemplo, informe escrito, proyecto de ciencias, presentación oral}\CommentTok{]}\NormalTok{.}

\NormalTok{Contexto:}
\SpecialStringTok{{-} }\NormalTok{Asignatura: }\CommentTok{[}\OtherTok{ASIGNATURA}\CommentTok{]}\NormalTok{.}
\SpecialStringTok{{-} }\NormalTok{Tipo de establecimiento: }\CommentTok{[}\OtherTok{público/subvencionado/particular}\CommentTok{]}\NormalTok{.}
\SpecialStringTok{{-} }\NormalTok{Criterios de evaluación que usamos: }\CommentTok{[}\OtherTok{describir brevemente o pegar criterios/rúbrica}\CommentTok{]}\NormalTok{.}
\SpecialStringTok{{-} }\NormalTok{Realidad del grupo: }\CommentTok{[}\OtherTok{por ejemplo, curso diverso, presencia de PIE, diferencias en hábitos de estudio}\CommentTok{]}\NormalTok{.}

\NormalTok{Instrucciones:}
\SpecialStringTok{1. }\NormalTok{Elabora una lista de verificación (“checklist”) que los estudiantes puedan usar antes de entregar su trabajo.}
\SpecialStringTok{   {-} }\NormalTok{Máximo 10 ítems, redactados en primera persona (“revisé que…”, “me aseguré de…”).}
\SpecialStringTok{2. }\NormalTok{Propón 5 preguntas de autoevaluación que les ayuden a reflexionar sobre su proceso (no solo sobre la nota).}
\SpecialStringTok{3. }\NormalTok{Sugiere 3 ideas de “próximos pasos” que puedan tomar si se dan cuenta de que necesitan mejorar.}

\NormalTok{Formato de salida:}
\SpecialStringTok{{-} }\NormalTok{Sección “Checklist de revisión antes de entregar” con la lista numerada.}
\SpecialStringTok{{-} }\NormalTok{Sección “Preguntas para autoevaluar mi trabajo” con 5 preguntas.}
\SpecialStringTok{{-} }\NormalTok{Sección “Qué puedo hacer para mejorar” con 3 sugerencias breves.}

\NormalTok{Criterios y restricciones:}
\SpecialStringTok{{-} }\NormalTok{Usa lenguaje claro, dirigido a estudiantes de }\CommentTok{[}\OtherTok{NIVEL}\CommentTok{]}\NormalTok{.}
\SpecialStringTok{{-} }\NormalTok{Evita frases negativas; formula los ítems y preguntas en clave de apoyo y mejora.}
\SpecialStringTok{{-} }\NormalTok{Aclara que estas herramientas son para que el/la estudiante mejore su trabajo, no para reemplazar la retroalimentación del/la docente.}
\end{Highlighting}
\end{Shaded}

\section{Desafío}\label{desafuxedo}

Armar una propuesta o borrador lista para usar, eligiendo un tema que
les sirva o les gustaria planificar para un curso (OA pendiente,
habilidad clave o contenido difícil).

El desafío es diseñar, en tiempo acotado, una propuesta \textbf{lista
para usar}: planificar dos clases, construir una evaluación breve con su
rúbrica, crear un material para estudiantes y redactar un correo. Todo
lo que proponga la IA debe ser \textbf{revisado, corregido y decidido}
por el equipo docente.

\textbf{A generar (intentar hacer la mayor cantidad posible de
borradores):}

\begin{itemize}
\item
  Una \textbf{micro-secuencia de dos clases} (dos bloques seguidos)
  sobre un tema que elijan, con:

  \begin{itemize}
  \tightlist
  \item
    \textbf{2--3 objetivos de aprendizaje claros} (OA u objetivos
    específicos).
  \item
    \textbf{Actividades de inicio, desarrollo y cierre} para cada clase.
  \item
    \textbf{Indicación rápida de recursos necesarios} (pizarra,
    cuaderno, proyector, etc.).
  \end{itemize}
\item
  Una \textbf{actividad evaluativa breve} alineada con la
  micro-secuencia, por ejemplo:

  \begin{itemize}
  \tightlist
  \item
    \textbf{3--5 ítems} (preguntas abiertas, de desarrollo u opción
    múltiple), \textbf{o}
  \item
    \textbf{Una tarea corta} (producto escrito, ejercicio aplicado,
    pequeña presentación).
  \end{itemize}
\item
  Una \textbf{rúbrica simple o pauta de corrección} para esa actividad,
  con:

  \begin{itemize}
  \tightlist
  \item
    \textbf{3--4 criterios} (por ejemplo: comprensión del contenido,
    claridad, uso de conceptos, presentación).
  \item
    \textbf{3 o 4 niveles de logro}, descritos en \textbf{lenguaje
    claro} para estudiantes.
  \end{itemize}
\item
  \textbf{Un material adicional para estudiantes}, distinto de la
  planificación, por ejemplo:

  \begin{itemize}
  \tightlist
  \item
    Una \textbf{hoja de trabajo o guía breve} para usar durante la
    clase, \textbf{o}
  \item
    Una \textbf{``ficha de criterios''} donde se explique la rúbrica en
    lenguaje sencillo (por ejemplo: ``qué significa hacerlo bien'',
    ``qué se espera de mí'').
  \end{itemize}

  \textbf{Extensión sugerida:} \emph{1 plana máximo}, lista para
  imprimir o subir a la plataforma del colegio.
\item
  Un \textbf{mensaje o correo breve para apoderadas y apoderados},
  comunicando una salida pedagógica al Museo de la Educación Gabriela
  Mistral y la firma de consentimiento.
\end{itemize}

Se recomienda \textbf{subir como referencia} pautas, guías o actividades
que ya hayan usado antes, para que la IA \textbf{genere un borrador
similar en estilo y estructura}, que luego el equipo docente revisa y
ajusta.

\pandocbounded{\includegraphics[keepaspectratio]{assets/Referencia.png}}

\textbf{Recomendación de uso (flujo de trabajo con IA):} mantén la
herramienta de IA abierta (por ejemplo, a la izquierda) y, en paralelo,
un documento de trabajo (Word/Google Docs) donde vayas pegando y
ordenando lo útil que te entregue la IA (borradores, listas, tablas). Al
final.

\bookmarksetup{startatroot}

\chapter{IA e inclusión educativa}\label{ia-e-inclusiuxf3n-educativa}

\section{Ajuste de lenguaje, formato y extensión de
materiales}\label{ajuste-de-lenguaje-formato-y-extensiuxf3n-de-materiales}

En muchos cursos, el primer obstáculo para la participación de las y los
estudiantes es el \textbf{lenguaje y el formato} de los materiales:
textos muy largos, vocabulario técnico, consignas confusas o poco
explícitas. La IA puede apoyar como \textbf{asistente de edición y de
borradores}, ayudando a simplificar, acortar, ampliar o reestructurar
materiales \textbf{sin cambiar el contenido disciplinar central}
definido por el equipo pedagógico.

Usada de forma crítica y situada, la IA permite generar
\textbf{versiones paralelas} de un mismo recurso (más breve, más
explicada, más visual, con apoyos adicionales, etc.), de modo que el
profesorado pueda elegir cuál se ajusta mejor a las características de
su curso, o bien ofrecer varias alternativas dentro de la misma clase,
en línea con principios de \textbf{Diseño Universal para el Aprendizaje
(DUA)}.

\textbf{{Usos posibles (ejemplos concretos)}.}

\textbf{Ajustes de lenguaje}

\begin{itemize}
\tightlist
\item
  Reescribir una guía que fue pensada para adultos en un lenguaje
  comprensible para estudiantes de enseñanza básica o media.\\
\item
  Simplificar el vocabulario de un texto, manteniendo las ideas
  principales y explicando palabras clave con ejemplos cotidianos.\\
\item
  Reformular consignas de actividades para que sean más claras, paso a
  paso y con verbos de acción explícitos.
\end{itemize}

\textbf{Ajustes de formato}

\begin{itemize}
\tightlist
\item
  Cambiar la presentación de un texto largo a un formato con subtítulos,
  viñetas o cuadros comparativos.\\
\item
  Transformar párrafos extensos en una \textbf{ficha de estudio} breve
  que destaque ideas claves, conceptos y ejemplos.\\
\item
  Crear un pequeño glosario de términos importantes para acompañar una
  lectura o una guía.
\end{itemize}

\textbf{Apoyos adicionales}

\begin{itemize}
\tightlist
\item
  Generar recordatorios de conceptos previos que el curso ya trabajó y
  que son necesarios para entender el nuevo contenido.\\
\item
  Redactar una versión ``para familias'' de una circular o explicación
  de proyecto, con lenguaje claro y cercano.\\
\item
  Proponer ejemplos y situaciones cercanas a la realidad de estudiantes
  en escuelas públicas, rurales o urbanas, según el contexto.
\end{itemize}

\textbf{Ejemplo de prompt para ajustar lenguaje, formato y extensión de
un material}

\begin{Shaded}
\begin{Highlighting}[]
\NormalTok{Actúa como profesor/a de }\CommentTok{[}\OtherTok{ASIGNATURA}\CommentTok{]}\NormalTok{ con experiencia en educación inclusiva y en DUA en escuelas públicas chilenas.}

\NormalTok{Tu objetivo es ayudarme a adaptar el siguiente texto para que sea más accesible a estudiantes de }\CommentTok{[}\OtherTok{NIVEL}\CommentTok{]}\NormalTok{ en el marco de las Bases Curriculares de Chile.}

\NormalTok{Contexto:}
\SpecialStringTok{{-} }\NormalTok{Curso diverso en niveles de lectura, con presencia de estudiantes del PIE.}
\SpecialStringTok{{-} }\NormalTok{El texto original es una explicación sobre }\CommentTok{[}\OtherTok{TEMA}\CommentTok{]}\NormalTok{ pensada para personas adultas.}
\SpecialStringTok{{-} }\NormalTok{Lo necesito para trabajarlo en una clase de aproximadamente }\CommentTok{[}\OtherTok{N°}\CommentTok{]}\NormalTok{ minutos.}
\SpecialStringTok{{-} }\NormalTok{Recursos disponibles: }\CommentTok{[}\OtherTok{por ejemplo, pizarra y cuadernos; no hay computadores para todo el curso}\CommentTok{]}\NormalTok{.}

\NormalTok{Instrucciones:}
\SpecialStringTok{1. }\NormalTok{Genera una primera versión del texto:}
\SpecialStringTok{   {-} }\NormalTok{Con lenguaje más sencillo, sin perder las ideas principales.}
\SpecialStringTok{   {-} }\NormalTok{De extensión máxima de }\CommentTok{[}\OtherTok{N°}\CommentTok{]}\NormalTok{ palabras.}
\SpecialStringTok{   {-} }\NormalTok{Organizada en párrafos breves y con viñetas cuando sea útil.}
\SpecialStringTok{2. }\NormalTok{Genera una segunda versión en formato de “ficha de estudio” que incluya:}
\SpecialStringTok{   {-} }\NormalTok{3–5 definiciones clave en lenguaje simple.}
\SpecialStringTok{   {-} }\NormalTok{2–3 ejemplos cotidianos cercanos a la realidad de estudiantes en escuelas chilenas.}
\SpecialStringTok{   {-} }\NormalTok{3 preguntas de autoevaluación para que las y los estudiantes revisen si entendieron.}
\SpecialStringTok{3. }\NormalTok{Si consideras que falta algún concepto básico para que se entienda el texto, indícalo al final en 3–4 líneas, sin agregarlo dentro del texto principal.}

\NormalTok{Formato de salida:}
\SpecialStringTok{{-} }\NormalTok{Título: “Versión simplificada para }\CommentTok{[}\OtherTok{NIVEL}\CommentTok{]}\NormalTok{” y luego el texto adaptado.}
\SpecialStringTok{{-} }\NormalTok{Título: “Ficha de apoyo” y luego la ficha de estudio con secciones claras.}

\NormalTok{Criterios y restricciones:}
\SpecialStringTok{{-} }\NormalTok{No cambies el contenido disciplinar central; solo ajusta lenguaje, formato y extensión.}
\SpecialStringTok{{-} }\NormalTok{No incluyas imágenes ni recursos que requieran conexión a internet.}
\SpecialStringTok{{-} }\NormalTok{Usa ejemplos y situaciones cercanas a la realidad de estudiantes en escuelas públicas chilenas.}
\SpecialStringTok{{-} }\NormalTok{No incluyas nombres propios ni datos personales de estudiantes ni familias.}
\end{Highlighting}
\end{Shaded}

\begin{tcolorbox}[enhanced jigsaw, colframe=quarto-callout-note-color-frame, opacityback=0, coltitle=black, colbacktitle=quarto-callout-note-color!10!white, titlerule=0mm, left=2mm, breakable, leftrule=.75mm, bottomtitle=1mm, opacitybacktitle=0.6, toptitle=1mm, bottomrule=.15mm, toprule=.15mm, arc=.35mm, rightrule=.15mm, title=\textcolor{quarto-callout-note-color}{\faInfo}\hspace{0.5em}{Nota}, colback=white]

\textbf{Idea fuerza para el profesorado:}\\
La IA puede ayudar a ajustar lenguaje, formato y extensión de los
materiales para abrir puertas a más estudiantes, pero el sentido
pedagógico de esos materiales y su coherencia con el currículum siempre
lo define el equipo docente.

\end{tcolorbox}

\section{Ejemplos de adaptaciones para distintas trayectorias y
necesidades}\label{ejemplos-de-adaptaciones-para-distintas-trayectorias-y-necesidades}

En un mismo curso conviven estudiantes con trayectorias muy diversas:
quienes han cambiado de escuela con frecuencia, quienes participan en
programas de integración escolar (PIE), quienes tienen responsabilidades
de cuidado o trabajo, quienes se están incorporando recientemente al
sistema escolar chileno, entre muchas otras realidades. La IA puede
ayudar a pensar variantes de actividades y materiales que dialoguen con
esta diversidad, siempre bajo el criterio profesional docente y
resguardando la confidencialidad de información sensible.

El propósito no es etiquetar estudiantes (``los buenos'', ``los malos'',
``los que saben, los que no''), sino eliminar barreras y contar con un
abanico de opciones que permita ajustar apoyos y desafíos sin perder de
vista el currículum común, la pertenencia al grupo curso y los
principios del Diseño Universal para el Aprendizaje (DUA).

\textbf{{Usos posibles (ejemplos concretos)}.}

\textbf{Apoyos para trayectorias diversas}

\begin{itemize}
\tightlist
\item
  Proponer actividades alternativas para estudiantes con ausencias
  prolongadas o reincorporaciones tardías (por ejemplo, una versión de
  la tarea que permita retomar lo esencial sin repetir toda la
  unidad).\\
\item
  Sugerir apoyos adicionales para quienes se están adaptando al idioma o
  al sistema escolar chileno (glosarios, ejemplos contextualizados,
  consignas más guiadas).\\
\item
  Ajustar tareas para estudiantes que viven situaciones de cuidado o
  trabajo, priorizando lo esencial del objetivo de aprendizaje con
  tiempos y formatos más flexibles.
\end{itemize}

\textbf{Variantes de una misma actividad}

\begin{itemize}
\tightlist
\item
  Diseñar una versión de la actividad que se pueda realizar tanto en el
  aula como en el hogar, con o sin acceso a tecnologías.\\
\item
  Proponer opciones de producto final (texto breve, esquema, audio,
  afiche) para que estudiantes puedan demostrar lo aprendido de
  distintas maneras.\\
\item
  Generar ejemplos y contextos vinculados a distintas realidades
  territoriales y culturales (rural/urbano, norte/centro/sur, etc.).
\end{itemize}

\textbf{Articulación con PIE y DUA}

\begin{itemize}
\tightlist
\item
  Sugerir ideas que complementen los planes individualizados ya
  acordados por el PIE (por ejemplo, actividades de repaso o apoyo
  visual).\\
\item
  Proponer ejemplos de andamiajes (preguntas guía, pasos numerados,
  modelos resueltos) que luego el equipo revise y adapte a cada caso.
\end{itemize}

\textbf{Ejemplo de prompt para generar adaptaciones según trayectorias
diversas}

\begin{Shaded}
\begin{Highlighting}[]
\NormalTok{Actúa como especialista en educación inclusiva y en Diseño Universal para el Aprendizaje (DUA) en contexto escolar chileno.}

\NormalTok{Tu objetivo es proponer adaptaciones de una actividad de }\CommentTok{[}\OtherTok{ASIGNATURA}\CommentTok{]}\NormalTok{ sobre }\CommentTok{[}\OtherTok{TEMA}\CommentTok{]}\NormalTok{ para estudiantes con trayectorias y necesidades diversas, manteniendo el mismo objetivo de aprendizaje.}

\NormalTok{Contexto:}
\SpecialStringTok{{-} }\NormalTok{Curso: }\CommentTok{[}\OtherTok{NIVEL, por ejemplo “8º básico”}\CommentTok{]}\NormalTok{ en escuela }\CommentTok{[}\OtherTok{pública/subvencionada/particular}\CommentTok{]}\NormalTok{ con cursos numerosos.}
\SpecialStringTok{{-} }\NormalTok{En el grupo hay:}
\SpecialStringTok{  {-} }\NormalTok{Estudiantes que han tenido ausencias prolongadas.}
\SpecialStringTok{  {-} }\NormalTok{Estudiantes recién incorporados al sistema escolar chileno.}
\SpecialStringTok{  {-} }\NormalTok{Estudiantes que participan en el PIE.}
\SpecialStringTok{{-} }\NormalTok{Recursos disponibles: }\CommentTok{[}\OtherTok{pizarra, cuadernos, proyector, impresora, etc.}\CommentTok{]}\NormalTok{.}
\SpecialStringTok{{-} }\NormalTok{Ya existe un trabajo previo del equipo PIE y acuerdos de curso que debemos respetar.}

\NormalTok{Actividad original:}
\CommentTok{[}\OtherTok{PEGAR AQUÍ LA DESCRIPCIÓN DE LA ACTIVIDAD DE CLASE QUE YA DISEÑÓ EL/LA DOCENTE, SIN NOMBRES NI DATOS SENSIBLES}\CommentTok{]}\NormalTok{.}

\NormalTok{Instrucciones:}
\SpecialStringTok{1. }\NormalTok{Propón al menos 3 adaptaciones posibles de la actividad original:}
\SpecialStringTok{   {-} }\NormalTok{Adaptación A: pensada para estudiantes con ausencias prolongadas o reincorporaciones tardías.}
\SpecialStringTok{   {-} }\NormalTok{Adaptación B: pensada para estudiantes recién incorporados al sistema escolar chileno o con nivel de lectura diferente.}
\SpecialStringTok{   {-} }\NormalTok{Adaptación C: pensada para estudiantes que requieren mayor apoyo estructurado (por ejemplo, acompañados por PIE).}
\SpecialStringTok{2. }\NormalTok{Para cada adaptación, indica:}
\SpecialStringTok{   {-} }\NormalTok{Propósito.}
\SpecialStringTok{   {-} }\NormalTok{Pasos principales de la actividad.}
\SpecialStringTok{   {-} }\NormalTok{Tipo de apoyo que se ofrece (andamiajes, ejemplos, materiales de apoyo, consignas más guiadas, etc.).}
\SpecialStringTok{   {-} }\NormalTok{Cómo se vincula explícitamente con el mismo objetivo de aprendizaje de la actividad original.}
\SpecialStringTok{3. }\NormalTok{Incluye una breve sugerencia sobre cómo presentar estas opciones al curso sin etiquetar ni separar a las y los estudiantes.}

\NormalTok{Formato de salida:}
\SpecialStringTok{{-} }\NormalTok{Subtítulos: “Adaptación A”, “Adaptación B” y “Adaptación C”.}
\SpecialStringTok{{-} }\NormalTok{Bajo cada subtítulo, un párrafo de 8–10 líneas describiendo la propuesta.}

\NormalTok{Criterios y restricciones:}
\SpecialStringTok{{-} }\NormalTok{No cambies el objetivo de aprendizaje; solo ajusta la forma de abordarlo.}
\SpecialStringTok{{-} }\NormalTok{No incluyas datos personales ni ejemplos que permitan identificar estudiantes específicos.}
\SpecialStringTok{{-} }\NormalTok{Usa lenguaje claro y respetuoso, evitando etiquetas como “buenos/malos alumnos”.}
\SpecialStringTok{{-} }\NormalTok{Formula las propuestas de manera que puedan ser útiles para todo el grupo, no solo para un estudiante en particular.}
\end{Highlighting}
\end{Shaded}

\begin{tcolorbox}[enhanced jigsaw, colframe=quarto-callout-note-color-frame, opacityback=0, coltitle=black, colbacktitle=quarto-callout-note-color!10!white, titlerule=0mm, left=2mm, breakable, leftrule=.75mm, bottomtitle=1mm, opacitybacktitle=0.6, toptitle=1mm, bottomrule=.15mm, toprule=.15mm, arc=.35mm, rightrule=.15mm, title=\textcolor{quarto-callout-note-color}{\faInfo}\hspace{0.5em}{Nota}, colback=white]

\textbf{Idea fuerza para el profesorado:}\\
Las adaptaciones apoyadas por IA deben ampliar las posibilidades de
participación y conexión con el currículum común, sin separar ni
etiquetar. La herramienta sugiere opciones; el equipo docente y el PIE
deciden qué tiene sentido en su curso.

\end{tcolorbox}

\section{Posibilidades y límites en contextos de integración y educación
especial}\label{posibilidades-y-luxedmites-en-contextos-de-integraciuxf3n-y-educaciuxf3n-especial}

En contextos de integración y educación especial, la IA abre
oportunidades relevantes: permite generar materiales con distintos
niveles de apoyo, crear ejemplos personalizados, proponer formatos
alternativos (texto, viñetas, esquemas), entre otros. Sin embargo,
también existen límites importantes: la IA no conoce a las y los
estudiantes, no reemplaza la evaluación diagnóstica ni las decisiones
del equipo PIE, y puede reproducir sesgos o proponer estrategias poco
adecuadas si no se la orienta y revisa críticamente.

Por eso es clave usar estas herramientas como soporte para pensar
estrategias, nunca como sustituto de la observación profesional, de los
acuerdos con las familias o de los planes individualizados (PII,
adecuaciones curriculares, informes del PIE). Además, es fundamental no
ingresar datos sensibles en las herramientas de IA (RUT, nombres,
diagnósticos, antecedentes de salud, situaciones familiares específicas,
etc.).

\textbf{{Usos posibles (ejemplos concretos)}.}

\textbf{Apoyos generales para todo el curso}

\begin{itemize}
\tightlist
\item
  Sugerir ideas de adaptaciones curriculares no significativas que
  puedan beneficiar al grupo completo (por ejemplo, consignas más
  claras, uso de organizadores gráficos, actividades por estaciones).\\
\item
  Generar materiales complementarios como glosarios, tarjetas visuales,
  pasos de una rutina o ejemplos resueltos que apoyen la comprensión de
  todo el curso.\\
\item
  Proponer formas alternativas de demostrar aprendizaje (productos
  orales, visuales, manipulativos) para diversificar las evidencias sin
  modificar los objetivos de aprendizaje.
\end{itemize}

\textbf{Trabajo con equipo PIE y familias}

\begin{itemize}
\tightlist
\item
  Sistematizar acuerdos del equipo PIE en un lenguaje claro para
  compartir con el resto del profesorado (borrador de orientaciones
  internas que luego el equipo revisa).\\
\item
  Redactar borradores de orientaciones generales para familias sobre
  cómo apoyar tareas en casa, que después el equipo ajuste a su
  realidad.\\
\item
  Ordenar información dispersa (notas de reuniones, acuerdos, ideas) en
  esquemas que faciliten la coordinación entre docentes, asistentes de
  la educación y profesionales del PIE.
\end{itemize}

\textbf{Límites y cuidados}

\begin{itemize}
\tightlist
\item
  Recordar que la IA no puede definir adecuaciones curriculares
  significativas ni planes individuales por sí sola.\\
\item
  Evitar pedir a la IA diagnósticos, etiquetas o decisiones individuales
  sobre estudiantes.\\
\item
  Revisar críticamente las sugerencias, ya que la IA puede reproducir
  estereotipos o suposiciones sobre discapacidad, lengua, género, etc.
\end{itemize}

\textbf{Ejemplo de prompt para explorar posibilidades y límites en un
curso con PIE}

\begin{Shaded}
\begin{Highlighting}[]
\NormalTok{Actúa como profesional de apoyo en un Programa de Integración Escolar (PIE) con experiencia en adecuaciones curriculares no significativas y en DUA.}

\NormalTok{Tu objetivo es sugerir ideas de apoyo pedagógico para un curso de }\CommentTok{[}\OtherTok{NIVEL}\CommentTok{]}\NormalTok{ que trabaja el contenido }\CommentTok{[}\OtherTok{TEMA}\CommentTok{]}\NormalTok{ en la asignatura de }\CommentTok{[}\OtherTok{ASIGNATURA}\CommentTok{]}\NormalTok{, respetando el currículum común y el trabajo del equipo PIE.}

\NormalTok{Contexto:}
\SpecialStringTok{{-} }\NormalTok{Escuela pública con PIE y cursos numerosos.}
\SpecialStringTok{{-} }\NormalTok{En el curso hay estudiantes con:}
\SpecialStringTok{  {-} }\NormalTok{Dificultades específicas de aprendizaje.}
\SpecialStringTok{  {-} }\NormalTok{TDAH.}
\SpecialStringTok{  {-} }\NormalTok{Discapacidad intelectual leve.}
\SpecialStringTok{{-} }\NormalTok{El equipo docente ya definió los objetivos de aprendizaje para todo el curso.}
\SpecialStringTok{{-} }\NormalTok{Ya existen planes y acuerdos del PIE que se deben respetar (no los detallaré aquí por confidencialidad).}

\NormalTok{Instrucciones:}
\SpecialStringTok{1. }\NormalTok{Propón 5 ideas de apoyos generales que puedan beneficiar a todo el curso (no solo a estudiantes del PIE), tales como:}
\SpecialStringTok{   {-} }\NormalTok{Ajustes en la presentación de la información.}
\SpecialStringTok{   {-} }\NormalTok{Estrategias de trabajo colaborativo.}
\SpecialStringTok{   {-} }\NormalTok{Formas alternativas de demostrar lo aprendido (oral, escrito, gráfico, manipulativo).}
\SpecialStringTok{2. }\NormalTok{Luego, sugiere 3 ejemplos de adecuaciones curriculares no significativas para el mismo contenido, aclarando:}
\SpecialStringTok{   {-} }\NormalTok{Qué se mantiene igual para todos (objetivo de aprendizaje, criterio de logro).}
\SpecialStringTok{   {-} }\NormalTok{Qué se adapta (tiempo, formato, cantidad de ejercicios, grado de apoyo, etc.).}
\SpecialStringTok{3. }\NormalTok{Finalmente, incluye una breve sección “Límites y cuidados” con 4–5 puntos que recuerden:}
\SpecialStringTok{   {-} }\NormalTok{Que las decisiones deben ser revisadas y validadas por el equipo PIE y el profesorado del curso.}
\SpecialStringTok{   {-} }\NormalTok{Que no se debe incluir información sensible de estudiantes en herramientas de IA.}
\SpecialStringTok{   {-} }\NormalTok{Que la IA no reemplaza la evaluación diagnóstica ni la comunicación con las familias.}

\NormalTok{Formato de salida:}
\SpecialStringTok{{-} }\NormalTok{Sección con subtítulo “Apoyos generales para todo el curso”.}
\SpecialStringTok{{-} }\NormalTok{Sección con subtítulo “Ejemplos de adecuaciones no significativas”.}
\SpecialStringTok{{-} }\NormalTok{Sección con subtítulo “Límites y cuidados”.}

\NormalTok{Criterios y restricciones:}
\SpecialStringTok{{-} }\NormalTok{No sugieras diagnósticos ni decisiones individuales sobre estudiantes.}
\SpecialStringTok{{-} }\NormalTok{Usa lenguaje respetuoso y centrado en apoyos, no en déficits.}
\SpecialStringTok{{-} }\NormalTok{Enfatiza que la IA es un apoyo para pensar estrategias pedagógicas, no un reemplazo de la evaluación profesional ni de los acuerdos del equipo PIE.}
\SpecialStringTok{{-} }\NormalTok{No incluyas nombres propios, RUT ni ningún dato que permita identificar a estudiantes o familias.}
\end{Highlighting}
\end{Shaded}

\begin{tcolorbox}[enhanced jigsaw, colframe=quarto-callout-note-color-frame, opacityback=0, coltitle=black, colbacktitle=quarto-callout-note-color!10!white, titlerule=0mm, left=2mm, breakable, leftrule=.75mm, bottomtitle=1mm, opacitybacktitle=0.6, toptitle=1mm, bottomrule=.15mm, toprule=.15mm, arc=.35mm, rightrule=.15mm, title=\textcolor{quarto-callout-note-color}{\faInfo}\hspace{0.5em}{Nota}, colback=white]

\textbf{Idea fuerza para el profesorado:}\\
En contextos de integración y educación especial, la IA puede ser un
buen apoyo para imaginar estrategias y apoyos generales, pero las
decisiones sobre adecuaciones, planes individuales y comunicación con
familias siguen estando en manos del equipo profesional y de la
comunidad escolar. La herramienta aporta ideas; el criterio pedagógico y
ético lo ponen las personas.

\end{tcolorbox}

\bookmarksetup{startatroot}

\chapter{Privacidad, sesgos y brechas de
acceso}\label{privacidad-sesgos-y-brechas-de-acceso}

\section{Privacidad, sesgos y brechas de
acceso}\label{privacidad-sesgos-y-brechas-de-acceso-1}

El uso de herramientas de inteligencia artificial en la escuela implica
trabajar con datos, decisiones y recursos tecnológicos que no son
neutros. Antes de pensar en ``todo lo que la IA puede hacer'', es clave
preguntarse qué riesgos abre, quién controla estas tecnologías y cómo
podemos minimizarlos desde la responsabilidad profesional docente y el
enfoque de derechos.

\subsection{Privacidad y protección de
datos}\label{privacidad-y-protecciuxf3n-de-datos}

En términos de privacidad, cualquier información que permita identificar
a una persona (nombre, RUT, curso específico, diagnóstico, dirección,
teléfono, correo, antecedentes familiares, etc.) no debiera ser
ingresada en herramientas de IA abiertas. Tampoco es recomendable copiar
actas sensibles, listados completos de notas, informes psicológicos u
otros documentos similares.

Cuando se requiera trabajar con ejemplos, es preferible anonimizar los
datos (``estudiante A'', ``curso de 6º básico'', ``apoderado'') o usar
situaciones ficticias. La regla práctica es: si no lo pegarías en una
red social, no lo pegues en una IA abierta.

\subsection{Sesgos, narrativas y poder}\label{sesgos-narrativas-y-poder}

Los modelos de IA aprenden a partir de grandes volúmenes de texto e
imagen producidos en contextos desiguales. Eso significa que pueden
reproducir y amplificar sesgos de género, raza, clase, territorio,
discapacidad, orientación sexual, religión, entre otros.

En los ejemplos que mostramos en el taller se observa:

\begin{itemize}
\tightlist
\item
  Un reportaje sobre el uso de ChatGPT como herramienta de diplomacia
  pública, donde Isarel invierte grandes sumas para influir en cómo los
  sistemas de IA responden a preguntas sobre un conflicto político. Esto
  muestra que la IA puede ser usada para modelar narrativas oficiales y
  reforzar ciertos puntos de vista sobre temas sensibles (por ejemplo,
  guerras, ocupaciones, violaciones a derechos humanos), invisibilizando
  otros.
\end{itemize}

\pandocbounded{\includegraphics[keepaspectratio]{assets/no.png}}

\begin{itemize}
\tightlist
\item
  Un conjunto de materiales sobre racismo algorítmico y
  tecnopatriarcado: algoritmos de contratación que favorecen a hombres,
  sistemas de moderación de contenidos que censuran cuerpos femeninos, o
  tecnologías de reconocimiento facial usadas para control policial y
  estatal. La idea de tecnopatriarcado subraya que la industria
  tecnológica se desarrolla en un mundo atravesado por relaciones de
  poder patriarcales, racistas y coloniales; si no se interviene
  críticamente, la IA tiende a reforzar esas mismas estructuras.
\end{itemize}

\pandocbounded{\includegraphics[keepaspectratio]{assets/no1.png}}

Para la escuela esto implica que, al trabajar con IA, no solo debemos
fijarnos en ``si la respuesta está bien escrita'', sino también en desde
qué mirada está hablando la herramienta:

\begin{itemize}
\tightlist
\item
  ¿A quién muestra como protagonista y a quién invisibiliza?
\item
  ¿Cómo nombra a ciertos grupos (migrantes, pueblos originarios,
  mujeres, personas LGBTIQ+, personas con discapacidad)?
\item
  ¿Qué conflictos aparecen como ``controvertidos'', ``complejos'' o
  directamente silenciados?
\end{itemize}

\subsection{Brechas de acceso y uso}\label{brechas-de-acceso-y-uso}

El uso de IA también exige conectividad, dispositivos y competencias
digitales, lo que puede profundizar las brechas entre establecimientos
(públicos/particulares, rurales/urbanos) y dentro de un mismo curso. No
es lo mismo diseñar actividades con IA en un liceo con laboratorio
disponible todos los días que en una escuela rural donde solo hay un
proyector y conectividad inestable.

\subsection{Orientaciones básicas para la práctica
escolar}\label{orientaciones-buxe1sicas-para-la-pruxe1ctica-escolar}

\begin{itemize}
\tightlist
\item
  Evitar ingresar datos personales identificables de estudiantes,
  familias o colegas.
\item
  Trabajar con ejemplos anonimizados (``estudiante A'', ``apoderado'',
  ``curso de 6º básico'') o con situaciones ficticias.
\item
  No copiar en la IA diagnósticos, informes psicológicos, antecedentes
  de salud ni situaciones familiares específicas.
\item
  Revisar críticamente los ejemplos y respuestas de la IA para detectar
  estereotipos o sesgos (racistas, sexistas, colonialistas,
  capacitistas, etc.), discutiéndolos con el curso cuando sea
  pertinente.
\item
  Triangular siempre la información sobre temas controversiales
  (conflictos armados, migración, seguridad, etc.) con múltiples fuentes
  y no solo con la respuesta de una IA.
\item
  No tomar decisiones disciplinarias, diagnósticas o de atención
  individual basadas únicamente en lo que sugiere una IA.
\item
  Considerar las desigualdades de acceso a dispositivos y conectividad
  al diseñar actividades que involucren tecnología, buscando opciones
  que no excluyan a quienes tienen menos recursos.
\end{itemize}

\begin{tcolorbox}[enhanced jigsaw, colframe=quarto-callout-note-color-frame, opacityback=0, coltitle=black, colbacktitle=quarto-callout-note-color!10!white, titlerule=0mm, left=2mm, breakable, leftrule=.75mm, bottomtitle=1mm, opacitybacktitle=0.6, toptitle=1mm, bottomrule=.15mm, toprule=.15mm, arc=.35mm, rightrule=.15mm, title=\textcolor{quarto-callout-note-color}{\faInfo}\hspace{0.5em}{Nota}, colback=white]

\textbf{Idea fuerza para docentes:} la IA no es neutra ni objetiva.
Puede ayudarnos, pero trabaja con datos marcados por desigualdades y
relaciones de poder. Nuestra tarea profesional es cuidar la privacidad,
detectar sesgos y tomar decisiones que no profundicen las brechas
educativas.

\end{tcolorbox}

\section{Criterios pedagógicos para decidir cuándo usar
IA}\label{criterios-pedaguxf3gicos-para-decidir-cuuxe1ndo-usar-ia}

No toda tarea pedagógica necesita o se beneficia del uso de IA. Una
decisión responsable considera cuándo tiene sentido apoyarse en estas
herramientas y cuándo es mejor prescindir de ellas. La IA puede ser
especialmente útil para tareas repetitivas, de borrador o de exploración
de ideas; en cambio, las decisiones sobre el sentido pedagógico, la
evaluación fina y el acompañamiento socioemocional siguen siendo de
la/el docente y de los equipos escolares.

\subsection{Criterios posibles para
decidir}\label{criterios-posibles-para-decidir}

\textbf{Pertinencia pedagógica}\\
Preguntarse si la IA aporta algo que no podría hacerse razonablemente
con otros recursos, o si solo se está usando ``porque está de moda''.
Por ejemplo, puede ser pertinente para generar borradores de
actividades, pero no para definir el informe final de personalidad de un
estudiante.

\textbf{Ahorro de tiempo sin pérdida de calidad}\\
Priorizar la IA en tareas de redacción inicial (borradores de guías,
propuestas de actividades, ejemplos) que luego se revisan y ajustan, no
en la toma de decisiones evaluativas finales ni en informes sensibles.

\textbf{Control docente del proceso}\\
Asegurarse de que la/el profesor/a mantenga el control sobre los
objetivos, los criterios de evaluación y las decisiones de cierre,
incluso cuando use IA para generar insumos. La herramienta propone; el
equipo docente decide.

\textbf{Equidad y acceso}\\
Evaluar si el uso de IA generará nuevas desigualdades dentro del curso o
entre cursos (por ejemplo, si solo algunos estudiantes pueden usar
dispositivos en casa). Si se pide una tarea con IA, ofrecer siempre
alternativas sin IA que permitan demostrar los mismos aprendizajes.

\textbf{Transparencia y explicabilidad}\\
Evitar decisiones importantes que no se puedan justificar frente a
estudiantes y familias más allá de ``lo dijo la IA''. Si una decisión se
apoya en un insumo generado por IA, debe poder explicarse con criterios
pedagógicos y normativos claros.

\textbf{Cuidado con temas sensibles y controversiales}\\
En temas como racismo, género, territorio, violencia, conflictos
internacionales o política, la IA puede reproducir visiones sesgadas. Es
preferible usarla, si acaso, para explorar ejemplos que luego se
analizan críticamente, y no como fuente principal de verdad.

\begin{tcolorbox}[enhanced jigsaw, colframe=quarto-callout-note-color-frame, opacityback=0, coltitle=black, colbacktitle=quarto-callout-note-color!10!white, titlerule=0mm, left=2mm, breakable, leftrule=.75mm, bottomtitle=1mm, opacitybacktitle=0.6, toptitle=1mm, bottomrule=.15mm, toprule=.15mm, arc=.35mm, rightrule=.15mm, title=\textcolor{quarto-callout-note-color}{\faInfo}\hspace{0.5em}{Nota}, colback=white]

\textbf{Idea fuerza para docentes:} la IA es útil cuando reduce carga de
trabajo sin sacrificar criterios pedagógicos ni equidad. Si complica el
proceso, aumenta las brechas o pone en riesgo la confianza de
estudiantes y familias, es mejor no usarla.

\end{tcolorbox}

\section{Cómo conversar con estudiantes y familias sobre el uso
responsable}\label{cuxf3mo-conversar-con-estudiantes-y-familias-sobre-el-uso-responsable}

La presencia de la IA en la vida cotidiana de niñas, niños y jóvenes
hace necesario abordar el tema abiertamente en la escuela. Más que
prohibir o celebrar sin matices, se trata de generar conversaciones que
permitan desarrollar criterios, pensar riesgos y oportunidades, y
acordar normas de uso responsable.

Con estudiantes, estas conversaciones pueden articularse con formación
ciudadana, orientación, ética, filosofía, lenguaje, historia u otras
asignaturas. Con familias, es importante transmitir información clara y
sencilla, evitando tecnicismos, para que puedan acompañar a sus hijos e
hijas en el uso de estas herramientas.

A partir de las noticias e imágenes que mostramos en el taller, se
pueden trabajar preguntas como:

\begin{itemize}
\tightlist
\item
  ¿Qué significa que un gobierno invierta millones para influir en cómo
  responde la IA sobre un conflicto político?
\item
  ¿Por qué organismos como Naciones Unidas hablan de racismo
  algorítmico?
\item
  ¿Qué nos quiere decir el concepto de tecnopatriarcado sobre quién
  diseña estas tecnologías y a quién benefician o perjudican?
\end{itemize}

\subsection{Ideas clave para trabajar en aula y con la
comunidad}\label{ideas-clave-para-trabajar-en-aula-y-con-la-comunidad}

\begin{itemize}
\tightlist
\item
  Explicar en lenguaje simple qué es y qué no es la IA (no es
  ``inteligencia humana'', no ``sabe todo'', puede equivocarse y
  reproducir injusticias).
\item
  Conversar sobre la importancia de no compartir datos personales y de
  preguntar siempre qué se hace con la información.
\item
  Trabajar ejemplos de sesgos y errores de la IA, analizando cómo
  afectan a distintos grupos y qué relaciones de poder se reflejan en
  esos ejemplos.
\item
  Discutir los límites del uso de IA en tareas escolares: qué se
  considera apoyo legítimo (por ejemplo, pedir ideas o ejemplos) y qué
  se considera copia o falta a la honestidad académica (entregar un
  trabajo hecho íntegramente por la IA).
\item
  Construir junto al curso acuerdos de uso responsable (por ejemplo,
  ``no pegamos datos personales'', ``si usamos IA citamos que lo
  hicimos'', ``si algo nos parece raro, lo verificamos''), que luego
  puedan compartirse con las familias.
\item
  Invitar a las familias a compartir sus dudas y experiencias,
  enfatizando que el objetivo de la escuela es acompañar y orientar, no
  vigilar el uso doméstico de tecnología.
\end{itemize}

\begin{tcolorbox}[enhanced jigsaw, colframe=quarto-callout-note-color-frame, opacityback=0, coltitle=black, colbacktitle=quarto-callout-note-color!10!white, titlerule=0mm, left=2mm, breakable, leftrule=.75mm, bottomtitle=1mm, opacitybacktitle=0.6, toptitle=1mm, bottomrule=.15mm, toprule=.15mm, arc=.35mm, rightrule=.15mm, title=\textcolor{quarto-callout-note-color}{\faInfo}\hspace{0.5em}{Nota}, colback=white]

\textbf{Idea fuerza para docentes:} hablar de IA con estudiantes y
familias es una oportunidad para formar criterio, cuidado y ciudadanía
digital. La herramienta cambia rápido, pero los principios de respeto,
honestidad y justicia siguen siendo los mismos: la IA debe estar al
servicio de esos principios, no al revés.

\end{tcolorbox}

\bookmarksetup{startatroot}

\chapter{Buenas prácticas y pasos pequeños para
empezar}\label{buenas-pruxe1cticas-y-pasos-pequeuxf1os-para-empezar}

\section{Herramientas de IA: mapa visual para elegir rápido (versión más
``lectiva'')}\label{herramientas-de-ia-mapa-visual-para-elegir-ruxe1pido-versiuxf3n-muxe1s-lectiva}

En el trabajo docente, la IA funciona mejor si la pensamos como un
\textbf{conjunto de herramientas} con fortalezas distintas. No existe
``la mejor IA para todo'': lo que cambia es \textbf{la tarea}
(planificar, redactar, sintetizar textos, crear material visual, generar
una cápsula) y el \textbf{contexto real} (tiempo, curso, recursos
disponibles, diversidad del aula).

\textbf{Regla de oro:} usa la IA como \textbf{asistente de borradores}.
La herramienta propone; \textbf{el equipo docente decide}, ajusta y
valida con el currículum, la experiencia pedagógica y el conocimiento
del curso.

\section{Herramientas de IA: ¿cuál usar para qué? (mapa
rápido)}\label{herramientas-de-ia-cuuxe1l-usar-para-quuxe9-mapa-ruxe1pido}

En el trabajo docente conviene pensar la IA como un \textbf{set de
herramientas} con fortalezas distintas. La clave es elegir la
herramienta según la tarea (planificación, síntesis, creación de
materiales) y usarla como \textbf{asistente de borradores}: tú y tu
equipo \textbf{deciden, ajustan y validan} con Bases Curriculares,
acuerdos del establecimiento y el contexto real del curso.

\begin{tcolorbox}[enhanced jigsaw, colframe=quarto-callout-tip-color-frame, opacityback=0, coltitle=black, colbacktitle=quarto-callout-tip-color!10!white, titlerule=0mm, left=2mm, breakable, leftrule=.75mm, bottomtitle=1mm, opacitybacktitle=0.6, toptitle=1mm, bottomrule=.15mm, toprule=.15mm, arc=.35mm, rightrule=.15mm, title=\textcolor{quarto-callout-tip-color}{\faLightbulb}\hspace{0.5em}{Sugerencia práctica (muy útil para ahorrar tiempo)}, colback=white]

Si ya tienen \textbf{pautas, guías o actividades previas} (sin datos
sensibles), pueden pegarlas o subirlas como referencia (anonimizadas)
para que la IA genere una versión \textbf{similar en estilo y
exigencia}, ajustada al curso y al tiempo disponible.

\end{tcolorbox}

\begin{tcolorbox}[enhanced jigsaw, colframe=quarto-callout-note-color-frame, opacityback=0, coltitle=black, colbacktitle=quarto-callout-note-color!10!white, titlerule=0mm, left=2mm, breakable, leftrule=.75mm, bottomtitle=1mm, opacitybacktitle=0.6, toptitle=1mm, bottomrule=.15mm, toprule=.15mm, arc=.35mm, rightrule=.15mm, title=\textcolor{quarto-callout-note-color}{\faInfo}\hspace{0.5em}{Recomendación para el uso cotidiano: trabaja con un set pequeño}, colback=white]

Para crear un \textbf{hábito de trabajo} (y no saturarse), conviene
elegir un set pequeño de 3--4 herramientas y usarlas de manera
consistente. Eso ayuda a construir una ``memoria de trabajo'' del equipo
(formatos, prompts, plantillas, criterios) y mejora la calidad de los
resultados.

\end{tcolorbox}

\section{Galería: ¿qué herramienta usar para
qué?}\label{galeruxeda-quuxe9-herramienta-usar-para-quuxe9}

\subsection{ChatGPT (texto y
borradores)}\label{chatgpt-texto-y-borradores}

\includegraphics[width=5.20833in,height=\textheight,keepaspectratio]{assets/chat.png}

\textbf{Sirve para:} escribir, ordenar y dar formato a
\textbf{planificaciones}, \textbf{actividades}, \textbf{rúbricas},
\textbf{comunicados}, \textbf{guías}, \textbf{variantes DUA}, ejemplos y
bancos de preguntas.

\textbf{Úsalo cuando:} - necesitas \textbf{redacción guiada} (listas,
tablas, pasos); - quieres iterar rápido (``mejorarlo'', ``hacerlo más
simple'', ``crear 3 variantes'').

\textbf{Evítalo para:} - decisiones finales de evaluación, -
diagnósticos o decisiones individuales, - casos con \textbf{datos
sensibles}.

\subsection{Gemini (imagen y video, según
disponibilidad)}\label{gemini-imagen-y-video-seguxfan-disponibilidad}

\includegraphics[width=5.20833in,height=\textheight,keepaspectratio]{assets/gemini.png}

\textbf{Sirve para:} crear \textbf{material visual} (afiches simples,
íconos, diagramas, escenas genéricas) y, si está disponible,
\textbf{video breve} tipo cápsula (apertura de unidad, motivación,
explicación general).

\textbf{Úsalo cuando:} - quieres apoyar una clase con un recurso
\textbf{más visual} (sin depender de mucho texto).

\textbf{Evítalo para:} - rostros reales de estudiantes, - referencias
identificables del colegio, - situaciones sensibles o estereotipos.

\subsection{NotebookLM (lectura y síntesis con
fuentes)}\label{notebooklm-lectura-y-suxedntesis-con-fuentes}

\includegraphics[width=5.20833in,height=\textheight,keepaspectratio]{assets/notebooklm.png}

\textbf{Sirve para:} entender rápido documentos largos que tú subes
(orientaciones, reglamentos, planes, artículos, actas). Ayuda a sacar
\textbf{ideas clave}, comparar documentos y redactar minutas o FAQ.

\textbf{Úsalo cuando:} - necesitas \textbf{síntesis con trazabilidad}
(basada en tus textos), - no quieres que la herramienta ``invente''
fuera de la fuente.

\textbf{Evítalo para:} - documentos con información sensible (informes
individuales, datos personales, antecedentes de salud).

\subsection{Otras herramientas útiles
(sugerencias)}\label{otras-herramientas-uxfatiles-sugerencias}

Estas herramientas no son ``obligatorias''. Son opciones que pueden
sumar si el establecimiento ya las usa o si están disponibles. La idea
es \textbf{probar con pocas}, quedarse con las más útiles y sostener un
flujo estable.

\subsubsection{Google Forms + IA (aplicación simple de evaluación
formativa)}\label{google-forms-ia-aplicaciuxf3n-simple-de-evaluaciuxf3n-formativa}

\textbf{Sirve para:} aplicar tickets de salida, quizzes y encuestas
breves, con resultados ordenados para retroalimentar.

\textbf{Úsalo cuando:} - necesitas una aplicación rápida y práctica, -
quieres evidencia simple para ajustar la enseñanza.

\textbf{Evítalo para:} - recolectar datos personales innecesarios, -
``automatizar'' decisiones de calificación sin revisión docente.

\subsubsection{Presentaciones y material de clase (rápido y
ordenado)}\label{presentaciones-y-material-de-clase-ruxe1pido-y-ordenado}

\begin{itemize}
\item
  \textbf{Canva (Magic Write / Presentations / diseño)}\\
  Para transformar una idea en \textbf{láminas, afiches y guías
  visuales} sin ``pelear'' con el formato.\\
  \textbf{Útil cuando:} necesitas algo \textbf{presentable} en poco
  tiempo.\\
  \textbf{Cuidado:} no subas fotos de estudiantes ni datos del curso.
\item
  \textbf{Microsoft Copilot (Word/PowerPoint)}\\
  Bueno si el colegio ya trabaja con \textbf{Office}; ayuda a generar
  borradores y dar formato.\\
  \textbf{Útil cuando:} tu flujo ya está en Word/PowerPoint.
\end{itemize}

\subsubsection{Evaluación formativa y bancos de
preguntas}\label{evaluaciuxf3n-formativa-y-bancos-de-preguntas}

\begin{itemize}
\item
  \textbf{QuestionWell (u otras herramientas de ``question
  generation'')}\\
  Genera preguntas, tickets de salida y quizzes por nivel.\\
  \textbf{Útil cuando:} quieres variedad rápida de ítems.\\
  \textbf{Cuidado:} revisar calidad, sesgos y alineación con OA.
\item
  \textbf{Formularios (Google Forms) + IA}\\
  Redactas preguntas con IA y luego las pegas en Forms para aplicar y
  retroalimentar.\\
  \textbf{Útil cuando:} buscas aplicación simple y resultados ordenados.
\end{itemize}

\subsubsection{Lenguaje claro, reescritura y apoyo a
lectura/escritura}\label{lenguaje-claro-reescritura-y-apoyo-a-lecturaescritura}

\begin{itemize}
\item
  \textbf{LanguageTool}\\
  Corrección ortográfica y estilo (muy práctico en español).\\
  \textbf{Útil cuando:} necesitas mejorar redacción de
  guías/comunicados.
\item
  \textbf{DeepL Write / DeepL}\\
  Reescritura y traducción (si trabajas con material bilingüe).\\
  \textbf{Útil cuando:} quieres versiones ``más claras'' o traducciones
  cuidando tono.\\
  \textbf{Cuidado:} revisar términos curriculares y nombres propios.
\end{itemize}

\subsubsection{Audio, accesibilidad y apoyos para
estudiantes}\label{audio-accesibilidad-y-apoyos-para-estudiantes}

\begin{itemize}
\item
  \textbf{Text-to-Speech (TTS)} (lectores integrados o servicios de
  voz)\\
  Para convertir guías en audio y apoyar el acceso a información
  (DUA).\\
  \textbf{Útil cuando:} quieres otra vía de acceso.\\
  \textbf{Cuidado:} no subir textos con datos sensibles.
\item
  \textbf{Whisper / transcripción (cuando se permite)}\\
  Para pasar audio a texto y armar minutas o guías.\\
  \textbf{Útil cuando:} tienes grabaciones propias (reuniones,
  clases).\\
  \textbf{Cuidado:} consentimiento y resguardo de privacidad.
\end{itemize}

\subsubsection{Video y edición rápida (sin
complicarse)}\label{video-y-ediciuxf3n-ruxe1pida-sin-complicarse}

\begin{itemize}
\item
  \textbf{CapCut}\\
  Edición simple, subtítulos automáticos y clips.\\
  \textbf{Útil cuando:} quieres cápsulas cortas o material
  motivacional.\\
  \textbf{Cuidado:} derechos de música e imágenes; no usar rostros
  reales de estudiantes.
\item
  \textbf{Loom}\\
  Grabar explicación breve (pantalla + voz) para apoyar a quienes
  faltaron.\\
  \textbf{Útil cuando:} necesitas ``ponerse al día'' sin repetir todo.\\
  \textbf{Cuidado:} no mostrar listados, notas ni datos del curso.
\end{itemize}

\subsubsection{Investigación docente, búsqueda con fuentes y ``apoyo con
evidencia''}\label{investigaciuxf3n-docente-buxfasqueda-con-fuentes-y-apoyo-con-evidencia}

\begin{itemize}
\item
  \textbf{Perplexity (u otros buscadores con IA)}\\
  Para explorar información con referencias (contexto, bibliografía).\\
  \textbf{Útil cuando:} necesitas buscar rápido y comparar fuentes.\\
  \textbf{Cuidado:} siempre verificar y no usar como única fuente.
\item
  \textbf{Elicit}\\
  Ayuda a encontrar y resumir artículos académicos.\\
  \textbf{Útil cuando:} estás escribiendo tu guía como \textbf{paper
  metodológico} o estado del arte.
\end{itemize}

\subsubsection{Pizarras y co-creación en
equipo}\label{pizarras-y-co-creaciuxf3n-en-equipo}

\begin{itemize}
\tightlist
\item
  \textbf{Miro / FigJam}\\
  Para trabajo colaborativo (mapas, lluvia de ideas, planificación).\\
  \textbf{Útil cuando:} el taller es grupal y quieres visualizar
  acuerdos.\\
  \textbf{Cuidado:} no pegar datos personales; usar ejemplos genéricos.
\end{itemize}

\bookmarksetup{startatroot}

\chapter{Extras}\label{extras}

\section{Contacto}\label{contacto}

\begin{itemize}
\tightlist
\item
  \href{mailto:katherine.aravena.h@gmail.com}{\nolinkurl{katherine.aravena.h@gmail.com}}
\item
  \href{mailto:Paula.cerda.t@ug.uchile.cl}{\nolinkurl{Paula.cerda.t@ug.uchile.cl}}
\item
  \href{mailto:juan.castano@ug.uchile.cl}{\nolinkurl{juan.castano@ug.uchile.cl}}
\end{itemize}

\section{Encuesta}\label{encuesta}

\begin{itemize}
\tightlist
\item
  \href{https://docs.google.com/forms/d/e/1FAIpQLSd7LhB3fQZ46iEG-4h5_EFQadj2gtLfIfPsfVcaZqHES6825Q/viewform}{Encuesta
  del taller (Google Forms)}
\end{itemize}

Gracias por participar. Esperamos que el taller haya sido útil y que se
lleven ideas prácticas para su trabajo docente. Si pueden completar la
encuesta, nos ayudan mucho a mejorar futuras versiones.

\section{}\label{section}

\textit{La presentación es interactiva y se visualiza solo en la versión HTML.}

\phantomsection\label{refs}
\begin{CSLReferences}{1}{0}
\bibitem[\citeproctext]{ref-vaswani2017attention}
Vaswani, A., Shazeer, N., Parmar, N., Uszkoreit, J., Jones, L., Gomez,
A. N., Kaiser, Ł., \& Polosukhin, I. (2017). Attention is all you need.
\emph{Advances in neural information processing systems}, \emph{30}.

\end{CSLReferences}


\backmatter

% --- Cuerpo del libro (capítulos) en arábigos ---
\mainmatter
\pagestyle{scrheadings}

% --- (Opcional) Estilo del índice general ---
% \addtocontents{toc}{\protect\thispagestyle{plain}}


\end{document}
