% Options for packages loaded elsewhere
% Options for packages loaded elsewhere
\PassOptionsToPackage{unicode}{hyperref}
\PassOptionsToPackage{hyphens}{url}
\PassOptionsToPackage{dvipsnames,svgnames,x11names}{xcolor}
%
\documentclass[
  spanish,
  a4paper,
  oneside,
  shorthands=off]{scrbook}
\usepackage{xcolor}
\usepackage[top=25mm,bottom=25mm,left=30mm,right=20mm]{geometry}
\usepackage{amsmath,amssymb}
\setcounter{secnumdepth}{5}
\usepackage{iftex}
\ifPDFTeX
  \usepackage[T1]{fontenc}
  \usepackage[utf8]{inputenc}
  \usepackage{textcomp} % provide euro and other symbols
\else % if luatex or xetex
  \usepackage{unicode-math} % this also loads fontspec
  \defaultfontfeatures{Scale=MatchLowercase}
  \defaultfontfeatures[\rmfamily]{Ligatures=TeX,Scale=1}
\fi
\usepackage{lmodern}
\ifPDFTeX\else
  % xetex/luatex font selection
  \setmainfont[]{Times New Roman}
\fi
% Use upquote if available, for straight quotes in verbatim environments
\IfFileExists{upquote.sty}{\usepackage{upquote}}{}
\IfFileExists{microtype.sty}{% use microtype if available
  \usepackage[]{microtype}
  \UseMicrotypeSet[protrusion]{basicmath} % disable protrusion for tt fonts
}{}
\makeatletter
\@ifundefined{KOMAClassName}{% if non-KOMA class
  \IfFileExists{parskip.sty}{%
    \usepackage{parskip}
  }{% else
    \setlength{\parindent}{0pt}
    \setlength{\parskip}{6pt plus 2pt minus 1pt}}
}{% if KOMA class
  \KOMAoptions{parskip=half}}
\makeatother
% Make \paragraph and \subparagraph free-standing
\makeatletter
\ifx\paragraph\undefined\else
  \let\oldparagraph\paragraph
  \renewcommand{\paragraph}{
    \@ifstar
      \xxxParagraphStar
      \xxxParagraphNoStar
  }
  \newcommand{\xxxParagraphStar}[1]{\oldparagraph*{#1}\mbox{}}
  \newcommand{\xxxParagraphNoStar}[1]{\oldparagraph{#1}\mbox{}}
\fi
\ifx\subparagraph\undefined\else
  \let\oldsubparagraph\subparagraph
  \renewcommand{\subparagraph}{
    \@ifstar
      \xxxSubParagraphStar
      \xxxSubParagraphNoStar
  }
  \newcommand{\xxxSubParagraphStar}[1]{\oldsubparagraph*{#1}\mbox{}}
  \newcommand{\xxxSubParagraphNoStar}[1]{\oldsubparagraph{#1}\mbox{}}
\fi
\makeatother

\usepackage{color}
\usepackage{fancyvrb}
\newcommand{\VerbBar}{|}
\newcommand{\VERB}{\Verb[commandchars=\\\{\}]}
\DefineVerbatimEnvironment{Highlighting}{Verbatim}{commandchars=\\\{\}}
% Add ',fontsize=\small' for more characters per line
\usepackage{framed}
\definecolor{shadecolor}{RGB}{241,243,245}
\newenvironment{Shaded}{\begin{snugshade}}{\end{snugshade}}
\newcommand{\AlertTok}[1]{\textcolor[rgb]{0.68,0.00,0.00}{#1}}
\newcommand{\AnnotationTok}[1]{\textcolor[rgb]{0.37,0.37,0.37}{#1}}
\newcommand{\AttributeTok}[1]{\textcolor[rgb]{0.40,0.45,0.13}{#1}}
\newcommand{\BaseNTok}[1]{\textcolor[rgb]{0.68,0.00,0.00}{#1}}
\newcommand{\BuiltInTok}[1]{\textcolor[rgb]{0.00,0.23,0.31}{#1}}
\newcommand{\CharTok}[1]{\textcolor[rgb]{0.13,0.47,0.30}{#1}}
\newcommand{\CommentTok}[1]{\textcolor[rgb]{0.37,0.37,0.37}{#1}}
\newcommand{\CommentVarTok}[1]{\textcolor[rgb]{0.37,0.37,0.37}{\textit{#1}}}
\newcommand{\ConstantTok}[1]{\textcolor[rgb]{0.56,0.35,0.01}{#1}}
\newcommand{\ControlFlowTok}[1]{\textcolor[rgb]{0.00,0.23,0.31}{\textbf{#1}}}
\newcommand{\DataTypeTok}[1]{\textcolor[rgb]{0.68,0.00,0.00}{#1}}
\newcommand{\DecValTok}[1]{\textcolor[rgb]{0.68,0.00,0.00}{#1}}
\newcommand{\DocumentationTok}[1]{\textcolor[rgb]{0.37,0.37,0.37}{\textit{#1}}}
\newcommand{\ErrorTok}[1]{\textcolor[rgb]{0.68,0.00,0.00}{#1}}
\newcommand{\ExtensionTok}[1]{\textcolor[rgb]{0.00,0.23,0.31}{#1}}
\newcommand{\FloatTok}[1]{\textcolor[rgb]{0.68,0.00,0.00}{#1}}
\newcommand{\FunctionTok}[1]{\textcolor[rgb]{0.28,0.35,0.67}{#1}}
\newcommand{\ImportTok}[1]{\textcolor[rgb]{0.00,0.46,0.62}{#1}}
\newcommand{\InformationTok}[1]{\textcolor[rgb]{0.37,0.37,0.37}{#1}}
\newcommand{\KeywordTok}[1]{\textcolor[rgb]{0.00,0.23,0.31}{\textbf{#1}}}
\newcommand{\NormalTok}[1]{\textcolor[rgb]{0.00,0.23,0.31}{#1}}
\newcommand{\OperatorTok}[1]{\textcolor[rgb]{0.37,0.37,0.37}{#1}}
\newcommand{\OtherTok}[1]{\textcolor[rgb]{0.00,0.23,0.31}{#1}}
\newcommand{\PreprocessorTok}[1]{\textcolor[rgb]{0.68,0.00,0.00}{#1}}
\newcommand{\RegionMarkerTok}[1]{\textcolor[rgb]{0.00,0.23,0.31}{#1}}
\newcommand{\SpecialCharTok}[1]{\textcolor[rgb]{0.37,0.37,0.37}{#1}}
\newcommand{\SpecialStringTok}[1]{\textcolor[rgb]{0.13,0.47,0.30}{#1}}
\newcommand{\StringTok}[1]{\textcolor[rgb]{0.13,0.47,0.30}{#1}}
\newcommand{\VariableTok}[1]{\textcolor[rgb]{0.07,0.07,0.07}{#1}}
\newcommand{\VerbatimStringTok}[1]{\textcolor[rgb]{0.13,0.47,0.30}{#1}}
\newcommand{\WarningTok}[1]{\textcolor[rgb]{0.37,0.37,0.37}{\textit{#1}}}

\usepackage{longtable,booktabs,array}
\usepackage{calc} % for calculating minipage widths
% Correct order of tables after \paragraph or \subparagraph
\usepackage{etoolbox}
\makeatletter
\patchcmd\longtable{\par}{\if@noskipsec\mbox{}\fi\par}{}{}
\makeatother
% Allow footnotes in longtable head/foot
\IfFileExists{footnotehyper.sty}{\usepackage{footnotehyper}}{\usepackage{footnote}}
\makesavenoteenv{longtable}
\usepackage{graphicx}
\makeatletter
\newsavebox\pandoc@box
\newcommand*\pandocbounded[1]{% scales image to fit in text height/width
  \sbox\pandoc@box{#1}%
  \Gscale@div\@tempa{\textheight}{\dimexpr\ht\pandoc@box+\dp\pandoc@box\relax}%
  \Gscale@div\@tempb{\linewidth}{\wd\pandoc@box}%
  \ifdim\@tempb\p@<\@tempa\p@\let\@tempa\@tempb\fi% select the smaller of both
  \ifdim\@tempa\p@<\p@\scalebox{\@tempa}{\usebox\pandoc@box}%
  \else\usebox{\pandoc@box}%
  \fi%
}
% Set default figure placement to htbp
\def\fps@figure{htbp}
\makeatother



\ifLuaTeX
\usepackage[bidi=basic]{babel}
\else
\usepackage[bidi=default]{babel}
\fi
\ifPDFTeX
\else
\babelfont{rm}[]{Times New Roman}
\fi
% get rid of language-specific shorthands (see #6817):
\let\LanguageShortHands\languageshorthands
\def\languageshorthands#1{}


\setlength{\emergencystretch}{3em} % prevent overfull lines

\providecommand{\tightlist}{%
  \setlength{\itemsep}{0pt}\setlength{\parskip}{0pt}}



 


% ---------- Paquetes tipograficos y microajustes ----------
\usepackage{microtype}        % mejor interletrado y justificado
\usepackage{csquotes}         % comillas tipograficas
\usepackage{iftex}
\ifPDFTeX\else
  % Seleccion de fuentes solo cuando XeTeX/LuaTeX esta disponible.
  \IfFontExistsTF{Times New Roman}{
    \setmainfont{Times New Roman}
  }{
    \IfFontExistsTF{TeX Gyre Termes}{
      \setmainfont{TeX Gyre Termes}
    }{
      \IfFontExistsTF{Latin Modern Roman}{\setmainfont{Latin Modern Roman}}{}
    }
  }
  \IfFontExistsTF{Times New Roman}{
    \setsansfont{Times New Roman}
  }{
    \IfFontExistsTF{TeX Gyre Heros}{
      \setsansfont{TeX Gyre Heros}
    }{
      \IfFontExistsTF{Latin Modern Sans}{\setsansfont{Latin Modern Sans}}{}
    }
  }
  \IfFontExistsTF{Times New Roman}{
    \setmonofont{Times New Roman}
  }{
    \IfFontExistsTF{TeX Gyre Cursor}{
      \setmonofont{TeX Gyre Cursor}
    }{
      \IfFontExistsTF{Latin Modern Mono}{\setmonofont{Latin Modern Mono}}{}
    }
  }
\fi
\usepackage{enumitem}         % listas compactas
\setlist{itemsep=.2em, topsep=.2em}

% ---------- KOMA-Script: estilo de titulos y espaciado ----------
\KOMAoptions{
  headings=big,
  parskip=half,
  fontsize=12pt,
  appendixprefix=true
}
\setlength{\parindent}{1.5em}
\setlength{\parskip}{0.6em}
\usepackage{indentfirst}
\usepackage{setspace}
\setstretch{1.5}


% ---------- Encabezados y pies (scrlayer-scrpage) ----------
\usepackage[automark,headsepline]{scrlayer-scrpage}
\clearpairofpagestyles
\automark[chapter]{chapter}
\ihead{\pagemark}
\ohead{\itshape\headmark}
\setheadsepline{0.4pt}
\renewcommand*{\chaptermarkformat}{}
\renewcommand*{\chapterpagestyle}{scrheadings}
\pagestyle{scrheadings}
\makeatletter
\let\ps@plain\ps@scrheadings
\makeatother
\cfoot{}

% ---------- Hipervinculos mas sobrios ----------
\usepackage{hyperref}
\hypersetup{
  colorlinks=true,
  linkcolor=blue,
  citecolor=blue,
  urlcolor=blue,
  pdfauthor={\@author},
  pdftitle={\@title}
}

% ---------- Leyendas de figuras/tablas ----------
\usepackage[labelfont=bf,textfont=it]{caption}
\captionsetup{
  skip=8pt
}

% ---------- Tabla de contenidos ----------
\KOMAoptions{toc=graduated}
\RedeclareSectionCommand[tocnumwidth=3em]{chapter}
\RedeclareSectionCommand[tocindent=3.25em,tocnumwidth=2.8em]{section}
\RedeclareSectionCommand[tocindent=6.5em,tocnumwidth=2.5em]{subsection}
\makeatletter
\newcommand*{\tocdotfill}{\leavevmode\leaders\hbox to .6em{\hss.\hss}\hfill}
\makeatother
\RedeclareSectionCommand[toclinefill=\tocdotfill]{chapter}
\RedeclareSectionCommand[toclinefill=\tocdotfill]{section}
\RedeclareSectionCommand[toclinefill=\tocdotfill]{subsection}
\renewcommand*{\contentsname}{Tabla de contenido}
\setkomafont{chapterentry}{\normalfont}
\setkomafont{chapterentrypagenumber}{\normalfont}

% ---------- Viudas/Huerfanas y cortes de pagina ----------
\clubpenalty=10000
\widowpenalty=10000
\displaywidowpenalty=10000

% ---------- Entorno abstract para scrbook ----------
\providecommand{\abstractname}{Resumen}
\makeatletter
\@ifundefined{abstract}{
  \newenvironment{abstract}{
    \cleardoublepage
    \thispagestyle{plain}
    \null\vfill
    \begin{center}
      {\bfseries\Large \abstractname\par}
    \end{center}\vspace{1em}
    \begingroup
  }{
    \par\endgroup
    \vfill\null
    \cleardoublepage
  }
}{}
\makeatother

% ---------- Soporte de subtitulo desde YAML ----------
\makeatletter
\providecommand{\subtitle}[1]{\gdef\@subtitle{#1}}
\providecommand{\@subtitle}{}
\makeatother

% --- Desactivar portada y abstract automaticos de Pandoc (PDF) ---
\AtBeginDocument{\let\maketitle\relax}
\renewenvironment{abstract}{}{}
\providecommand{\appendixname}{}
\providecommand{\appendixtocname}{}
\providecommand{\appendixpagename}{}
\renewcommand*{\appendixname}{Anexo}
\renewcommand*{\appendixtocname}{Anexos}
\renewcommand*{\appendixpagename}{Anexos}


% ---------- Indicadores para LOF/LOT condicionales ----------
\newif\iffacsofigexists
\newif\iffacsotableexists
\InputIfFileExists{\jobname.facsoflags}{}{}
\newwrite\FacsoFlagStream
\AtEndDocument{%
  \immediate\openout\FacsoFlagStream=\jobname.facsoflags
  \ifnum\value{figure}>0
    \immediate\write\FacsoFlagStream{\string\facsofigexiststrue}
  \else
    \immediate\write\FacsoFlagStream{\string\facsofigexistsfalse}
  \fi
  \ifnum\value{table}>0
    \immediate\write\FacsoFlagStream{\string\facsotableexiststrue}
  \else
    \immediate\write\FacsoFlagStream{\string\facsotableexistsfalse}
  \fi
  \immediate\closeout\FacsoFlagStream
}
\hypersetup{
  colorlinks=true,
  linkcolor=blue,
  urlcolor=blue,
  citecolor=blue
}
\makeatletter
\@ifpackageloaded{bookmark}{}{\usepackage{bookmark}}
\makeatother
\makeatletter
\@ifpackageloaded{caption}{}{\usepackage{caption}}
\AtBeginDocument{%
\ifdefined\contentsname
  \renewcommand*\contentsname{Tabla de contenidos}
\else
  \newcommand\contentsname{Tabla de contenidos}
\fi
\ifdefined\listfigurename
  \renewcommand*\listfigurename{Listado de Figuras}
\else
  \newcommand\listfigurename{Listado de Figuras}
\fi
\ifdefined\listtablename
  \renewcommand*\listtablename{Listado de Tablas}
\else
  \newcommand\listtablename{Listado de Tablas}
\fi
\ifdefined\figurename
  \renewcommand*\figurename{Lista de figuras}
\else
  \newcommand\figurename{Lista de figuras}
\fi
\ifdefined\tablename
  \renewcommand*\tablename{Lista de tablas}
\else
  \newcommand\tablename{Lista de tablas}
\fi
}
\@ifpackageloaded{float}{}{\usepackage{float}}
\floatstyle{ruled}
\@ifundefined{c@chapter}{\newfloat{codelisting}{h}{lop}}{\newfloat{codelisting}{h}{lop}[chapter]}
\floatname{codelisting}{Listado}
\newcommand*\listoflistings{\listof{codelisting}{Listado de Listados}}
\makeatother
\makeatletter
\makeatother
\makeatletter
\@ifpackageloaded{caption}{}{\usepackage{caption}}
\@ifpackageloaded{subcaption}{}{\usepackage{subcaption}}
\makeatother
\usepackage{bookmark}
\IfFileExists{xurl.sty}{\usepackage{xurl}}{} % add URL line breaks if available
\urlstyle{same}
\hypersetup{
  pdftitle={IA al servicio pedagógico},
  pdfauthor={Katherine Aravena Herrera; Manuel Sierra},
  pdflang={es},
  colorlinks=true,
  linkcolor={black},
  filecolor={Maroon},
  citecolor={blue},
  urlcolor={blue},
  pdfcreator={LaTeX via pandoc}}


\title{IA al servicio pedagógico}
\usepackage{etoolbox}
\makeatletter
\providecommand{\subtitle}[1]{% add subtitle to \maketitle
  \apptocmd{\@title}{\par {\large #1 \par}}{}{}
}
\makeatother
\subtitle{Herramientas prácticas para el trabajo docente}
\author{Katherine Aravena Herrera \and Manuel Sierra}
\date{12 de diciembre de 2025}
\begin{document}
\frontmatter
\maketitle

% --- title-pdf.tex personalizado para portada formal ---
\newcommand{\CoverFont}{
  \ifdefined\fontspec
    \fontspec{Times New Roman}
  \else
    \fontfamily{ptm}\selectfont
  \fi
}
\makeatletter
\providecommand{\subtitle}[1]{\gdef\@subtitle{#1}}
\providecommand{\@subtitle}{}
\providecommand{\frontmattercontext}{}
\providecommand{\advisorname}{}
\providecommand{\advisorlabel}{Profesor guia:}
\providecommand{\frontmatterlocation}{}
\InputIfFileExists{includes/cover-config.tex}{}{}

\newcommand{\PrintTitle}{%
  {\CoverFont\fontsize{24pt}{28pt}\selectfont \@title\par}%
}
\newcommand{\PrintSubtitle}{%
  \begingroup
  \edef\temp{\detokenize{\@subtitle}}%
  \ifx\temp\empty\relax
    % sin subtitulo
  \else
    {\CoverFont\large \@subtitle\par}%
  \fi
  \endgroup
}
\newcommand{\PrintAuthor}{%
  \begingroup
  \renewcommand{\and}{\\[0.35em]}% separa autores en lineas, sin tabular
  {\CoverFont\Large\bfseries \@author\par}%
  \endgroup
}
\newcommand{\PrintDate}{%
  {\CoverFont\small \@date\par}%
}
\makeatother

\begin{titlepage}
\thispagestyle{empty}
\begin{center}
\CoverFont
\vspace*{10mm}

% Logo o imagen institucional
\includegraphics[width=0.35\textwidth]{assets/cover.png}\par
\vspace{12mm}

% Titulo y subtitulo
\PrintTitle
\vspace{6mm}
\PrintSubtitle

\vspace{22mm}
\begingroup
\edef\temp{\detokenize{\frontmattercontext}}%
\ifx\temp\empty\relax
  % sin contexto adicional
\else
  {\normalsize \CoverFont \frontmattercontext\par}%
\fi
\endgroup

\vspace{18mm}
\PrintAuthor

\vspace{16mm}
\begingroup
\edef\temp{\detokenize{\advisorname}}%
\ifx\temp\empty\relax
  % sin tutor
\else
  \rule{0.45\textwidth}{0.4pt}\par
  {\small \CoverFont \advisorlabel\ \advisorname\par}%
\fi
\endgroup

\vfill
\begingroup
\edef\temp{\detokenize{\frontmatterlocation}}%
\ifx\temp\empty\relax
  % sin ubicacion
\else
  {\small \CoverFont \frontmatterlocation\par}%
\fi
\endgroup
\PrintDate

\end{center}
\end{titlepage}

% Preliminares (numeros romanos) y estilo simple
\frontmatter
\pagestyle{scrheadings}

\setcounter{tocdepth}{2} % (o 1/3 segun prefieras)
\tableofcontents
\iffacsofigexists
  \cleardoublepage
  \listoffigures
\fi
\cleardoublepage


\mainmatter
\bookmarksetup{startatroot}

\chapter*{Presentación}\label{presentaciuxf3n}
\addcontentsline{toc}{chapter}{Presentación}

\markboth{Presentación}{Presentación}

\small\textbf{Palabras clave: } Inteligencia artificial; Trabajo
docente; Planificación escolar; Evaluación formativa; Inclusión
educativa. \normalsize

El presente documento presenta el taller \emph{``IA al servicio
pedagógico: herramientas prácticas para el trabajo docente''}, dirigido
a profesoras y profesores que participan en la Escuela de Verano 2026
del Museo de la Educación Gabriela Mistral. El taller se inscribe en una
perspectiva crítica y situada sobre el uso de la inteligencia artificial
en contextos escolares, entendida como un apoyo al trabajo profesional
docente y no como su reemplazo.

El propósito central de esta instancia es ofrecer un espacio práctico y
reflexivo para explorar cómo herramientas de IA pueden contribuir a
aliviar la sobrecarga laboral, mejorar la planificación, la evaluación y
la retroalimentación, así como adaptar materiales a distintos niveles y
necesidades de estudiantes. A partir de ejemplos concretos, actividades
guiadas y momentos de conversación pedagógica, se busca que las y los
participantes desarrollen criterios informados para decidir cuándo, cómo
y para qué utilizar estas tecnologías, resguardando siempre la autonomía
profesional y el sentido pedagógico de las decisiones en el aula. Este
diseño organiza los objetivos, contenidos, secuencias de trabajo y
orientaciones metodológicas del taller, de modo que pueda ser
implementado y adaptado en distintos contextos escolares.

\bookmarksetup{startatroot}

\chapter{¿Qué es la IA y cómo
funciona?}\label{quuxe9-es-la-ia-y-cuxf3mo-funciona}

\section{¿Qué entendemos por inteligencia artificial
hoy?}\label{quuxe9-entendemos-por-inteligencia-artificial-hoy}

En este taller entenderemos por \emph{inteligencia artificial (IA)} un
conjunto de técnicas informáticas que permiten a los computadores
realizar tareas que, si las hiciera una persona, consideraríamos
``inteligentes'': reconocer patrones, generar texto o imágenes, resumir
información, traducir, clasificar, entre otras. No se trata de una
``mente'' ni de un sujeto, sino de programas que aprenden a partir de
grandes volúmenes de datos.

En la práctica cotidiana, cuando hablamos de IA hoy casi siempre nos
referimos a sistemas basados en \textbf{aprendizaje automático}
(\emph{machine learning}) y, en particular, a \textbf{modelos de
lenguaje} y otros modelos generativos que se entrenan con enormes
cantidades de textos, imágenes, audio o video. Estos modelos aprenden a
detectar regularidades estadísticas y a ``predecir'' qué palabra, imagen
o respuesta es más probable según el contexto.

Es importante distinguir entre:

\begin{itemize}
\tightlist
\item
  \textbf{IA general} (la idea de una inteligencia similar o superior a
  la humana, aún inexistente).
\item
  \textbf{IA específica o aplicada}, que resuelve tareas concretas (por
  ejemplo, sugerir actividades para una clase, redactar un correo,
  resumir un texto escolar).
\item
  \textbf{IA generativa}, que crea nuevos contenidos (textos, imágenes,
  código, etc.) combinando patrones aprendidos.
\end{itemize}

En el contexto escolar trabajaremos con esta tercera familia,
entendiendo que \textbf{son herramientas al servicio del juicio
pedagógico} de las y los docentes, y no sustitutos de su trabajo
profesional.

\begin{center}\rule{0.5\linewidth}{0.5pt}\end{center}

\section{Modelos de lenguaje: cómo funcionan ``a grandes
rasgos''}\label{modelos-de-lenguaje-cuxf3mo-funcionan-a-grandes-rasgos}

\subsection{Redes neuronales artificiales (idea
general)}\label{redes-neuronales-artificiales-idea-general}

Los modelos de lenguaje actuales se basan en \textbf{redes neuronales
artificiales}. Una red neuronal es un modelo matemático formado por
capas de ``neuronas'' conectadas entre sí. Cada neurona recibe números
como entrada, los combina y produce una salida. Al entrenar la red con
muchos ejemplos, va ajustando sus conexiones internas para cometer cada
vez menos errores.

En términos muy simples:

\begin{enumerate}
\def\labelenumi{\arabic{enumi}.}
\tightlist
\item
  Se representa el texto como números (tokens).
\item
  La red neuronal recibe esos números y genera una predicción (por
  ejemplo, la siguiente palabra).
\item
  Se compara la predicción con la respuesta correcta.
\item
  El modelo ajusta internamente sus parámetros para equivocarse menos la
  próxima vez.
\item
  Se repite el proceso millones de veces con enormes corpus de datos.
\end{enumerate}

Podemos visualizar una red neuronal básica así:

\begin{figure}[H]

\caption{Ejemplo de red neuronal artificial con una capa de entrada, una
oculta y una de salida.\\
Fuente: Cburnett, \emph{Artificial neural network}, Wikimedia Commons
(CC BY-SA 3.0).}

{\centering \pandocbounded{\includegraphics[keepaspectratio]{assets/ann.png}}

}

\end{figure}%

En los modelos actuales, en vez de unas pocas capas hay \textbf{decenas
o cientos de capas}, con millones o miles de millones de parámetros. Por
eso se habla de \emph{modelos grandes}.

\subsection{Del perceptrón al
Transformer}\label{del-perceptruxf3n-al-transformer}

Los modelos de lenguaje modernos utilizan una arquitectura llamada
\textbf{Transformer}, que fue propuesta en 2017 y cambió radicalmente la
forma de trabajar con texto. La idea central es el mecanismo de
\textbf{auto-atención} (\emph{self-attention}), que permite que el
modelo mire todas las palabras de una frase a la vez y decida a cuáles
prestar más atención para entender el contexto.

A grandes rasgos, un Transformer:

\begin{enumerate}
\def\labelenumi{\arabic{enumi}.}
\tightlist
\item
  \textbf{Recibe una secuencia de palabras} (convertidas en vectores
  numéricos).
\item
  \textbf{Calcula la atención}: para cada palabra, pondera cuánto se
  relaciona con las demás palabras de la secuencia.
\item
  \textbf{Transforma la representación interna del texto} pasando por
  varias capas que combinan atención y redes neuronales.
\item
  \textbf{Predice la siguiente palabra} (o el siguiente fragmento de
  texto) eligiendo la opción más probable según lo aprendido.
\end{enumerate}

La arquitectura típica de un Transformer puede representarse así:

\begin{figure}[H]

\caption{Arquitectura general de un modelo tipo Transformer con bloques
de codificador y decodificador.\\
Fuente: Yuening Jia, \emph{The Transformer - model architecture},
Wikimedia Commons (CC BY-SA 3.0).}

{\centering \pandocbounded{\includegraphics[keepaspectratio]{assets/transformer.png}}

}

\end{figure}%

En esta imagen no necesitamos comprender cada bloque en detalle para
trabajar pedagógicamente con IA. Lo importante para el taller es retener
tres ideas:

\begin{itemize}
\tightlist
\item
  El modelo \textbf{no ``piensa'' como una persona}, sino que calcula
  probabilidades sobre la base de patrones aprendidos.
\item
  El modelo \textbf{no sabe qué es verdadero o falso} por sí mismo: solo
  ha visto textos, no la realidad.
\item
  El modelo funciona como un \textbf{completador de texto muy
  sofisticado}, que podemos guiar mediante instrucciones claras
  (\emph{prompts}).
\end{itemize}

\begin{center}\rule{0.5\linewidth}{0.5pt}\end{center}

\section{Qué puede hacer y qué no puede hacer en
educación}\label{quuxe9-puede-hacer-y-quuxe9-no-puede-hacer-en-educaciuxf3n}

\subsection{¿Qué sí puede hacer?}\label{quuxe9-suxed-puede-hacer}

En el contexto escolar, modelos de lenguaje e IA generativa pueden
apoyar, entre otras tareas:

\begin{itemize}
\tightlist
\item
  \textbf{Planificación y preparación de clases}: sugerir objetivos,
  actividades, secuencias, preguntas guía.
\item
  \textbf{Elaboración de materiales}: generar borradores de guías,
  ejercicios, ejemplos contextualizados, casos para debatir.
\item
  \textbf{Evaluación y retroalimentación}: proponer ítems y criterios,
  sugerir comentarios para que la/el docente revise y adapte.
\item
  \textbf{Adaptación de materiales}: simplificar un texto, cambiar el
  nivel de complejidad, generar versiones alternativas para diferentes
  cursos.
\item
  \textbf{Organización del trabajo cotidiano}: redactar comunicaciones,
  sistematizar acuerdos de reuniones, sintetizar documentos extensos.
\end{itemize}

En todos los casos, la IA actúa como \textbf{asistente}: entrega
propuestas iniciales que la/el docente revisa, corrige y contextualiza
según su criterio profesional.

\subsection{¿Qué no puede hacer (ni debería
hacer)?}\label{quuxe9-no-puede-hacer-ni-deberuxeda-hacer}

Hay tareas que, por razones técnicas, éticas o pedagógicas, la IA
\textbf{no debiera reemplazar}:

\begin{itemize}
\tightlist
\item
  \textbf{Conocer a las y los estudiantes} en su historia, contexto
  familiar, trayectoria escolar, emociones y vínculos.
\item
  \textbf{Tomar decisiones evaluativas de alto impacto}
  (aprobación/reprobación, repitencia, derivaciones) sin mediación
  humana.
\item
  \textbf{Definir objetivos de formación} y sentidos educativos de largo
  plazo.
\item
  \textbf{Sostener vínculos afectivos y pedagógicos} que requieren
  presencia humana, cuidado y responsabilidad.
\item
  \textbf{Garantizar veracidad y ausencia de sesgos}: los modelos pueden
  inventar datos, reproducir estereotipos o desinformación.
\end{itemize}

Para este taller asumiremos, por tanto, que la IA es una
\textbf{herramienta de apoyo} al trabajo pedagógico, que requiere
siempre:

\begin{itemize}
\tightlist
\item
  Revisión crítica.
\item
  Adaptación al contexto.
\item
  Alineamiento con el proyecto educativo y el marco curricular.
\end{itemize}

\begin{center}\rule{0.5\linewidth}{0.5pt}\end{center}

\section{Mitos, temores y preguntas frecuentes en la
escuela}\label{mitos-temores-y-preguntas-frecuentes-en-la-escuela}

En el trabajo con docentes suelen aparecer inquietudes que es importante
acoger y discutir colectivamente. Esta sección ofrece una primera
sistematización que luego se puede complementar con las experiencias del
grupo.

\subsection{Mitos frecuentes}\label{mitos-frecuentes}

\begin{itemize}
\item
  \textbf{``La IA viene a reemplazar a profesoras y profesores.''}\\
  En realidad, los modelos actuales no pueden asumir las múltiples
  dimensiones del trabajo docente: vínculo, ética, juicio profesional,
  gestión del aula, trabajo con familias, etc. Sí pueden automatizar
  tareas repetitivas y de baja creatividad, liberando tiempo para lo
  pedagógicamente más relevante.
\item
  \textbf{``Si uso IA, las estudiantes solo copiarán y pegarán.''}\\
  El riesgo existe, pero depende de cómo se diseñen las tareas. Podemos
  usar IA para \textbf{profundizar el aprendizaje} (por ejemplo,
  pidiendo que comparen, evalúen, refuten o mejoren respuestas generadas
  por la herramienta).
\item
  \textbf{``La IA siempre tiene la razón.''}\\
  Los modelos se equivocan, inventan referencias y pueden reproducir
  sesgos. Es clave enseñar a verificar la información y a leer
  críticamente los resultados.
\item
  \textbf{``Esto es solo una moda pasajera.''}\\
  Aunque muchas herramientas cambiarán, la presencia de modelos de
  lenguaje en educación y trabajo probablemente se volverá estructural.
  Por eso es importante desarrollar criterios propios, en vez de
  prohibir o adoptar acríticamente.
\end{itemize}

\subsection{Temores y preocupaciones
legítimas}\label{temores-y-preocupaciones-leguxedtimas}

\begin{itemize}
\tightlist
\item
  \textbf{Plagio y copia en tareas}: ¿cómo evaluar cuando el acceso a IA
  es masivo? ¿Qué tareas tienen sentido hoy?
\item
  \textbf{Brechas de acceso}: ¿qué pasa con estudiantes y escuelas que
  no tienen buena conectividad o dispositivos?
\item
  \textbf{Privacidad de datos}: ¿qué información es seguro ingresar en
  estas herramientas? ¿Qué no deberíamos nunca subir?
\item
  \textbf{Sobrecarga de exigencias}: temor a que la IA se convierta en
  ``una tarea más'' para el profesorado, en vez de aliviar el trabajo.
\end{itemize}

En el taller trabajaremos estos temas de manera abierta, buscando
acordar \textbf{criterios realistas y situados}. La idea no es resolver
todos los debates, sino abrir una conversación informada que permita a
cada docente tomar decisiones responsables en su propio contexto
escolar.

\bookmarksetup{startatroot}

\chapter{Claves de prompt engineering
docente}\label{claves-de-prompt-engineering-docente}

\section{Qué es un ``prompt'' y por qué importa en el trabajo
docente}\label{quuxe9-es-un-prompt-y-por-quuxe9-importa-en-el-trabajo-docente}

En términos simples, un \emph{prompt} es el mensaje o instrucción que le
damos a la inteligencia artificial para que haga algo por nosotras/os:
proponer ideas de clase, redactar una rúbrica, adaptar un texto, etc. Es
la ``consigna'' que guía la respuesta del modelo.

En el trabajo docente, la calidad del \emph{prompt} es clave porque:

\begin{itemize}
\tightlist
\item
  define \textbf{qué tarea} realizará la IA (por ejemplo, ``sugerir
  actividades para\ldots{}'', ``proponer preguntas de evaluación
  sobre\ldots{}'');\\
\item
  entrega el \textbf{contexto pedagógico} (curso, asignatura, objetivos
  de aprendizaje, características del establecimiento);\\
\item
  indica \textbf{cómo} queremos recibir la respuesta (tabla, lista,
  texto breve, tono formal o cercano, extensión aproximada);\\
\item
  establece \textbf{límites y resguardos} (no inventar datos, no usar
  lenguaje técnico, no tomar decisiones pedagógicas por la/el docente).
\end{itemize}

Cuando el \emph{prompt} es vago (``ayúdame con una clase de
fracciones'') la respuesta suele ser genérica y poco útil. En cambio,
cuando el \emph{prompt} está bien diseñado, con objetivo claro,
contexto, rol, formato y criterios, la IA puede convertirse en un apoyo
real para el trabajo docente, siempre bajo el criterio profesional de
la/el profesora/or.

Una forma sencilla de pensarlo es: \textbf{un buen \emph{prompt} se
parece a una buena consigna de trabajo para estudiantes}: clara, ubicada
en contexto y con criterios de logro explícitos.

\section{\texorpdfstring{Estructura de un buen \emph{prompt}
pedagógico}{Estructura de un buen prompt pedagógico}}\label{estructura-de-un-buen-prompt-pedaguxf3gico}

En esta sección se propone una ``columna vertebral'' para redactar
\emph{prompts} útiles en educación. No se trata de reglas rígidas, sino
de una guía que cada docente puede adaptar a su estilo.

\subsection{Los 5 componentes
básicos}\label{los-5-componentes-buxe1sicos}

\begin{enumerate}
\def\labelenumi{\arabic{enumi}.}
\tightlist
\item
  \textbf{Rol de la IA}\\
  Indica desde qué lugar queremos que responda la herramienta. Ejemplos:

  \begin{itemize}
  \tightlist
  \item
    ``Actúa como profesora de Historia de enseñanza media\ldots{}''\\
  \item
    ``Actúa como asesor pedagógico con experiencia en evaluación
    formativa\ldots{}''\\
  \item
    ``Actúa como especialista en educación inclusiva y diseño universal
    para el aprendizaje\ldots{}''
  \end{itemize}
\item
  \textbf{Objetivo / tarea principal}\\
  Qué queremos lograr con la respuesta, en una frase concreta.

  \begin{itemize}
  \tightlist
  \item
    ``Tu objetivo es proponer tres actividades breves para\ldots{}''\\
  \item
    ``Tu tarea es elaborar una rúbrica de evaluación para\ldots{}''\\
  \item
    ``Tu misión es sintetizar el siguiente texto para que lo pueda
    comprender un curso de 6º básico\ldots{}''
  \end{itemize}
\item
  \textbf{Contexto pedagógico}\\
  Información relevante sobre curso, asignatura, tipo de
  establecimiento, realidad del grupo, etc.

  \begin{itemize}
  \tightlist
  \item
    nivel y curso (5º básico, 2º medio técnico-profesional, etc.);\\
  \item
    asignatura y contenido específico;\\
  \item
    características del establecimiento (público, subvencionado, rural,
    urbano, etc.);\\
  \item
    características del grupo (heterogeneidad, presencia de PIE, etc.).
  \end{itemize}
\item
  \textbf{Formato de salida esperado}\\
  Cómo queremos recibir la información:

  \begin{itemize}
  \tightlist
  \item
    ``entrega la respuesta en una tabla con columnas\ldots{}'';\\
  \item
    ``escribe una lista numerada de máximo 5 puntos\ldots{}'';\\
  \item
    ``redacta un texto breve (máx. 200 palabras) para
    estudiantes\ldots{}'';\\
  \item
    ``proporciona ejemplos de preguntas de alternativa y de desarrollo
    corto''.
  \end{itemize}
\item
  \textbf{Criterios y restricciones (guardarraíles)}\\
  Aquí fijamos límites para resguardar el sentido pedagógico y la
  viabilidad:

  \begin{itemize}
  \tightlist
  \item
    ``usa lenguaje sencillo, sin tecnicismos'';\\
  \item
    ``no inventes datos ni fuentes, si no sabes dilo explícitamente'';\\
  \item
    ``no entregues actividades que requieran recursos costosos'';\\
  \item
    ``respeta lenguaje inclusivo y enfoque no sexista''.
  \end{itemize}
\end{enumerate}

\subsection{\texorpdfstring{Técnicas de \emph{prompting} para
docentes}{Técnicas de prompting para docentes}}\label{tuxe9cnicas-de-prompting-para-docentes}

A continuación se presentan algunas técnicas sencillas para mejorar la
calidad de las respuestas de la IA en el trabajo pedagógico cotidiano:

\begin{itemize}
\item
  \textbf{Sé específica/o y precisa/o}\\
  Evita pedir ``ideas para una clase'' de forma genérica. En cambio,
  indica curso, asignatura, contenido y objetivo. Mientras más claro sea
  el encargo, más útil será la respuesta.\\
  \emph{Ejemplo:} ``Propón tres actividades breves para introducir el
  concepto de fracción equivalente en 4º básico en una escuela
  pública''.
\item
  \textbf{Usa restricciones claras (guardarraíles)}\\
  Señala lo que la IA \textbf{no} debe hacer o aquello que quieres
  acotar: extensión, tono, tipo de recursos, nivel de lenguaje, etc.\\
  \emph{Ejemplo:} ``No uses jerga técnica, máximo 200 palabras,
  actividades que se puedan realizar sin computadores ni celulares''.
\item
  \textbf{Itera y refina: piensa el diálogo como un proceso}\\
  No esperes que la primera respuesta sea perfecta. Puedes pedir
  ajustes, correcciones o versiones mejoradas:\\
  \emph{Ejemplos:} ``Hazlo más breve'', ``adapta esto para 6º básico'',
  ``agrega una opción para estudiantes con NEE''. Tratarlo como una
  conversación mejora mucho el resultado.
\item
  \textbf{Encadena prompts (\emph{chain prompting})}\\
  Para tareas complejas, es mejor dividir el trabajo en pasos: primero
  objetivos, luego actividades, después evaluación. También puedes pedir
  que la propia IA sugiera el siguiente paso.\\
  \emph{Ejemplo:}

  \begin{enumerate}
  \def\labelenumi{\arabic{enumi})}
  \tightlist
  \item
    ``Ayúdame a definir los objetivos de una unidad sobre migración en
    2º medio''.\\
  \item
    ``Con esos objetivos, propón una secuencia de 5 clases''.\\
  \item
    ``Ahora sugiere posibles instrumentos de evaluación para esta
    secuencia''.
  \end{enumerate}
\item
  \textbf{Pide pasos, criterios o fuentes para revisar la calidad}\\
  Solicita que la IA muestre ``cómo llegó'' a lo que propone, qué
  supuestos usa o qué referencias generales considera. Esto ayuda a
  detectar errores y a tomar decisiones informadas.\\
  \emph{Ejemplos:} ``Explica en 4 pasos cómo construiste esta rúbrica'',
  ``indica qué supuestos estás usando'', ``si mencionas información
  factual, señala qué tipo de fuente deberíamos consultar para
  verificarla''.
\item
  \textbf{Modela el tipo de reflexión que quieres que haga}\\
  Puedes pedirle explícitamente que revise críticamente su propia
  respuesta antes de entregarla:\\
  \emph{Ejemplo:} ``Antes de responder, indica dos posibles problemas o
  limitaciones de tu propuesta y luego ofrece una versión mejorada que
  los aborde''.
\end{itemize}

Estas técnicas no reemplazan el criterio profesional docente, pero sí
ayudan a que la IA se convierta en una herramienta más ajustada a las
necesidades reales del aula.

\section{\texorpdfstring{Plantillas de \emph{prompts} reutilizables para
docencia}{Plantillas de prompts reutilizables para docencia}}\label{plantillas-de-prompts-reutilizables-para-docencia}

A continuación se proponen algunas plantillas listas para copiar, pegar
y adaptar según las necesidades de cada docente.

\subsection{\texorpdfstring{Plantillas base de \emph{prompt}
docente}{Plantillas base de prompt docente}}\label{plantillas-base-de-prompt-docente}

\begin{Shaded}
\begin{Highlighting}[]
\NormalTok{Actúa como }\CommentTok{[}\OtherTok{ROL O PERFIL QUE NECESITO}\CommentTok{]}\NormalTok{.}

\NormalTok{Tu objetivo es }\CommentTok{[}\OtherTok{TAREA PRINCIPAL QUE QUIERO LOGRAR}\CommentTok{]}\NormalTok{.}

\NormalTok{Contexto:}
\SpecialStringTok{{-} }\NormalTok{Curso y nivel: }\CommentTok{[}\OtherTok{CURSO, NIVEL}\CommentTok{]}\NormalTok{.}
\SpecialStringTok{{-} }\NormalTok{Asignatura y contenido: }\CommentTok{[}\OtherTok{ASIGNATURA, CONTENIDO}\CommentTok{]}\NormalTok{.}
\SpecialStringTok{{-} }\NormalTok{Tipo de establecimiento y características del grupo: }\CommentTok{[}\OtherTok{BREVE DESCRIPCIÓN}\CommentTok{]}\NormalTok{.}

\NormalTok{Instrucciones:}
\SpecialStringTok{{-} }\CommentTok{[}\OtherTok{PRIMERA INSTRUCCIÓN ESPECÍFICA}\CommentTok{]}\NormalTok{.}
\SpecialStringTok{{-} }\CommentTok{[}\OtherTok{SEGUNDA INSTRUCCIÓN ESPECÍFICA}\CommentTok{]}\NormalTok{.}
\SpecialStringTok{{-} }\CommentTok{[}\OtherTok{TERCERA INSTRUCCIÓN ESPECÍFICA}\CommentTok{]}\NormalTok{.}

\NormalTok{Formato de salida:}
\SpecialStringTok{{-} }\CommentTok{[}\OtherTok{TIPO DE FORMATO: TABLA, LISTA, TEXTO BREVE, ETC.}\CommentTok{]}\NormalTok{.}

\NormalTok{Criterios y restricciones:}
\SpecialStringTok{{-} }\CommentTok{[}\OtherTok{TONO, LENGUAJE, EXTENSIÓN, RECURSOS DISPONIBLES}\CommentTok{]}\NormalTok{.}
\SpecialStringTok{{-} }\CommentTok{[}\OtherTok{RESGUARDOS ÉTICOS Y PEDAGÓGICOS}\CommentTok{]}\NormalTok{.}

\NormalTok{Antes de responder, si falta información importante, hazme hasta 3 preguntas breves para aclarar el contexto.}
\end{Highlighting}
\end{Shaded}

\subsection{Diseñador de prompts pedagógicos para
docentes}\label{diseuxf1ador-de-prompts-pedaguxf3gicos-para-docentes}

\begin{Shaded}
\begin{Highlighting}[]
\NormalTok{Actúa como especialista en educación y en prompt engineering para docentes.}

\NormalTok{Tu tarea es ayudarme a DISEÑAR un prompt claro y efectivo que luego pueda usar con otra IA.}

\NormalTok{Contexto:}
\SpecialStringTok{{-} }\NormalTok{Objetivo general del prompt que necesito: }\CommentTok{[}\OtherTok{por ejemplo, “planificar una unidad de Historia para 8º básico sobre dictadura y memoria”}\CommentTok{]}\NormalTok{.}
\SpecialStringTok{{-} }\NormalTok{Tipo de producto que quiero obtener con ese futuro prompt: }\CommentTok{[}\OtherTok{por ejemplo, “una tabla con clases, objetivos, actividades y evaluaciones”}\CommentTok{]}\NormalTok{.}
\SpecialStringTok{{-} }\NormalTok{Público destinatario de los resultados: }\CommentTok{[}\OtherTok{por ejemplo, “docentes de escuela pública con alta carga laboral”}\CommentTok{]}\NormalTok{.}

\NormalTok{Instrucciones:}
\SpecialStringTok{1. }\NormalTok{Formula un único prompt completo y bien redactado que yo pueda copiar y usar tal cual.}
\SpecialStringTok{2. }\NormalTok{Asegúrate de que el prompt incluya: rol de la IA, objetivo, contexto pedagógico, formato de salida, criterios y restricciones.}
\SpecialStringTok{3. }\NormalTok{Al final, sugiere brevemente cómo podría ajustar ese prompt (por ejemplo, para otros niveles o asignaturas).}

\NormalTok{Formato de salida:}
\SpecialStringTok{{-} }\NormalTok{Primero, escribe el prompt final entre comillas.}
\SpecialStringTok{{-} }\NormalTok{Luego, en 3–4 viñetas, entrega sugerencias de ajustes posibles.}

\NormalTok{No expliques qué es un prompt ni teorices: concéntrate en darme un buen prompt listo para usar.}
\end{Highlighting}
\end{Shaded}

\bookmarksetup{startatroot}

\chapter{Herramientas prácticas para el trabajo
docente}\label{herramientas-pruxe1cticas-para-el-trabajo-docente}

\section{IA para planificación de clases y
unidades}\label{ia-para-planificaciuxf3n-de-clases-y-unidades}

La planificación es una de las tareas que más tiempo demanda en el
trabajo docente. La inteligencia artificial puede funcionar aquí como un
asistente de borradores: ayuda a proponer ideas, ordenar secuencias y
sugerir actividades, pero la decisión final sobre qué, cómo y cuándo
enseñar sigue siendo siempre pedagógica y profesional, basada en las
Bases Curriculares y en el conocimiento que cada docente tiene de su
curso.

Usada de manera crítica y situada, la IA puede ahorrar tiempo en la fase
inicial de la planificación, para que las y los profesores puedan
dedicar más energía a la reflexión pedagógica y a la adaptación fina de
las propuestas a su realidad escolar concreta.

\textbf{Usos posibles:}

\begin{itemize}
\tightlist
\item
  Proponer borradores de secuencias de clases para una unidad
  específica.\\
\item
  Sugerir objetivos de aprendizaje y actividades de inicio, desarrollo y
  cierre, alineadas con OA u objetivos priorizados.\\
\item
  Ayudar a alinear actividades con objetivos de aprendizaje definidos
  por el/la docente.\\
\item
  Ofrecer variantes de una misma clase según el tiempo disponible o el
  perfil del grupo.\\
\item
  Generar ideas iniciales para luego ser revisadas, corregidas y
  ajustadas por el equipo docente.
\end{itemize}

\textbf{Ejemplo de prompt para planificación de una unidad:}

\begin{Shaded}
\begin{Highlighting}[]
\NormalTok{Actúa como profesor/a de }\CommentTok{[}\OtherTok{ASIGNATURA}\CommentTok{]}\NormalTok{ en }\CommentTok{[}\OtherTok{NIVEL}\CommentTok{]}\NormalTok{ en una escuela }\CommentTok{[}\OtherTok{tipo de establecimiento: pública, subvencionada, particular}\CommentTok{]}\NormalTok{ de Chile.}

\NormalTok{Tu objetivo es ayudarme a elaborar un borrador de planificación para UNA unidad didáctica sobre }\CommentTok{[}\OtherTok{TEMA O CONTENIDO}\CommentTok{]}\NormalTok{.}

\NormalTok{Contexto:  }
\SpecialStringTok{{-} }\NormalTok{Curso: }\CommentTok{[}\OtherTok{por ejemplo, 8º básico, 40 estudiantes}\CommentTok{]}\NormalTok{.  }
\SpecialStringTok{{-} }\NormalTok{Marco curricular: Bases Curriculares chilenas para }\CommentTok{[}\OtherTok{ASIGNATURA}\CommentTok{]}\NormalTok{; OA a trabajar: }\CommentTok{[}\OtherTok{pegar aquí si es posible}\CommentTok{]}\NormalTok{.  }
\SpecialStringTok{{-} }\NormalTok{Tiempo disponible: }\CommentTok{[}\OtherTok{número aproximado de clases u horas pedagógicas}\CommentTok{]}\NormalTok{.  }
\SpecialStringTok{{-} }\NormalTok{Características del grupo: }\CommentTok{[}\OtherTok{curso diverso, presencia de PIE, diferencias importantes en niveles de logro, etc.}\CommentTok{]}\NormalTok{.  }

\NormalTok{Instrucciones:  }
\SpecialStringTok{{-} }\NormalTok{Propón una secuencia de }\CommentTok{[}\OtherTok{X}\CommentTok{]}\NormalTok{ clases para esta unidad.  }
\SpecialStringTok{{-} }\NormalTok{Para cada clase, indica: objetivo específico, actividad principal de aprendizaje, recurso clave a utilizar y forma sencilla de evaluación (por ejemplo, pregunta de salida, ejercicio breve).  }
\SpecialStringTok{{-} }\NormalTok{Incluye al menos una sugerencia de ajuste o variación para estudiantes que requieran más apoyo y otra para quienes puedan profundizar.  }

\NormalTok{Formato de salida:  }
\SpecialStringTok{{-} }\NormalTok{Tabla con columnas: Clase – Objetivo específico – Actividad principal – Recurso – Evaluación sugerida – Ajustes/variantes.  }

\NormalTok{Criterios y restricciones:  }
\SpecialStringTok{{-} }\NormalTok{Usa lenguaje claro y concreto, adecuado al contexto escolar chileno.  }
\SpecialStringTok{{-} }\NormalTok{No inventes contenidos fuera de las Bases Curriculares; organiza y ejemplifica a partir de los OA indicados.  }
\SpecialStringTok{{-} }\NormalTok{No propongas actividades que requieran recursos tecnológicos que no haya mencionado.  }
\end{Highlighting}
\end{Shaded}

\section{IA para generación y adaptación de
materiales}\label{ia-para-generaciuxf3n-y-adaptaciuxf3n-de-materiales}

Otra tarea muy demandante en el trabajo docente es la elaboración,
revisión y adaptación de materiales: guías, ejercicios, textos,
ejemplos, preguntas, lecturas, imagenes, entre otros. La IA puede
convertirse en un apoyo relevante para producir primeros borradores de
estos recursos, que luego la/el docente revisa, ajusta y contextualiza
según su curso y establecimiento.

Esto puede ahorrar tiempo en la redacción inicial y abrir posibilidades
para ofrecer más variedad de ejemplos y actividades, manteniendo siempre
el criterio profesional sobre la pertinencia pedagógica y el nivel de
dificultad.

\textbf{Usos posibles:}

\begin{itemize}
\tightlist
\item
  Generar ejercicios adicionales a partir de un contenido ya definido.\\
\item
  Adaptar un mismo texto a distintos niveles de complejidad (más
  sencillo o más desafiante).\\
\item
  Crear versiones alternativas de una actividad para diferentes niveles
  de logro.\\
\item
  Transformar un contenido en distintos formatos: preguntas de opción
  múltiple, desarrollo, verdadero/falso, organizadores gráficos, etc.\\
\item
  Reformular instrucciones para que sean más claras para estudiantes de
  distintos cursos.
\end{itemize}

\textbf{Ejemplo de prompt para generar y adaptar materiales:}

\begin{Shaded}
\begin{Highlighting}[]
\NormalTok{Actúa como profesor/a de }\CommentTok{[}\OtherTok{ASIGNATURA}\CommentTok{]}\NormalTok{ con experiencia en diseño de materiales didácticos para escuelas públicas chilenas.}

\NormalTok{Tu objetivo es ayudarme a generar y adaptar materiales para trabajar el tema }\CommentTok{[}\OtherTok{TEMA ESPECÍFICO}\CommentTok{]}\NormalTok{ con un curso de }\CommentTok{[}\OtherTok{NIVEL}\CommentTok{]}\NormalTok{ en el marco de las Bases Curriculares.}

\NormalTok{Contexto:  }
\SpecialStringTok{{-} }\NormalTok{Tipo de establecimiento: }\CommentTok{[}\OtherTok{público, subvencionado, técnico{-}profesional, rural, urbano, etc.}\CommentTok{]}\NormalTok{.  }
\SpecialStringTok{{-} }\NormalTok{Características del grupo: }\CommentTok{[}\OtherTok{tamaño del curso, diversidad de niveles, presencia de PIE, etc.}\CommentTok{]}\NormalTok{.  }
\SpecialStringTok{{-} }\NormalTok{Recursos disponibles: }\CommentTok{[}\OtherTok{pizarra, cuadernos, proyector, fotocopias, laboratorio, etc.}\CommentTok{]}\NormalTok{.  }

\NormalTok{Instrucciones:  }
\SpecialStringTok{1. }\NormalTok{Propón }\CommentTok{[}\OtherTok{NÚMERO}\CommentTok{]}\NormalTok{ ejercicios o actividades breves para trabajar este contenido.  }
\SpecialStringTok{   {-} }\NormalTok{Para cada actividad, indica: propósito, descripción breve y tiempo estimado.  }
\SpecialStringTok{2. }\NormalTok{Luego, elige UNA de las actividades y genera:  }
\SpecialStringTok{   {-} }\NormalTok{Una versión más sencilla para estudiantes que necesitan mayor apoyo.  }
\SpecialStringTok{   {-} }\NormalTok{Una versión más desafiante para estudiantes que avanzan más rápido.  }

\NormalTok{Formato de salida:  }
\SpecialStringTok{{-} }\NormalTok{Primero, lista numerada de actividades con su propósito, descripción y tiempo.  }
\SpecialStringTok{{-} }\NormalTok{Después, subtítulos “Versión más sencilla” y “Versión más desafiante” para la actividad elegida, explicadas en no más de 6 líneas cada una.  }

\NormalTok{Criterios y restricciones:  }
\SpecialStringTok{{-} }\NormalTok{Usa lenguaje claro, adecuado al nivel del curso.  }
\SpecialStringTok{{-} }\NormalTok{No incluyas recursos costosos ni que requieran conexión a internet permanente.  }
\SpecialStringTok{{-} }\NormalTok{No cambies el contenido central; solo ajusta la complejidad y el tipo de apoyo.  }
\end{Highlighting}
\end{Shaded}

\section{IA para diversificar actividades según
curso}\label{ia-para-diversificar-actividades-seguxfan-curso}

En un mismo curso suelen convivir estudiantes con ritmos de aprendizaje
distintos, intereses variados, trayectorias escolares diversas y, en
muchos casos, con participación de programas de integración escolar
(PIE). La IA puede ayudar a generar variantes de una misma actividad
para atender esta diversidad, siempre que la/el docente mantenga el
control sobre los objetivos y la evaluación.

El papel de la IA no es decidir quién hace qué, sino ofrecer un abanico
de opciones que la/el profesor/a puede asignar, combinar o adaptar a su
grupo, evitando etiquetar estudiantes y resguardando una mirada
inclusiva.

\textbf{Usos posibles:}

\begin{itemize}
\tightlist
\item
  Diseñar tres niveles de dificultad para una misma actividad (más
  guiada, intermedia, profundización).\\
\item
  Proponer actividades que privilegien diferentes canales: oral,
  escrito, visual, manipulativo.\\
\item
  Generar desafíos extra para estudiantes que avanzan más rápido.\\
\item
  Sugerir apoyos adicionales (preguntas guía, ejemplos resueltos,
  organizadores gráficos) para quienes requieren más acompañamiento.
\end{itemize}

\textbf{Ejemplo de prompt para diversificar una actividad base:}

\begin{Shaded}
\begin{Highlighting}[]
\NormalTok{Actúa como especialista en diferenciación pedagógica e inclusión educativa en contexto escolar chileno.}

\NormalTok{Tu objetivo es ayudarme a diversificar una actividad sobre }\CommentTok{[}\OtherTok{TEMA}\CommentTok{]}\NormalTok{ para un curso de }\CommentTok{[}\OtherTok{NIVEL}\CommentTok{]}\NormalTok{, manteniendo el mismo contenido central.}

\NormalTok{Contexto:  }
\SpecialStringTok{{-} }\NormalTok{Tipo de establecimiento: }\CommentTok{[}\OtherTok{público/subvencionado/particular}\CommentTok{]}\NormalTok{.  }
\SpecialStringTok{{-} }\NormalTok{Curso diverso en niveles de logro; hay estudiantes que requieren más apoyo y otros que avanzan más rápido.  }
\SpecialStringTok{{-} }\NormalTok{Recursos disponibles: }\CommentTok{[}\OtherTok{pizarra, cuadernos, proyector, impresora, etc.}\CommentTok{]}\NormalTok{.  }

\NormalTok{Actividad base:  }
\CommentTok{[}\OtherTok{PEGAR AQUÍ LA ACTIVIDAD ORIGINAL QUE YA DISEÑÓ EL/LA DOCENTE}\CommentTok{]}\NormalTok{.}

\NormalTok{Instrucciones:  }
\SpecialStringTok{1. }\NormalTok{A partir de la actividad base, propone tres versiones:  }
\SpecialStringTok{   {-} }\NormalTok{Una “Versión más guiada” para estudiantes que necesitan mayor apoyo.  }
\SpecialStringTok{   {-} }\NormalTok{Una “Versión intermedia” para la mayoría del curso.  }
\SpecialStringTok{   {-} }\NormalTok{Una “Versión de profundización” para quienes pueden avanzar más rápido.  }
\SpecialStringTok{2. }\NormalTok{Para cada versión, indica:  }
\SpecialStringTok{   {-} }\NormalTok{Propósito.  }
\SpecialStringTok{   {-} }\NormalTok{Pasos principales de la actividad.  }
\SpecialStringTok{   {-} }\NormalTok{Tipo de apoyo o desafío que se incluye (ejemplos, preguntas guía, uso de material concreto, etc.).  }

\NormalTok{Formato de salida:  }
\SpecialStringTok{{-} }\NormalTok{Subtítulos: “Versión más guiada”, “Versión intermedia” y “Versión de profundización”.  }
\SpecialStringTok{{-} }\NormalTok{Bajo cada subtítulo, un breve párrafo de 6 a 8 líneas describiendo la propuesta.  }

\NormalTok{Criterios:  }
\SpecialStringTok{{-} }\NormalTok{Mantén el mismo contenido central en las tres versiones.  }
\SpecialStringTok{{-} }\NormalTok{Usa lenguaje claro y respetuoso, sin etiquetar a estudiantes como “buenos” o “malos”.  }
\SpecialStringTok{{-} }\NormalTok{No propongas actividades que dependan de tecnologías que no se han mencionado.  }
\end{Highlighting}
\end{Shaded}

\section{IA para organizar el trabajo cotidiano (síntesis,
comunicaciones, actas,
etc.)}\label{ia-para-organizar-el-trabajo-cotidiano-suxedntesis-comunicaciones-actas-etc.}

Además de las tareas directamente pedagógicas, el trabajo docente
incluye una importante carga administrativa y de organización: lectura
de documentos extensos, elaboración de actas, sistematización de
acuerdos, redacción de comunicaciones a familias, entre otras. La IA
puede ser un apoyo para sintetizar, ordenar y redactar borradores,
siempre que se tenga especial cuidado de no exponer datos sensibles de
estudiantes o colegas.

Utilizada con resguardos, la IA puede ayudar a disminuir el tiempo
dedicado a la redacción inicial de estos documentos, de modo que las y
los profesores puedan concentrarse en la toma de decisiones, la
coordinación con sus equipos y la atención directa a estudiantes.

\textbf{Usos posibles:}

\begin{itemize}
\tightlist
\item
  Resumir documentos largos, orientaciones o actas en síntesis breves y
  claras.\\
\item
  Transformar un conjunto de notas dispersas en listas de acuerdos o
  tareas.\\
\item
  Proponer borradores de comunicaciones a familias, que luego la/el
  docente revisa y ajusta.\\
\item
  Sugerir estructuras de actas o pautas para registrar acuerdos de
  reuniones.\\
\item
  Ordenar ideas para proyectos, unidades o talleres en esquemas más
  claros.
\end{itemize}

\textbf{Ejemplo de prompt para síntesis y organización de información:}

\begin{Shaded}
\begin{Highlighting}[]
\NormalTok{Actúa como asistente de organización de trabajo docente en una escuela pública de Chile.}

\NormalTok{Tu objetivo es ayudarme a sintetizar y ordenar información para facilitar mi trabajo.}

\NormalTok{Contexto:  }
\SpecialStringTok{{-} }\NormalTok{Soy profesor/a de }\CommentTok{[}\OtherTok{NIVEL/ASIGNATURA}\CommentTok{]}\NormalTok{.  }
\SpecialStringTok{{-} }\NormalTok{Necesito convertir un conjunto de notas e información en un resumen claro y una lista de acuerdos o tareas.  }

\NormalTok{Instrucciones:  }
\SpecialStringTok{1. }\NormalTok{A partir del texto que pegaré a continuación, genera:  }
\SpecialStringTok{   {-} }\NormalTok{Un resumen breve de máximo 150 palabras.  }
\SpecialStringTok{   {-} }\NormalTok{Una lista de 5 a 7 acuerdos o tareas concretas.  }
\SpecialStringTok{2. }\NormalTok{Si detectas información poco clara o contradictoria, señálalo al final en un apartado de observaciones.  }

\NormalTok{Texto a sintetizar:  }
\CommentTok{[}\OtherTok{PEGAR AQUÍ NOTAS, ACTA O DOCUMENTO SIN DATOS SENSIBLES}\CommentTok{]}\NormalTok{.  }

\NormalTok{Formato de salida:  }
\SpecialStringTok{{-} }\NormalTok{Subtítulo “Resumen breve” y el resumen en un párrafo.  }
\SpecialStringTok{{-} }\NormalTok{Subtítulo “Lista de acuerdos o tareas” y la lista numerada.  }
\SpecialStringTok{{-} }\NormalTok{Si corresponde, subtítulo “Observaciones” con 2 a 3 líneas.  }

\NormalTok{Criterios y restricciones:  }
\SpecialStringTok{{-} }\NormalTok{No inventes acuerdos ni información que no esté en el texto.  }
\SpecialStringTok{{-} }\NormalTok{Usa lenguaje claro y profesional.  }
\SpecialStringTok{{-} }\NormalTok{No incluyas nombres propios ni datos personales, aunque aparezcan en el texto original; reemplázalos por descripciones generales (por ejemplo, “un/a estudiante”, “un/a apoderado/a”).  }
\end{Highlighting}
\end{Shaded}

\textbf{Ejemplo de prompt para redactar un comunicado breve a familias:}

\begin{Shaded}
\begin{Highlighting}[]
\NormalTok{Actúa como profesor/a jefe con experiencia en comunicación clara y respetuosa con familias en el sistema escolar chileno.}

\NormalTok{Tu objetivo es ayudarme a redactar un comunicado breve para apoderadas y apoderados sobre }\CommentTok{[}\OtherTok{TEMA: por ejemplo, reunión, actividad especial, cambio de horario}\CommentTok{]}\NormalTok{.}

\NormalTok{Contexto:  }
\SpecialStringTok{{-} }\NormalTok{Curso: }\CommentTok{[}\OtherTok{por ejemplo, 5º básico}\CommentTok{]}\NormalTok{.  }
\SpecialStringTok{{-} }\NormalTok{Tipo de establecimiento: }\CommentTok{[}\OtherTok{público/subvencionado/particular}\CommentTok{]}\NormalTok{.  }
\SpecialStringTok{{-} }\NormalTok{Medio de envío: }\CommentTok{[}\OtherTok{agenda, correo electrónico, WhatsApp}\CommentTok{]}\NormalTok{.  }

\NormalTok{Instrucciones:  }
\SpecialStringTok{{-} }\NormalTok{Redacta un texto de máximo 180 palabras.  }
\SpecialStringTok{{-} }\NormalTok{Usa un tono cercano, respetuoso y profesional.  }
\SpecialStringTok{{-} }\NormalTok{Explica de forma sencilla qué ocurrirá, cuándo, dónde y por qué es importante la actividad o la información.  }
\SpecialStringTok{{-} }\NormalTok{Incluye, si corresponde, qué se espera de las familias (asistencia, autorización, envío de materiales, etc.).  }

\NormalTok{Formato de salida:  }
\SpecialStringTok{{-} }\NormalTok{Texto continuo, listo para copiar y pegar en el medio de comunicación indicado.  }

\NormalTok{Criterios y restricciones:  }
\SpecialStringTok{{-} }\NormalTok{Evita tecnicismos; si mencionas “inteligencia artificial” u otros conceptos, explícalos en palabras simples.  }
\SpecialStringTok{{-} }\NormalTok{No incluyas datos personales ni información que no te haya proporcionado.  }
\SpecialStringTok{{-} }\NormalTok{Mantén un enfoque colaborativo, reconociendo el rol de las familias en el proceso educativo.  }
\end{Highlighting}
\end{Shaded}

En todos estos casos, la IA se utiliza como apoyo para organizar,
sintetizar y redactar, pero la revisión final, la adecuación al contexto
y la decisión sobre qué se envía o se registra siguen siendo
responsabilidad profesional de la/el docente y de los equipos escolares.

\bookmarksetup{startatroot}

\chapter{Usos concretos para evaluación y
retroalimentación}\label{usos-concretos-para-evaluaciuxf3n-y-retroalimentaciuxf3n}

\section{IA para proponer ítems, rúbricas y
criterios}\label{ia-para-proponer-uxedtems-ruxfabricas-y-criterios}

El diseño de evaluaciones es una de las tareas más exigentes del trabajo
docente: requiere tiempo para formular buenas preguntas, pensar
criterios claros y construir instrumentos alineados con los objetivos de
aprendizaje. La inteligencia artificial puede apoyar este proceso como
\textbf{generador de primeros borradores}, que luego la/el docente
revisa, ajusta y valida según las Bases Curriculares chilenas y el
proyecto educativo de su establecimiento.

Usada críticamente, la IA ayuda a ganar tiempo en la fase inicial de
diseño, a ampliar el repertorio de ejemplos y a explicitar criterios,
sin sustituir en ningún caso las decisiones profesionales sobre qué y
cómo evaluar.

\textbf{Usos posibles:}

\begin{itemize}
\tightlist
\item
  Proponer \textbf{ítems de evaluación} (alternativa, desarrollo,
  verdadero/falso, preguntas abiertas) coherentes con OA u objetivos
  priorizados.
\item
  Sugerir \textbf{rúbricas simples o pautas de corrección} para trabajos
  escritos, proyectos, presentaciones orales, etc.
\item
  Ayudar a \textbf{redactar criterios de evaluación en lenguaje claro}
  para estudiantes y familias.
\item
  Ofrecer \textbf{variantes de una misma pregunta} con distintos niveles
  de dificultad.
\item
  Generar \textbf{bancos iniciales de preguntas} que luego el equipo
  docente revisa, selecciona y ajusta.
\end{itemize}

\textbf{Ejemplo de prompt para proponer ítems, rúbricas y criterios:}

\begin{Shaded}
\begin{Highlighting}[]
\NormalTok{Actúa como asesor/a en evaluación formativa en el sistema escolar chileno.}

\NormalTok{Tu objetivo es ayudarme a diseñar una evaluación breve y una rúbrica sencilla para }\CommentTok{[}\OtherTok{TIPO DE TAREA: por ejemplo, informe escrito, presentación oral, proyecto grupal}\CommentTok{]}\NormalTok{ en }\CommentTok{[}\OtherTok{ASIGNATURA}\CommentTok{]}\NormalTok{ para }\CommentTok{[}\OtherTok{NIVEL}\CommentTok{]}\NormalTok{.}

\NormalTok{Contexto:}
\SpecialStringTok{{-} }\NormalTok{Curso: }\CommentTok{[}\OtherTok{por ejemplo, 8º básico, 40 estudiantes}\CommentTok{]}\NormalTok{.}
\SpecialStringTok{{-} }\NormalTok{Tipo de establecimiento: }\CommentTok{[}\OtherTok{público/subvencionado/particular}\CommentTok{]}\NormalTok{.}
\SpecialStringTok{{-} }\NormalTok{Contenidos y habilidades a evaluar: }\CommentTok{[}\OtherTok{describir brevemente o pegar OA de las Bases Curriculares}\CommentTok{]}\NormalTok{.}
\SpecialStringTok{{-} }\NormalTok{Tiempo estimado para la tarea: }\CommentTok{[}\OtherTok{por ejemplo, 2 clases}\CommentTok{]}\NormalTok{.}

\NormalTok{Instrucciones:}
\SpecialStringTok{1. }\NormalTok{Propón entre 5 y 8 ítems o tareas concretas que permitan evaluar estos contenidos y habilidades.}
\SpecialStringTok{2. }\NormalTok{Diseña una rúbrica sencilla con 3 o 4 criterios (por ejemplo, dominio de contenidos, claridad, organización, uso de recursos) y 4 niveles de logro.}
\SpecialStringTok{3. }\NormalTok{Redacta los criterios y descriptores en lenguaje claro, comprensible para estudiantes y familias.}
\SpecialStringTok{4. }\NormalTok{Incluye una breve nota final con recomendaciones para usar esta rúbrica en clave formativa (por ejemplo, para retroalimentar y no solo calificar).}

\NormalTok{Formato de salida:}
\SpecialStringTok{{-} }\NormalTok{Primero, lista numerada de ítems o tareas.}
\SpecialStringTok{{-} }\NormalTok{Luego, tabla con la rúbrica (filas = criterios, columnas = niveles de logro).}
\SpecialStringTok{{-} }\NormalTok{Finalmente, una sección breve llamada “Recomendaciones para el uso formativo”.}

\NormalTok{Criterios y restricciones:}
\SpecialStringTok{{-} }\NormalTok{Usa lenguaje claro y adecuado al nivel del curso.}
\SpecialStringTok{{-} }\NormalTok{No inventes contenidos fuera de las Bases Curriculares; mantente dentro de lo que se ha trabajado en clase.}
\SpecialStringTok{{-} }\NormalTok{No incluyas ejemplos que requieran recursos tecnológicos que no he mencionado.}
\end{Highlighting}
\end{Shaded}

\section{Ejemplos de retroalimentación formativa apoyada por
IA}\label{ejemplos-de-retroalimentaciuxf3n-formativa-apoyada-por-ia}

La retroalimentación formativa es una de las prácticas más potentes para
el aprendizaje, pero también una de las más demandantes en tiempo. La IA
puede colaborar generando \textbf{borradores de comentarios} que la/el
docente revisa, ajusta y personaliza para sus estudiantes, manteniendo
el foco en criterios claros y en orientaciones concretas de mejora.

Más que sustituir la voz del profesor o profesora, la IA puede ayudar a
encontrar palabras claras, respetuosas y específicas, especialmente
cuando hay muchos trabajos que retroalimentar en poco tiempo.

\textbf{Usos posibles:}

\begin{itemize}
\tightlist
\item
  Redactar \textbf{comentarios modelo} para distintos niveles de logro,
  a partir de una rúbrica o pauta.
\item
  Proponer \textbf{frases de retroalimentación centradas en el proceso},
  no solo en el resultado.
\item
  Sugerir \textbf{preguntas que inviten a la reflexión} del/la
  estudiante sobre su propio trabajo.
\item
  Generar \textbf{versiones más breves o más extensas} de un comentario
  según el contexto (agenda, plataforma, conversación presencial).
\item
  Apoyar la elaboración de \textbf{retroalimentación escrita para
  estudiantes con distintas necesidades} (por ejemplo, lenguaje más
  sencillo).
\end{itemize}

\textbf{Ejemplo de prompt para generar retroalimentación formativa:}

\begin{Shaded}
\begin{Highlighting}[]
\NormalTok{Actúa como asesor/a en evaluación formativa y retroalimentación para el sistema escolar chileno.}

\NormalTok{Tu objetivo es ayudarme a redactar comentarios de retroalimentación formativa para estudiantes de }\CommentTok{[}\OtherTok{NIVEL}\CommentTok{]}\NormalTok{ que han realizado la siguiente tarea:}
\CommentTok{[}\OtherTok{DESCRIBIR BREVE LA TAREA: por ejemplo, “ensayo argumentativo sobre el uso de celulares en el aula”}\CommentTok{]}\NormalTok{.}

\NormalTok{Contexto:}
\SpecialStringTok{{-} }\NormalTok{Asignatura: }\CommentTok{[}\OtherTok{ASIGNATURA}\CommentTok{]}\NormalTok{.}
\SpecialStringTok{{-} }\NormalTok{Tipo de establecimiento: }\CommentTok{[}\OtherTok{público/subvencionado/particular}\CommentTok{]}\NormalTok{.}
\SpecialStringTok{{-} }\NormalTok{Criterios de evaluación: }\CommentTok{[}\OtherTok{pegar criterios o rúbrica resumida}\CommentTok{]}\NormalTok{.}
\SpecialStringTok{{-} }\NormalTok{Necesito ejemplos para tres niveles de desempeño: inicial, intermedio y avanzado.}

\NormalTok{Instrucciones:}
\SpecialStringTok{1. }\NormalTok{Propón 3 comentarios de retroalimentación formativa:}
\SpecialStringTok{   {-} }\NormalTok{Uno para desempeño inicial.}
\SpecialStringTok{   {-} }\NormalTok{Uno para desempeño intermedio.}
\SpecialStringTok{   {-} }\NormalTok{Uno para desempeño avanzado.}
\SpecialStringTok{2. }\NormalTok{Cada comentario debe:}
\SpecialStringTok{   {-} }\NormalTok{Mencionar al menos un aspecto logrado.}
\SpecialStringTok{   {-} }\NormalTok{Señalar con claridad qué se puede mejorar.}
\SpecialStringTok{   {-} }\NormalTok{Sugerir un próximo paso concreto para el/la estudiante.}
\SpecialStringTok{3. }\NormalTok{Redacta los comentarios en segunda persona (“tú”) y en lenguaje cercano y respetuoso.}

\NormalTok{Formato de salida:}
\SpecialStringTok{{-} }\NormalTok{Subtítulos: “Desempeño inicial”, “Desempeño intermedio”, “Desempeño avanzado”.}
\SpecialStringTok{{-} }\NormalTok{Bajo cada subtítulo, un comentario de máximo 6 líneas.}

\NormalTok{Criterios y restricciones:}
\SpecialStringTok{{-} }\NormalTok{No uses un tono punitivo ni culposo.}
\SpecialStringTok{{-} }\NormalTok{Evita tecnicismos; prioriza explicaciones sencillas.}
\SpecialStringTok{{-} }\NormalTok{Mantén el foco en el aprendizaje y la mejora, no solo en la calificación.}
\end{Highlighting}
\end{Shaded}

\section{IA como apoyo a la autorregulación del
aprendizaje}\label{ia-como-apoyo-a-la-autorregulaciuxf3n-del-aprendizaje}

La autorregulación del aprendizaje implica que las y los estudiantes
puedan planificar, monitorear y evaluar su propio trabajo. La IA puede
contribuir ofreciendo \textbf{andamiajes} para que revisen sus
producciones, comparen con criterios de calidad y tomen decisiones sobre
cómo mejorar, siempre bajo acompañamiento docente y con resguardos
claros.

El énfasis no está en que la IA ``corrija'' por ellos, sino en que
entregue preguntas guía, listas de chequeo y sugerencias que ayuden a
los estudiantes a mirar críticamente lo que hicieron y a hacerse
responsables de sus procesos de aprendizaje.

\textbf{Usos posibles:}

\begin{itemize}
\tightlist
\item
  Generar \textbf{listas de verificación (checklists)} para que
  estudiantes revisen sus trabajos antes de entregarlos.
\item
  Proponer \textbf{preguntas de autoevaluación} alineadas con criterios
  o rúbricas.
\item
  Sugerir \textbf{estrategias de estudio o de mejora} a partir de
  dificultades detectadas.
\item
  Transformar criterios de evaluación en \textbf{lenguaje amigable para
  estudiantes}, que puedan usar como guía.
\item
  Crear \textbf{esquemas de planificación} (por ejemplo, pasos para
  desarrollar un proyecto) que los estudiantes puedan completar.
\end{itemize}

\textbf{Ejemplo de prompt para apoyar la autorregulación del
aprendizaje:}

\begin{Shaded}
\begin{Highlighting}[]
\NormalTok{Actúa como orientador/a pedagógico/a especializado/a en autorregulación del aprendizaje en educación básica/media.}

\NormalTok{Tu objetivo es ayudarme a generar apoyos para que estudiantes de }\CommentTok{[}\OtherTok{NIVEL}\CommentTok{]}\NormalTok{ se autoevalúen y mejoren su trabajo en }\CommentTok{[}\OtherTok{TIPO DE TAREA: por ejemplo, informe escrito, proyecto de ciencias, presentación oral}\CommentTok{]}\NormalTok{.}

\NormalTok{Contexto:}
\SpecialStringTok{{-} }\NormalTok{Asignatura: }\CommentTok{[}\OtherTok{ASIGNATURA}\CommentTok{]}\NormalTok{.}
\SpecialStringTok{{-} }\NormalTok{Tipo de establecimiento: }\CommentTok{[}\OtherTok{público/subvencionado/particular}\CommentTok{]}\NormalTok{.}
\SpecialStringTok{{-} }\NormalTok{Criterios de evaluación que usamos: }\CommentTok{[}\OtherTok{describir brevemente o pegar criterios/rúbrica}\CommentTok{]}\NormalTok{.}

\NormalTok{Instrucciones:}
\SpecialStringTok{1. }\NormalTok{Elabora una lista de verificación (“checklist”) que los estudiantes puedan usar antes de entregar su trabajo.}
\SpecialStringTok{   {-} }\NormalTok{Máximo 10 ítems, redactados en primera persona (“revisé que…”, “me aseguré de…”).}
\SpecialStringTok{2. }\NormalTok{Propón 5 preguntas de autoevaluación que les ayuden a reflexionar sobre su proceso (no solo sobre la nota).}
\SpecialStringTok{3. }\NormalTok{Sugiere 3 ideas de “próximos pasos” que puedan tomar si se dan cuenta de que necesitan mejorar.}

\NormalTok{Formato de salida:}
\SpecialStringTok{{-} }\NormalTok{Sección “Checklist de revisión antes de entregar” con la lista numerada.}
\SpecialStringTok{{-} }\NormalTok{Sección “Preguntas para autoevaluar mi trabajo” con 5 preguntas.}
\SpecialStringTok{{-} }\NormalTok{Sección “Qué puedo hacer para mejorar” con 3 sugerencias breves.}

\NormalTok{Criterios y restricciones:}
\SpecialStringTok{{-} }\NormalTok{Usa lenguaje claro, dirigido a estudiantes de }\CommentTok{[}\OtherTok{NIVEL}\CommentTok{]}\NormalTok{.}
\SpecialStringTok{{-} }\NormalTok{Evita frases negativas; formula los ítems y preguntas en clave de apoyo y mejora.}
\SpecialStringTok{{-} }\NormalTok{Aclara que estas herramientas son para que el/la estudiante mejore su trabajo, no para reemplazar la retroalimentación del/la docente.}
\end{Highlighting}
\end{Shaded}

\bookmarksetup{startatroot}

\chapter{IA e inclusión educativa}\label{ia-e-inclusiuxf3n-educativa}

\section{Ajuste de lenguaje, formato y extensión de
materiales}\label{ajuste-de-lenguaje-formato-y-extensiuxf3n-de-materiales}

En muchos cursos, el primer obstáculo para la participación de las y los
estudiantes es el \textbf{lenguaje y el formato} de los materiales:
textos muy largos, vocabulario técnico, consignas confusas o poco
accesibles. La IA puede apoyar como asistente de edición, ayudando a
simplificar, acortar, ampliar o reestructurar materiales sin cambiar el
contenido central definido por el equipo pedagógico.

Usada de forma crítica, la IA permite generar \textbf{versiones
paralelas} de un mismo recurso (más breve, más explicada, más visual,
etc.), de modo que el profesorado pueda elegir cuál se ajusta mejor a
las características de su curso, o bien ofrecer varias alternativas
dentro de la misma clase.

\textbf{Usos posibles:}

\begin{itemize}
\tightlist
\item
  Simplificar el vocabulario de un texto, manteniendo las ideas
  principales.
\item
  Reducir la extensión de un documento largo a una versión ``para
  estudiantes'' o ``para las familias''.
\item
  Reescribir consignas de actividades para que sean más claras y
  explícitas.
\item
  Cambiar el formato de presentación (párrafos → viñetas, cuadro
  comparativo, glosario, etc.).
\item
  Generar versiones con apoyos adicionales: ejemplos, definiciones
  breves, recordatorios de conceptos clave.
\end{itemize}

\textbf{Ejemplo de prompt para ajustar lenguaje, formato y extensión de
un material:}

\begin{Shaded}
\begin{Highlighting}[]
\NormalTok{Actúa como profesor/a de }\CommentTok{[}\OtherTok{ASIGNATURA}\CommentTok{]}\NormalTok{ con experiencia en educación inclusiva en escuelas públicas chilenas.}

\NormalTok{Tu objetivo es ayudarme a adaptar el siguiente texto para que sea más accesible a estudiantes de }\CommentTok{[}\OtherTok{NIVEL}\CommentTok{]}\NormalTok{ en el marco de las Bases Curriculares.}

\NormalTok{Contexto:}
\SpecialStringTok{{-} }\NormalTok{Curso diverso en niveles de lectura, con presencia de estudiantes del PIE.}
\SpecialStringTok{{-} }\NormalTok{El texto original es una explicación sobre }\CommentTok{[}\OtherTok{TEMA}\CommentTok{]}\NormalTok{ pensada para adultos.}
\SpecialStringTok{{-} }\NormalTok{Lo necesito para trabajarlo en una clase de }\CommentTok{[}\OtherTok{N°}\CommentTok{]}\NormalTok{ minutos.}

\NormalTok{Instrucciones:}
\SpecialStringTok{1. }\NormalTok{Genera una primera versión del texto:}
\SpecialStringTok{   {-} }\NormalTok{Con lenguaje más sencillo, sin perder las ideas principales.}
\SpecialStringTok{   {-} }\NormalTok{De extensión máxima de }\CommentTok{[}\OtherTok{N°}\CommentTok{]}\NormalTok{ palabras.}
\SpecialStringTok{   {-} }\NormalTok{Organizada en párrafos breves y con viñetas cuando sea útil.}
\SpecialStringTok{2. }\NormalTok{Genera una segunda versión:}
\SpecialStringTok{   {-} }\NormalTok{En formato de “ficha de estudio” con:}
\SpecialStringTok{     {-} }\NormalTok{Definiciones clave.}
\SpecialStringTok{     {-} }\NormalTok{Ejemplos cotidianos.}
\SpecialStringTok{     {-} }\NormalTok{3 preguntas de autoevaluación para estudiantes.}
\SpecialStringTok{3. }\NormalTok{Si consideras que falta algún concepto básico para que se entienda el texto, indícalo al final en 3–4 líneas, sin agregarlo dentro del texto principal.}

\NormalTok{Formato de salida:}
\SpecialStringTok{{-} }\NormalTok{Título: “Versión simplificada para }\CommentTok{[}\OtherTok{NIVEL}\CommentTok{]}\NormalTok{”.}
\SpecialStringTok{{-} }\NormalTok{Luego, la ficha de estudio bajo el título “Ficha de apoyo”.}

\NormalTok{Criterios y restricciones:}
\SpecialStringTok{{-} }\NormalTok{No cambies el contenido disciplinar central; solo ajusta lenguaje, formato y extensión.}
\SpecialStringTok{{-} }\NormalTok{No incluyas imágenes ni recursos que requieran conexión a internet.}
\SpecialStringTok{{-} }\NormalTok{Usa ejemplos cercanos a la realidad de estudiantes en escuelas chilenas.}
\end{Highlighting}
\end{Shaded}

\begin{center}\rule{0.5\linewidth}{0.5pt}\end{center}

\section{Ejemplos de adaptaciones para distintas trayectorias y
necesidades}\label{ejemplos-de-adaptaciones-para-distintas-trayectorias-y-necesidades}

En un mismo curso conviven estudiantes con trayectorias muy diversas:
quienes han cambiado de escuela con frecuencia, quienes participan en
programas de integración escolar, quienes tienen responsabilidades de
cuidado en sus hogares, quienes se están incorporando recientemente al
sistema escolar chileno, entre muchas otras realidades. La IA puede
ayudar a pensar \textbf{variantes de actividades y materiales} que
dialoguen con esta diversidad, siempre bajo el criterio profesional
docente y resguardando la confidencialidad de información sensible.

El propósito no es etiquetar estudiantes (``los buenos'', ``los malos'',
``los que saben, los que no''), sino contar con un abanico de opciones
que permita ajustar apoyos y desafíos sin perder de vista el currículum
común ni la pertenencia al grupo curso.

\textbf{Usos posibles:}

\begin{itemize}
\tightlist
\item
  Proponer actividades alternativas para estudiantes con ausencias
  prolongadas o reincorporaciones tardías.
\item
  Sugerir apoyos adicionales para quienes se están adaptando al idioma o
  al sistema escolar chileno.
\item
  Ajustar tareas para estudiantes que viven situaciones de cuidado,
  trabajo u otras responsabilidades.
\item
  Diseñar actividades que se puedan realizar tanto en el aula como en el
  hogar, según la realidad de cada estudiante.
\item
  Generar ejemplos y contextos cercanos a distintas realidades
  territoriales y culturales.
\end{itemize}

\textbf{Ejemplo de prompt para generar adaptaciones según trayectorias
diversas:}

\begin{Shaded}
\begin{Highlighting}[]
\NormalTok{Actúa como especialista en educación inclusiva y en diseño universal para el aprendizaje en contexto escolar chileno.}

\NormalTok{Tu objetivo es proponer adaptaciones de una actividad de }\CommentTok{[}\OtherTok{ASIGNATURA}\CommentTok{]}\NormalTok{ sobre }\CommentTok{[}\OtherTok{TEMA}\CommentTok{]}\NormalTok{ para estudiantes con trayectorias y necesidades diversas, manteniendo el mismo objetivo de aprendizaje.}

\NormalTok{Contexto:}
\SpecialStringTok{{-} }\NormalTok{Curso: }\CommentTok{[}\OtherTok{NIVEL, por ejemplo “8º básico”}\CommentTok{]}\NormalTok{ en escuela }\CommentTok{[}\OtherTok{pública/subvencionada/particular}\CommentTok{]}\NormalTok{.}
\SpecialStringTok{{-} }\NormalTok{En el grupo hay:}
\SpecialStringTok{  {-} }\NormalTok{Estudiantes que han tenido ausencias prolongadas.}
\SpecialStringTok{  {-} }\NormalTok{Estudiantes recién incorporados al sistema escolar chileno.}
\SpecialStringTok{  {-} }\NormalTok{Estudiantes que participan en el PIE.}
\SpecialStringTok{{-} }\NormalTok{Recursos disponibles: }\CommentTok{[}\OtherTok{pizarra, cuadernos, proyector, impresora, etc.}\CommentTok{]}\NormalTok{.}

\NormalTok{Actividad original:}
\CommentTok{[}\OtherTok{PEGAR AQUÍ LA DESCRIPCIÓN DE LA ACTIVIDAD DE CLASE QUE YA DISEÑÓ EL/LA DOCENTE}\CommentTok{]}\NormalTok{.}

\NormalTok{Instrucciones:}
\SpecialStringTok{1. }\NormalTok{Propón al menos 3 adaptaciones posibles de la actividad original:}
\SpecialStringTok{   {-} }\NormalTok{Adaptación A: pensada para estudiantes con ausencias prolongadas.}
\SpecialStringTok{   {-} }\NormalTok{Adaptación B: pensada para estudiantes recién incorporados al sistema escolar chileno.}
\SpecialStringTok{   {-} }\NormalTok{Adaptación C: pensada para estudiantes que requieren mayor apoyo estructurado (por ejemplo, del PIE).}
\SpecialStringTok{2. }\NormalTok{Para cada adaptación, indica:}
\SpecialStringTok{   {-} }\NormalTok{Propósito.}
\SpecialStringTok{   {-} }\NormalTok{Pasos principales de la actividad.}
\SpecialStringTok{   {-} }\NormalTok{Tipo de apoyo que se ofrece (andamiajes, ejemplos, materiales de apoyo, etc.).}
\SpecialStringTok{   {-} }\NormalTok{Cómo se vincula explícitamente con el mismo objetivo de aprendizaje de la actividad original.}

\NormalTok{Formato de salida:}
\SpecialStringTok{{-} }\NormalTok{Subtítulos: “Adaptación A”, “Adaptación B” y “Adaptación C”.}
\SpecialStringTok{{-} }\NormalTok{Bajo cada subtítulo, un párrafo de 8–10 líneas describiendo la propuesta.}

\NormalTok{Criterios y restricciones:}
\SpecialStringTok{{-} }\NormalTok{No cambies el objetivo de aprendizaje; solo ajusta la forma de abordarlo.}
\SpecialStringTok{{-} }\NormalTok{No incluyas datos personales ni ejemplos que permitan identificar estudiantes específicos.}
\SpecialStringTok{{-} }\NormalTok{Usa lenguaje claro y respetuoso, evitando etiquetas como “buenos/malos alumnos”.}
\end{Highlighting}
\end{Shaded}

\begin{center}\rule{0.5\linewidth}{0.5pt}\end{center}

\section{Posibilidades y límites en contextos de integración y educación
especial}\label{posibilidades-y-luxedmites-en-contextos-de-integraciuxf3n-y-educaciuxf3n-especial}

En contextos de integración y educación especial, la IA abre
oportunidades relevantes: permite generar materiales con distintos
niveles de apoyo, crear ejemplos personalizados, proponer formatos
alternativos (audio, viñetas, esquemas), entre otros. Sin embargo,
también existen límites importantes: la IA no conoce a las y los
estudiantes, no reemplaza la evaluación diagnóstica ni las decisiones
del equipo PIE, y puede reproducir sesgos o proponer estrategias poco
adecuadas si no se la orienta bien.

Por eso es clave usar estas herramientas como \textbf{soporte para el
trabajo del equipo educativo}, nunca como sustituto de la observación
profesional, de los acuerdos con las familias o de los planes
individualizados (PII, adecuaciones curriculares, etc.).

\textbf{Usos posibles:}

\begin{itemize}
\tightlist
\item
  Sugerir ideas de adaptaciones curriculares no significativas, que
  luego el equipo revisa y valida.
\item
  Generar materiales de apoyo complementarios (glosarios, tarjetas
  visuales, pasos de una rutina, etc.).
\item
  Proponer formas alternativas de demostrar aprendizaje (productos
  orales, visuales, manipulativos).
\item
  Sistematizar acuerdos del equipo PIE en un lenguaje claro para
  compartir con el resto del profesorado.
\item
  Redactar borradores de orientaciones generales para familias sobre
  cómo apoyar tareas en casa.
\end{itemize}

\textbf{Ejemplo de prompt para explorar posibilidades y límites en un
curso con PIE:}

\begin{Shaded}
\begin{Highlighting}[]
\NormalTok{Actúa como profesional de apoyo en un Programa de Integración Escolar (PIE) con experiencia en adecuaciones curriculares no significativas.}

\NormalTok{Tu objetivo es sugerir ideas de apoyo pedagógico para un curso de }\CommentTok{[}\OtherTok{NIVEL}\CommentTok{]}\NormalTok{ que trabaja el contenido }\CommentTok{[}\OtherTok{TEMA}\CommentTok{]}\NormalTok{ en la asignatura de }\CommentTok{[}\OtherTok{ASIGNATURA}\CommentTok{]}\NormalTok{, respetando el currículum común y el trabajo del equipo PIE.}

\NormalTok{Contexto:}
\SpecialStringTok{{-} }\NormalTok{Escuela pública con PIE.}
\SpecialStringTok{{-} }\NormalTok{En el curso hay estudiantes con:}
\SpecialStringTok{  {-} }\NormalTok{Dificultades específicas de aprendizaje.}
\SpecialStringTok{  {-} }\NormalTok{TDAH.}
\SpecialStringTok{  {-} }\NormalTok{Discapacidad intelectual leve.}
\SpecialStringTok{{-} }\NormalTok{El equipo docente ya definió los objetivos de aprendizaje para todo el curso.}

\NormalTok{Instrucciones:}
\SpecialStringTok{1. }\NormalTok{Propón 5 ideas de apoyos generales que puedan beneficiar a todo el curso (no solo a estudiantes del PIE), tales como:}
\SpecialStringTok{   {-} }\NormalTok{Ajustes en la presentación de la información.}
\SpecialStringTok{   {-} }\NormalTok{Estrategias de trabajo colaborativo.}
\SpecialStringTok{   {-} }\NormalTok{Formas alternativas de demostrar lo aprendido.}
\SpecialStringTok{2. }\NormalTok{Luego, sugiere 3 ejemplos de adecuaciones no significativas para el mismo contenido, aclarando:}
\SpecialStringTok{   {-} }\NormalTok{Qué se mantiene igual para todos.}
\SpecialStringTok{   {-} }\NormalTok{Qué se adapta (tiempo, formato, cantidad de ejercicios, etc.).}
\SpecialStringTok{3. }\NormalTok{Finalmente, incluye una breve sección “Límites y cuidados” con 4–5 puntos que recuerden:}
\SpecialStringTok{   {-} }\NormalTok{Que las decisiones deben ser revisadas por el equipo PIE.}
\SpecialStringTok{   {-} }\NormalTok{Que no se debe incluir información sensible de estudiantes en la IA.}

\NormalTok{Formato de salida:}
\SpecialStringTok{{-} }\NormalTok{Secciones con subtítulos: “Apoyos generales para todo el curso”, “Ejemplos de adecuaciones no significativas” y “Límites y cuidados”.}

\NormalTok{Criterios y restricciones:}
\SpecialStringTok{{-} }\NormalTok{No sugieras diagnósticos ni decisiones individuales sobre estudiantes.}
\SpecialStringTok{{-} }\NormalTok{Usa lenguaje respetuoso y centrado en apoyos, no en déficits.}
\SpecialStringTok{{-} }\NormalTok{Enfatiza que la IA es un apoyo para pensar estrategias, no un reemplazo de la evaluación profesional.}
\end{Highlighting}
\end{Shaded}

\bookmarksetup{startatroot}

\chapter{Riesgos, límites y criterios
éticos}\label{riesgos-luxedmites-y-criterios-uxe9ticos}

\section{Privacidad, sesgos y brechas de
acceso}\label{privacidad-sesgos-y-brechas-de-acceso}

El uso de herramientas de inteligencia artificial en la escuela implica
trabajar con datos, decisiones y recursos tecnológicos que no son
neutros. Por eso, antes de pensar en ``todo lo que la IA puede hacer'',
es clave preguntarse \textbf{qué riesgos abre} y cómo minimizarlos desde
la responsabilidad profesional docente.

En términos de privacidad, cualquier información que permita identificar
a una persona (nombre, RUT, curso específico, diagnóstico, dirección,
teléfono, correo, etc.) \textbf{no debiera ser ingresada} en
herramientas de IA abiertas. Tampoco es recomendable copiar actas
sensibles, listados completos de notas, informes psicológicos u otros
documentos similares. Cuando se requiera trabajar con ejemplos, es
preferible \textbf{anonimizar los datos} o usar situaciones ficticias.

Los modelos de IA también aprenden a partir de grandes volúmenes de
texto producidos en contextos desiguales, por lo que pueden reproducir
\textbf{sesgos de género, clase, raza, territorio, discapacidad}, entre
otros. Del mismo modo, su uso exige conectividad y dispositivos, lo que
puede profundizar las \textbf{brechas de acceso} entre establecimientos
y estudiantes.

Algunas orientaciones básicas:

\begin{itemize}
\tightlist
\item
  Evitar ingresar datos personales identificables de estudiantes,
  familias o colegas.
\item
  Trabajar con ejemplos anonimizados (``estudiante A'', ``apoderado'',
  ``curso de 6º básico'') o con situaciones ficticias.
\item
  Revisar críticamente los ejemplos y respuestas de la IA para detectar
  estereotipos o sesgos.
\item
  No tomar decisiones disciplinarias, diagnósticas o de atención
  individual basadas únicamente en lo que sugiere una IA.
\item
  Considerar las desigualdades de acceso a dispositivos y conectividad
  al diseñar actividades que involucren tecnología.
\end{itemize}

\section{Criterios pedagógicos para decidir cuándo usar
IA}\label{criterios-pedaguxf3gicos-para-decidir-cuuxe1ndo-usar-ia}

No toda tarea pedagógica necesita o se beneficia del uso de IA. Una
decisión profesional responsable considera \textbf{cuándo tiene sentido}
apoyarse en estas herramientas y cuándo es mejor prescindir de ellas. La
IA puede ser especialmente útil para tareas repetitivas, de borrador o
de exploración de ideas, mientras que las decisiones de sentido
pedagógico, de evaluación fina y de acompañamiento socioemocional
requieren la presencia y el juicio de la/el docente.

Algunos criterios posibles para decidir:

\begin{itemize}
\item
  \textbf{Pertinencia pedagógica}\\
  Preguntarse si la IA aporta algo que no podría hacerse de manera
  razonable con otros recursos, o si solo se está usando ``porque está
  de moda''.
\item
  \textbf{Ahorro de tiempo sin pérdida de calidad}\\
  Priorizar la IA en tareas de redacción inicial (borradores, propuestas
  de actividades, ejemplos) que luego se revisan y ajustan, no en la
  toma de decisiones evaluativas finales.
\item
  \textbf{Control docente del proceso}\\
  Asegurarse de que la/el profesor/a mantenga el control sobre
  objetivos, criterios de evaluación y decisiones de cierre, incluso
  cuando use IA para generar insumos.
\item
  \textbf{Equidad y acceso}\\
  Evaluar si el uso de IA generará nuevas desigualdades dentro del curso
  o entre cursos (por ejemplo, si solo algunos estudiantes pueden usar
  dispositivos).
\item
  \textbf{Transparencia y explicabilidad}\\
  Evitar decisiones importantes que no se puedan justificar frente a
  estudiantes y familias más allá de ``lo dijo la IA''.
\end{itemize}

\section{Cómo conversar con estudiantes y familias sobre el uso
responsable}\label{cuxf3mo-conversar-con-estudiantes-y-familias-sobre-el-uso-responsable}

La presencia de la IA en la vida cotidiana de niñas, niños y jóvenes
hace necesario abordar el tema \textbf{abiertamente} en la escuela. Más
que prohibir o celebrar sin matices, se trata de generar conversaciones
que permitan desarrollar criterios, pensar riesgos y oportunidades, y
acordar \textbf{normas de uso responsable}.

Con estudiantes, estas conversaciones pueden articularse con objetivos
de formación ciudadana, ética, orientación o asignaturas específicas.
Con familias, es importante transmitir información clara y sencilla,
evitando tecnicismos, para que puedan acompañar a sus hijos e hijas en
el uso de estas herramientas.

Algunas ideas clave para trabajar en aula y con la comunidad:

\begin{itemize}
\tightlist
\item
  Explicar en lenguaje simple qué es y qué no es la IA (no es
  ``inteligencia humana'', no ``sabe todo'', puede equivocarse).
\item
  Conversar sobre la importancia de \textbf{no compartir datos
  personales} y de preguntar siempre qué se hace con la información.
\item
  Trabajar ejemplos de \textbf{sesgos y errores} de la IA, mostrando por
  qué es necesario revisar y contrastar lo que propone.
\item
  Discutir los límites del uso de IA en tareas escolares: qué se
  considera apoyo legítimo y qué se considera copia o falta a la
  honestidad académica.
\item
  Construir junto al curso \textbf{acuerdos de uso responsable}, que
  luego puedan compartirse con las familias.
\end{itemize}

\bookmarksetup{startatroot}

\chapter{Buenas prácticas y pasos pequeños para
empezar}\label{buenas-pruxe1cticas-y-pasos-pequeuxf1os-para-empezar}

\section{Integrar IA sin aumentar la carga
laboral}\label{integrar-ia-sin-aumentar-la-carga-laboral}

Uno de los riesgos más frecuentes al incorporar nuevas herramientas es
que terminen aumentando, en vez de disminuir, la carga de trabajo
docente. El objetivo de este taller es justamente lo contrario: que la
IA ayude a ganar tiempo y no a sumar tareas. Por eso, el principio
central es \textbf{integrar la IA en procesos que ya existen}, como la
planificación, la elaboración de materiales o la redacción de
comunicaciones, en lugar de crear responsabilidades nuevas.

La idea no es ``hacer más cosas'' gracias a la IA, sino \textbf{hacer lo
imprescindible con menos desgaste}, manteniendo el foco en lo
pedagógico, en la relación con las y los estudiantes y en el trabajo
colaborativo entre docentes.

Formas prácticas de integrar IA sin aumentar la carga laboral:

\begin{itemize}
\tightlist
\item
  \textbf{Empezar por un solo proceso}\\
  Elegir un ámbito acotado donde la IA pueda ayudar (por ejemplo,
  redactar borradores de objetivos o de comunicaciones a familias) y
  trabajar solo allí al inicio.
\item
  \textbf{Usar la IA solo en la fase de borrador}\\
  Dejar en manos de la IA la redacción inicial, pero mantener en manos
  docentes la selección, corrección y adaptación. No es necesario
  revisar cada detalle perfecto: basta con que el borrador ahorre
  tiempo.
\item
  \textbf{Reutilizar prompts y estructuras}\\
  Guardar en un documento compartido los prompts que funcionen bien,
  para copiarlos y adaptarlos rápidamente sin tener que inventar desde
  cero cada vez.
\item
  \textbf{Poner límites de tiempo}\\
  Definir cuánto tiempo se destinará a ``trabajar con IA'' en una tarea
  (por ejemplo, 10--15 minutos por planificación), para evitar que el
  uso de la herramienta se vuelva un nuevo foco de desgaste.
\item
  \textbf{Ajustar expectativas}\\
  Recordar que la IA no va a resolver todos los problemas ni reemplazará
  el juicio pedagógico: es una herramienta más, que se usa cuando aporta
  y se deja de lado cuando no agrega valor.
\end{itemize}

\section{Documentar y compartir experiencias entre
docentes}\label{documentar-y-compartir-experiencias-entre-docentes}

Para que el uso de la IA se vuelva una práctica sostenible y no una
experiencia aislada, es clave \textbf{documentar qué funciona y qué no},
y compartirlo entre colegas. No se trata de informes extensos, sino de
registros simples que permitan aprender en conjunto y evitar repetir
errores.

La documentación también ayuda a resguardar la dimensión ética: dejar
constancia de cómo se usó la IA, con qué criterios y en qué partes del
proceso, facilita la reflexión y la rendición de cuentas frente a
estudiantes, familias y equipos directivos.

Formas simples de documentar y compartir:

\begin{itemize}
\tightlist
\item
  \textbf{Crear un documento compartido} (drive, intranet) con prompts
  que funcionaron bien, explicando en qué contexto se usaron y qué
  ajustes se hicieron.
\item
  \textbf{Guardar ejemplos de productos generados por IA}\\
  Planificaciones, actividades, rúbricas o comunicaciones, acompañados
  de comentarios sobre qué se ajustó o corrigió y qué resultó útil.
\item
  \textbf{Destinar unos minutos en consejos de profesores o reuniones de
  departamento}\\
  Compartir brevemente experiencias (``qué probé'', ``qué resultó'',
  ``qué no haría de nuevo'') en torno al uso de IA.
\item
  \textbf{Registrar acuerdos mínimos de uso responsable}\\
  Por ejemplo: no copiar y pegar sin revisar, no subir datos sensibles,
  siempre explicitar a estudiantes cuándo se usó IA y con qué fin.
\item
  \textbf{Invitar a estudiantes a comentar}\\
  En algunos casos, pedir retroalimentación a estudiantes sobre
  materiales o actividades apoyadas por IA puede ayudar a ajustar el uso
  de la herramienta.
\end{itemize}

\section{Checklist inicial para cada
establecimiento}\label{checklist-inicial-para-cada-establecimiento}

Para apoyar a escuelas y liceos que quieran comenzar a usar IA de manera
ordenada y cuidadosa, puede ser útil contar con un pequeño
\textbf{checklist institucional} que oriente los primeros pasos. No es
un protocolo rígido, sino una guía para conversar y tomar decisiones
informadas.

Checklist propuesto:

\begin{itemize}
\tightlist
\item[$\square$]
  \textbf{Objetivo claro}\\
  ¿Tenemos definido para qué procesos queremos usar IA (planificación,
  materiales, evaluación, comunicaciones) y para cuáles no?
\item[$\square$]
  \textbf{Acuerdos de uso responsable}\\
  ¿Existe un consenso básico sobre el resguardo de datos sensibles, el
  rol insustituible del juicio pedagógico y la necesidad de revisión
  humana?
\item[$\square$]
  \textbf{Selección de herramientas}\\
  ¿Sabemos qué herramientas de IA vamos a usar (por ejemplo, versiones
  institucionales o gratuitas) y cuáles quedan explícitamente excluidas?
\item[$\square$]
  \textbf{Formación mínima para docentes}\\
  ¿Hemos ofrecido algún espacio de capacitación o acompañamiento
  (taller, guía escrita, pareja pedagógica) para que el uso no dependa
  solo de ensayo y error individual?
\item[$\square$]
  \textbf{Biblioteca inicial de prompts y ejemplos}\\
  ¿Contamos con un set básico de prompts adaptados al contexto del
  establecimiento, que cualquier docente pueda tomar y ajustar?
\item[$\square$]
  \textbf{Espacios de evaluación y ajuste}\\
  ¿Tenemos previsto revisar, al cabo de algunos meses, qué está
  funcionando, qué no y qué ajustes éticos o pedagógicos son necesarios?
\item[$\square$]
  \textbf{Comunicación con estudiantes y familias}\\
  ¿El establecimiento ha definido cómo explicará el uso de IA en la
  escuela, qué propósitos tiene y qué resguardos se tomarán?
\end{itemize}

Este checklist puede servir como punto de partida para que cada
comunidad educativa discuta y construya sus propias pautas, manteniendo
siempre el foco en la mejora de las condiciones de trabajo docente y en
el aprendizaje de las y los estudiantes, antes que en la adopción
acrítica de nuevas tecnologías.

\bookmarksetup{startatroot}

\chapter{Extras}\label{extras}

\section{Errores frecuentes al usar IA y cómo
corregirlos}\label{errores-frecuentes-al-usar-ia-y-cuxf3mo-corregirlos}

En esta sección se presentan errores típicos al redactar \emph{prompts}
y maneras simples de mejorarlos desde la práctica docente.

\subsection{Pedir ``de todo un poco'' en una sola
instrucción}\label{pedir-de-todo-un-poco-en-una-sola-instrucciuxf3n}

\textbf{Ejemplo de error}

\begin{quote}
Hazme una planificación completa de matemáticas para todo el año de 7º
básico.
\end{quote}

\textbf{Problema:} la tarea es demasiado amplia; la respuesta será
superficial y poco aplicable.

\textbf{Mejor alternativa}

\begin{Shaded}
\begin{Highlighting}[]
\NormalTok{Actúa como profesora de Matemática de 7º básico en una escuela pública urbana.}

\NormalTok{Tu objetivo es proponer una planificación sintética para UNA unidad didáctica sobre proporcionalidad.}

\NormalTok{Contexto:}
\SpecialStringTok{{-} }\NormalTok{Curso: 7º básico, 38 estudiantes, niveles de logro muy diversos.}
\SpecialStringTok{{-} }\NormalTok{Asignatura: Matemática.}
\SpecialStringTok{{-} }\NormalTok{Marco de referencia: Bases curriculares chilenas.}

\NormalTok{Instrucciones:}
\SpecialStringTok{{-} }\NormalTok{Propón una secuencia de 4 a 6 clases.}
\SpecialStringTok{{-} }\NormalTok{Para cada clase, indica objetivo, actividad principal y posible instrumento de evaluación.}
\SpecialStringTok{{-} }\NormalTok{Ten en cuenta que hay acceso limitado a computadores en el aula.}

\NormalTok{Formato de salida:}
\SpecialStringTok{{-} }\NormalTok{Tabla con columnas: Clase – Objetivo – Actividad principal – Evaluación sugerida.}

\NormalTok{Criterios:}
\SpecialStringTok{{-} }\NormalTok{Usa lenguaje claro y breve.}
\SpecialStringTok{{-} }\NormalTok{No des actividades que requieran conexión permanente a internet.}
\end{Highlighting}
\end{Shaded}

\subsection{Falta de contexto sobre el curso y el
establecimiento}\label{falta-de-contexto-sobre-el-curso-y-el-establecimiento}

\textbf{Ejemplo de error}

\begin{quote}
Dame ideas de actividades para lenguaje.
\end{quote}

\textbf{Problema:} la IA no sabe si es 2º básico o 4º medio, ni el tipo
de establecimiento.

\textbf{Mejor alternativa}

\begin{Shaded}
\begin{Highlighting}[]
\NormalTok{Actúa como profesora de Lengua y Literatura de 2º medio.}

\NormalTok{Tu objetivo es proponer tres actividades de comprensión lectora para trabajar un cuento latinoamericano breve.}

\NormalTok{Contexto:}
\SpecialStringTok{{-} }\NormalTok{Liceo público científico{-}humanista, cursos numerosos (45 estudiantes).}
\SpecialStringTok{{-} }\NormalTok{Hay estudiantes con diferentes niveles de lectura y al menos dos con NEE permanentes.}
\SpecialStringTok{{-} }\NormalTok{Tiempo disponible: una hora pedagógica por actividad.}

\NormalTok{Formato de salida:}
\SpecialStringTok{{-} }\NormalTok{Lista numerada con 3 actividades.}
\SpecialStringTok{{-} }\NormalTok{Para cada actividad incluye: propósito, descripción breve, tiempo estimado y posibles ajustes para estudiantes con NEE.}

\NormalTok{Criterios:}
\SpecialStringTok{{-} }\NormalTok{Usa materiales que se puedan fotocopiar o proyectar.}
\SpecialStringTok{{-} }\NormalTok{No requieras acceso individual a computadores.}
\end{Highlighting}
\end{Shaded}

\subsection{Delegar decisiones pedagógicas clave a la
IA}\label{delegar-decisiones-pedaguxf3gicas-clave-a-la-ia}

\textbf{Ejemplo de error}

\begin{quote}
Decide tú qué contenidos son más importantes para enseñar en Historia de
Chile en 8º básico.
\end{quote}

\textbf{Problema:} la selección de contenidos es una decisión curricular
y pedagógica que corresponde al equipo docente y a las bases
curriculares, no a la IA.

\textbf{Mejor alternativa}

\begin{Shaded}
\begin{Highlighting}[]
\NormalTok{Actúa como asesor pedagógico.}

\NormalTok{Tu objetivo es ayudarme a organizar una secuencia de actividades para trabajar los contenidos de Historia de Chile de 8º básico ya definidos en las bases curriculares.}

\NormalTok{Contexto:}
\SpecialStringTok{{-} }\NormalTok{Escuela básica pública.}
\SpecialStringTok{{-} }\NormalTok{Contenidos a abordar: }\CommentTok{[}\OtherTok{PEGAR LISTA DE CONTENIDOS SELECCIONADOS POR EL DOCENTE}\CommentTok{]}\NormalTok{.}

\NormalTok{Instrucciones:}
\SpecialStringTok{{-} }\NormalTok{Sugiere una posible secuencia de actividades para abordar estos contenidos en 6 a 8 clases.}
\SpecialStringTok{{-} }\NormalTok{No agregues contenidos nuevos; solo organiza y relaciona los que entrego.}

\NormalTok{Formato de salida:}
\SpecialStringTok{{-} }\NormalTok{Lista numerada de clases con una breve descripción de la actividad principal.}

\NormalTok{Criterios:}
\SpecialStringTok{{-} }\NormalTok{Considera que el curso cuenta con recursos limitados (pizarra, cuaderno, proyector ocasional).}
\end{Highlighting}
\end{Shaded}

\subsection{Incluir información sensible de estudiantes o
colegas}\label{incluir-informaciuxf3n-sensible-de-estudiantes-o-colegas}

\textbf{Ejemplo de error}

\begin{quote}
Te copio los nombres y diagnósticos de mis estudiantes para que me digas
cómo trabajar con cada uno.
\end{quote}

\textbf{Problema:} vulnera la confidencialidad y privacidad de
estudiantes.

\textbf{Mejor alternativa (enfoque anónimo y general)}

\begin{Shaded}
\begin{Highlighting}[]
\NormalTok{Actúa como especialista en educación inclusiva.}

\NormalTok{Tu objetivo es sugerir adaptaciones generales para un curso diverso de 6º básico.}

\NormalTok{Contexto:}
\SpecialStringTok{{-} }\NormalTok{Escuela pública con programa de integración escolar (PIE).}
\SpecialStringTok{{-} }\NormalTok{En el curso hay estudiantes con dificultades específicas de aprendizaje, TDAH y discapacidad intelectual leve.}

\NormalTok{Instrucciones:}
\SpecialStringTok{{-} }\NormalTok{Propón 5 estrategias generales de adaptación de materiales y actividades para favorecer la participación de todo el curso.}
\SpecialStringTok{{-} }\NormalTok{No necesitas nombres ni datos personales de estudiantes.}

\NormalTok{Formato:}
\SpecialStringTok{{-} }\NormalTok{Lista con 5 estrategias, cada una explicada en 3 o 4 líneas.}

\NormalTok{Criterios:}
\SpecialStringTok{{-} }\NormalTok{Considera un enfoque de diseño universal para el aprendizaje (DUA).}
\SpecialStringTok{{-} }\NormalTok{Evita lenguaje técnico innecesario.}
\end{Highlighting}
\end{Shaded}

\section{\texorpdfstring{Mini-galería: ejemplos completos de
\emph{prompts}
docentes}{Mini-galería: ejemplos completos de prompts docentes}}\label{mini-galeruxeda-ejemplos-completos-de-prompts-docentes}

En esta sección se presentan algunos ejemplos ``llenos'' de
\emph{prompts} basados en las plantillas anteriores, listos para probar
y adaptar.

\subsection{Ejemplo para planificación de una
clase}\label{ejemplo-para-planificaciuxf3n-de-una-clase}

\begin{Shaded}
\begin{Highlighting}[]
\NormalTok{Actúa como profesora de Matemática de 7º básico en una escuela pública urbana de Chile.}

\NormalTok{Tu objetivo es ayudarme a planificar UNA clase de 90 minutos para introducir el concepto de proporcionalidad.}

\NormalTok{Contexto:}
\SpecialStringTok{{-} }\NormalTok{Curso: 7º básico, 40 estudiantes, niveles de logro muy diversos.}
\SpecialStringTok{{-} }\NormalTok{Recursos disponibles: pizarra, cuadernos, proyector ocasional.}
\SpecialStringTok{{-} }\NormalTok{Hay estudiantes que se distraen con facilidad y otros que avanzan rápido.}

\NormalTok{Instrucciones:}
\SpecialStringTok{{-} }\NormalTok{Propón una clase de 90 minutos.}
\SpecialStringTok{{-} }\NormalTok{Incluye: objetivo específico, actividad de inicio, desarrollo y cierre.}
\SpecialStringTok{{-} }\NormalTok{Sugiere al menos una actividad que permita que estudiantes con distinto nivel trabajen en paralelo (por ejemplo, desafíos de diferente dificultad).}
\SpecialStringTok{{-} }\NormalTok{Indica una forma sencilla de evaluar si comprendieron la idea de proporcionalidad al final de la clase.}

\NormalTok{Formato de salida:}
\SpecialStringTok{{-} }\NormalTok{Esquema en lista con secciones: Objetivo – Inicio – Desarrollo – Cierre – Evaluación.}

\NormalTok{Criterios:}
\SpecialStringTok{{-} }\NormalTok{Usa lenguaje claro, sin tecnicismos innecesarios.}
\SpecialStringTok{{-} }\NormalTok{No requieras uso individual de computadores o celulares.}
\end{Highlighting}
\end{Shaded}

\subsection{Ejemplo para evaluación y
retroalimentación}\label{ejemplo-para-evaluaciuxf3n-y-retroalimentaciuxf3n}

\begin{Shaded}
\begin{Highlighting}[]
\NormalTok{Actúa como asesor en evaluación formativa en Lengua y Literatura.}

\NormalTok{Tu objetivo es ayudarme a diseñar una breve evaluación y posibles retroalimentaciones para un ensayo argumentativo en 2º medio.}

\NormalTok{Contexto:}
\SpecialStringTok{{-} }\NormalTok{Liceo público científico{-}humanista.}
\SpecialStringTok{{-} }\NormalTok{El curso está trabajando la escritura de ensayos sobre temas de actualidad (por ejemplo, uso de celulares en el aula).}
\SpecialStringTok{{-} }\NormalTok{Quiero evaluar especialmente: claridad de la tesis, coherencia de los argumentos y uso de conectores.}

\NormalTok{Instrucciones:}
\SpecialStringTok{{-} }\NormalTok{Propón una rúbrica sencilla de 3 criterios (tesis, argumentos, conectores) con 4 niveles de logro.}
\SpecialStringTok{{-} }\NormalTok{Redacta 3 ejemplos de comentarios de retroalimentación formativa: uno para estudiantes con desempeño alto, otro para desempeño medio y otro para desempeño inicial.}

\NormalTok{Formato:}
\SpecialStringTok{{-} }\NormalTok{Primero, tabla de rúbrica.}
\SpecialStringTok{{-} }\NormalTok{Luego, lista numerada con los 3 ejemplos de retroalimentación.}

\NormalTok{Criterios:}
\SpecialStringTok{{-} }\NormalTok{Usa un lenguaje respetuoso y motivador.}
\SpecialStringTok{{-} }\NormalTok{Los comentarios deben invitar a la mejora, no solo señalar errores.}
\end{Highlighting}
\end{Shaded}

\subsection{Ejemplo para inclusión
educativa}\label{ejemplo-para-inclusiuxf3n-educativa}

\begin{Shaded}
\begin{Highlighting}[]
\NormalTok{Actúa como especialista en educación inclusiva y diseño universal para el aprendizaje.}

\NormalTok{Tu objetivo es sugerir adaptaciones de un texto de Ciencias Naturales para que pueda ser trabajado en 6º básico con estudiantes que tienen diferentes niveles de lectura.}

\NormalTok{Contexto:}
\SpecialStringTok{{-} }\NormalTok{Escuela básica pública con PIE.}
\SpecialStringTok{{-} }\NormalTok{El contenido es “la función de nutrición en los seres humanos”.}
\SpecialStringTok{{-} }\NormalTok{Tengo un texto de 2 páginas con vocabulario técnico.}

\NormalTok{Instrucciones:}
\SpecialStringTok{{-} }\NormalTok{Propón 3 formas de adaptar el texto: (1) versión simplificada, (2) uso de apoyos visuales, (3) actividad oral o manipulativa complementaria.}
\SpecialStringTok{{-} }\NormalTok{Para cada forma, explica brevemente cómo implementarla en clase en no más de 5 líneas.}

\NormalTok{Formato:}
\SpecialStringTok{{-} }\NormalTok{Lista numerada con 3 adaptaciones, cada una descrita en un párrafo.}

\NormalTok{Criterios:}
\SpecialStringTok{{-} }\NormalTok{Usa lenguaje claro.}
\SpecialStringTok{{-} }\NormalTok{No requieras recursos costosos; asume acceso a impresora y pizarra.}
\end{Highlighting}
\end{Shaded}

\subsection{Ejemplo para comunicaciones con
familias}\label{ejemplo-para-comunicaciones-con-familias}

\begin{Shaded}
\begin{Highlighting}[]
\NormalTok{Actúa como profesora jefe de 5º básico en una escuela pública.}

\NormalTok{Tu objetivo es redactar un comunicado breve para las familias, invitándolas a una reunión donde se presentará el uso responsable de herramientas de inteligencia artificial en el trabajo escolar.}

\NormalTok{Contexto:}
\SpecialStringTok{{-} }\NormalTok{La reunión será presencial en la escuela.}
\SpecialStringTok{{-} }\NormalTok{Fecha: jueves 22 de enero, a las 18:00 horas.}
\SpecialStringTok{{-} }\NormalTok{Duración aproximada: 1 hora.}

\NormalTok{Instrucciones:}
\SpecialStringTok{{-} }\NormalTok{Redacta un texto de máximo 180 palabras.}
\SpecialStringTok{{-} }\NormalTok{Usa un tono cercano, respetuoso y claro.}
\SpecialStringTok{{-} }\NormalTok{Incluye: motivo de la reunión, fecha, hora, lugar y la importancia de la asistencia.}

\NormalTok{Formato:}
\SpecialStringTok{{-} }\NormalTok{Texto continuo, listo para copiar en un correo o mensaje de WhatsApp.}

\NormalTok{Criterios:}
\SpecialStringTok{{-} }\NormalTok{Evita tecnicismos; explica “inteligencia artificial” en palabras sencillas.}
\SpecialStringTok{{-} }\NormalTok{No prometas cambios que no he mencionado.}
\end{Highlighting}
\end{Shaded}



\backmatter

% --- Cuerpo del libro (capítulos) en arábigos ---
\mainmatter
\pagestyle{scrheadings}

% --- (Opcional) Estilo del índice general ---
% \addtocontents{toc}{\protect\thispagestyle{plain}}


\end{document}
